% Created 2011-05-03 Tue 11:51
\documentclass[presentation]{beamer}
\usepackage[utf8]{inputenc}
\usepackage[T1]{fontenc}
\usepackage{fixltx2e}
\usepackage{graphicx}
\usepackage{longtable}
\usepackage{float}
\usepackage{wrapfig}
\usepackage{soul}
\usepackage{textcomp}
\usepackage{marvosym}
\usepackage{wasysym}
\usepackage{latexsym}
\usepackage{amssymb}
\usepackage{hyperref}
\tolerance=1000
\usepackage[english]{babel} \usepackage{ae,aecompl}
\usepackage{mathpazo,courier,euler} \usepackage[scaled=.95]{helvet}
\usepackage{listings}
\lstset{language=Python, basicstyle=\ttfamily\bfseries,
commentstyle=\color{red}\itshape, stringstyle=\color{darkgreen},
showstringspaces=false, keywordstyle=\color{blue}\bfseries}
\providecommand{\alert}[1]{\textbf{#1}}

\title{}
\author{FOSSEE}
\date{}

\usetheme{Warsaw}\usecolortheme{default}\useoutertheme{infolines}\setbeamercovered{transparent}
\begin{document}











\begin{frame}

\begin{center}
\textcolor{blue}{Multiple Plots}
\end{center}
\begin{center}
\includegraphics[scale=0.25]{../images/iitb-logo.png}\\
Developed by FOSSEE Team, IIT-Bombay. \\ 
Funded by National Mission on Education through ICT

MHRD, Govt. of India
\end{center}
\end{frame}
\begin{frame}
\frametitle{Objectives}
\label{sec-2}

  At the end of this tutorial you will be able to

\begin{itemize}
\item draw multiple plots which are overlaid
\item use the figure command
\item use the legend command
\item switch between the plots and perform some operations on each of them like
    saving the plots.
\item create and switch between subplots
\end{itemize}
\end{frame}
\begin{frame}
\frametitle{Question1}
\label{sec-3}


\begin{itemize}
\item Draw two plots overlaid upon each other, with the first plot
    being a parabola of the form y = 4(x \^{} 2) and the second being a
    straight line of the form y = 2x + 3 in the interval -5 to 5.
\item Use colors to differentiate between the plots and use legends to
    indicate what each plot is doing.
\end{itemize}
\end{frame}
\begin{frame}
\frametitle{Question2}
\label{sec-4}


\begin{itemize}
\item Draw a line of the form y = x as one figure and another line
    of the form y = 2x + 3.
\item Switch back to the first figure, annotate the x and y intercepts.
\item Now switch to the second figure and annotate its x and y intercepts.
    Save each of them.
\end{itemize}
\end{frame}
\begin{frame}
\frametitle{Question3}
\label{sec-5}


\begin{itemize}
\item We know that the Pressure, Volume and Temperatures are held by
    the equation PV = nRT where nR is a constant. Let us assume
    nR=0.01 Joules/Kelvin and T = 200K.
    V can be in the range from 21cc to 100cc.
\item Draw two different plots as subplots, one being the Pressure
    versus Volume plot and the other being Pressure versus Temperature
    plot.
\end{itemize}
\end{frame}
\begin{frame}
\frametitle{Summary}
\label{sec-6}

  In this tutorial, we have learnt to –

\begin{itemize}
\item draw multiple plots which are overlaid.
\item use the figure command.
\item use the legend command.
\item switch between the plots and perform some operations on each
    of them like saving the plots.
\item create subplots and to switch between them.
\end{itemize}
\end{frame}
\begin{frame}
\frametitle{Acknowledgement}
\label{sec-7}

  \begin{block}{}
  \begin{center}
  \textcolor{blue}{\Large THANK YOU!} 
  \end{center}
  \end{block}
\begin{block}{}
  \begin{center}
    For more Information, visit our website\\
    \url{http://fossee.in/}
  \end{center}  
  \end{block}
\end{frame}
\end{frame}

\end{document}