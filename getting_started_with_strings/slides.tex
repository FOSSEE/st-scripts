% Created 2010-11-10 Wed 10:46
\documentclass[presentation]{beamer}
\usepackage[latin1]{inputenc}
\usepackage[T1]{fontenc}
\usepackage{fixltx2e}
\usepackage{graphicx}
\usepackage{longtable}
\usepackage{float}
\usepackage{wrapfig}
\usepackage{soul}
\usepackage{textcomp}
\usepackage{marvosym}
\usepackage{wasysym}
\usepackage{latexsym}
\usepackage{amssymb}
\usepackage{hyperref}
\tolerance=1000
\usepackage[english]{babel} \usepackage{ae,aecompl}
\usepackage{mathpazo,courier,euler} \usepackage[scaled=.95]{helvet}
\usepackage{listings}
\lstset{language=Python, basicstyle=\ttfamily\bfseries,
commentstyle=\color{red}\itshape, stringstyle=\color{darkgreen},
showstringspaces=false, keywordstyle=\color{blue}\bfseries}
\providecommand{\alert}[1]{\textbf{#1}}

\title{}
\author{FOSSEE}
\date{}

\usetheme{Warsaw}\usecolortheme{default}\useoutertheme{infolines}\setbeamercovered{transparent}
\begin{document}











\begin{frame}
\frametitle{Outline}
\label{sec-1}
\begin{itemize}

\item Defining strings\\
\label{sec-1_1}%
\item Concatenation\\
\label{sec-1_2}%
\item Accessing individual elements\\
\label{sec-1_3}%
\item Immutability of strings\\
\label{sec-1_4}%
\end{itemize} % ends low level
\end{frame}
\begin{frame}
\frametitle{Question 1}
\label{sec-2}

  Obtain the string \texttt{\%\% -------------------- \%\%} (20 hyphens) without
  typing out all the twenty hyphens.
\end{frame}
\begin{frame}[fragile]
\frametitle{Solution 1}
\label{sec-3}

\lstset{language=Python}
\begin{lstlisting}
s = "%% " + "-"*20 + " %%"
\end{lstlisting}
\end{frame}
\begin{frame}[fragile]
\frametitle{Question 2}
\label{sec-4}

  Given a string, \texttt{s} which is \texttt{Hello World} , what is the output of::
\lstset{language=Python}
\begin{lstlisting}
s[-5] 
s[-10]
s[-15]
\end{lstlisting}
\end{frame}
\begin{frame}[fragile]
\frametitle{Solution 2}
\label{sec-5}

\lstset{language=Python}
\begin{lstlisting}
'W'
'e'
IndexError
\end{lstlisting}
\end{frame}
\begin{frame}
\frametitle{Summary}
\label{sec-6}

  In this tutorial we have learnt
\begin{itemize}
\item How to define strings
\item Different ways of defining a string
\item String concatenation and repetition
\item Accessing individual elements of the string
\item Immutability of strings
\end{itemize}

  
\end{frame}
\begin{frame}
\frametitle{Thank you!}
\label{sec-7}

  \begin{block}{}
  \begin{center}
  This spoken tutorial has been produced by the
  \textcolor{blue}{FOSSEE} team, which is funded by the 
  \end{center}
  \begin{center}
    \textcolor{blue}{National Mission on Education through \\
      Information \& Communication Technology \\ 
      MHRD, Govt. of India}.
  \end{center}  
  \end{block}
\end{frame}

\end{document}
