% Created 2011-05-16 Mon 12:57
\documentclass[presentation]{beamer}
\usepackage[latin1]{inputenc}
\usepackage[T1]{fontenc}
\usepackage{fixltx2e}
\usepackage{graphicx}
\usepackage{longtable}
\usepackage{float}
\usepackage{wrapfig}
\usepackage{soul}
\usepackage{textcomp}
\usepackage{marvosym}
\usepackage{wasysym}
\usepackage{latexsym}
\usepackage{amssymb}
\usepackage{hyperref}
\tolerance=1000
\usepackage[english]{babel} \usepackage{ae,aecompl}
\usepackage{mathpazo,courier,euler} \usepackage[scaled=.95]{helvet}
\usepackage{listings}
\lstset{language=Python, basicstyle=\ttfamily\bfseries,
commentstyle=\color{red}\itshape, stringstyle=\color{darkgreen},
showstringspaces=false, keywordstyle=\color{blue}\bfseries}
\providecommand{\alert}[1]{\textbf{#1}}

\title{}
\author{FOSSEE}
\date{}

\usetheme{Warsaw}\usecolortheme{default}\useoutertheme{infolines}\setbeamercovered{transparent}
\begin{document}











\begin{frame}

\begin{center}
\vspace{12pt}
\textcolor{blue}{\huge Getting started with Strings}
\end{center}
\vspace{18pt}
\begin{center}
\vspace{10pt}
\includegraphics[scale=0.95]{../images/fossee-logo.png}\\
\vspace{5pt}
\scriptsize Developed by FOSSEE Team, IIT-Bombay. \\ 
\scriptsize Funded by National Mission on Education through ICT\\
\scriptsize  MHRD,Govt. of India\\
\includegraphics[scale=0.30]{../images/iitb-logo.png}\\
\end{center}
\end{frame}
\begin{frame}
\frametitle{Objectives}
\label{sec-2}

  At the end of this tutorial, you will be able to, 

\begin{itemize}
\item Define strings in differnt ways.
\item Concatenate strings.
\item Print a string repeatedly.
\item Access individual elements of the string.
\item Learn immutability of strings.
\end{itemize}
\end{frame}
\begin{frame}
\frametitle{Question 1}
\label{sec-3}

  Obtain the string \verb~%% -------------------- %%~ (20 hyphens) without
  typing out all the twenty hyphens.
\end{frame}
\begin{frame}[fragile]
\frametitle{Question 2}
\label{sec-4}

  Given a string, \verb~s~ which is \verb~Hello World~ , what is the output of::
\lstset{language=Python}
\begin{lstlisting}
s[-5] 
s[-10]
s[-15]
\end{lstlisting}
\end{frame}
\begin{frame}
\frametitle{Summary}
\label{sec-5}

  In this tutorial, we have learnt,

\begin{itemize}
\item To define strings in differnt ways.
\item To concatenate strings by performing addition.
\item To repeat a string `n' number of times by doing multiplication.
\item To access individual elements of the string by using their subscripts.
\item About the immutability of strings.
\end{itemize}
\end{frame}
\begin{frame}
\frametitle{Evaluation}
\label{sec-6}


\begin{enumerate}
\item Write code to assign s, the string ``' is called the apostrophe``
\item Given strings s and t, ``s = ``Hello''`` and ``t = ``World''`` and an
   integer r, ``r = 2``. What is the output of s * r + s * t?
\item How will you change s='hello' to s='Hello'.
\begin{itemize}
\item s[ 0 ]= H
\item s[ 0 ]='H'
\item strings are immutable,hence cannot be manipulated.
\end{itemize}
\end{enumerate}
\end{frame}
\begin{frame}
\frametitle{Solutions}
\label{sec-7}


\begin{enumerate}
\item s = ``` is called the apostrophe''
\item HelloHelloWorldWorld
\item Strings are immutable,hence cannot be manipulated.
\end{enumerate}
\end{frame}
\begin{frame}

  \begin{block}{}
  \begin{center}
  \textcolor{blue}{\Large THANK YOU!} 
  \end{center}
  \end{block}
\begin{block}{}
  \begin{center}
    For more Information, visit our website\\
    \url{http://fossee.in/}
  \end{center}  
  \end{block}
 
\end{frame}

\end{document}