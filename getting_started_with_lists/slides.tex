% Created 2011-05-24 Tue 09:52
\documentclass[presentation]{beamer}
\usepackage[utf8]{inputenc}
\usepackage[T1]{fontenc}
\usepackage{fixltx2e}
\usepackage{graphicx}
\usepackage{longtable}
\usepackage{float}
\usepackage{wrapfig}
\usepackage{soul}
\usepackage{textcomp}
\usepackage{marvosym}
\usepackage{wasysym}
\usepackage{latexsym}
\usepackage{amssymb}
\usepackage{hyperref}
\tolerance=1000
\usepackage[english]{babel} \usepackage{ae,aecompl}
\usepackage{mathpazo,courier,euler} \usepackage[scaled=.95]{helvet}
\usepackage{listings}
\lstset{language=Python, basicstyle=\ttfamily\bfseries,
commentstyle=\color{red}\itshape, stringstyle=\color
{darkgreen},
showstringspaces=false, keywordstyle=\color{blue}\bfseries}
\providecommand{\alert}[1]{\textbf{#1}}

\title{}
\author{FOSSEE}
\date{2010-09-14 Tue}

\usetheme{Warsaw}\usecolortheme{default}\useoutertheme{infolines}\setbeamercovered{transparent}
\begin{document}












\begin{frame}

\begin{center}
\vspace{12pt}
\textcolor{blue}{\huge Getting started with Lists}
\end{center}
\vspace{18pt}
\begin{center}
\vspace{10pt}
\includegraphics[scale=0.95]{../images/fossee-logo.png}\\
\vspace{5pt}
\scriptsize Developed by FOSSEE Team, IIT-Bombay. \\ 
\scriptsize Funded by National Mission on Education through ICT\\
\scriptsize  MHRD,Govt. of India\\
\includegraphics[scale=0.30]{../images/iitb-logo.png}\\
\end{center}
\end{frame}
\begin{frame}
\frametitle{Objectives}
\label{sec-2}

  At the end of this tutorial, you will be able to, 

\begin{itemize}
\item Create lists
\item Access list elements
\item Append elements to lists
\item Delete elements from lists
\end{itemize}
\end{frame}
\begin{frame}
\frametitle{Exercise 1}
\label{sec-3}


\begin{itemize}
\item What happens when you do nonempty[-1].
\end{itemize}
\end{frame}
\begin{frame}
\frametitle{Exercise 2}
\label{sec-4}


\begin{itemize}
\item What is the syntax to get the element `and' 
     in the list,listinlist ?
\item How would you get `and' using negative indices?
\end{itemize}
\end{frame}
\begin{frame}[fragile]
\frametitle{Solution 2}
\label{sec-5}

  
\lstset{language=Python}
\begin{lstlisting}

listinlist[1]
listinlist[-5]
\end{lstlisting}
\end{frame}
\begin{frame}
\frametitle{Exercise 3}
\label{sec-6}



\begin{itemize}
\item Remove the third element from the list, listinlist.
\item Remove `and' from the list, listinlist.
\end{itemize}
\end{frame}
\begin{frame}[fragile]
\frametitle{Solution 3}
\label{sec-7}

\lstset{language=Python}
\begin{lstlisting}

del(listinlist[2])
listinlist.remove('and')
\end{lstlisting}
\end{frame}
\begin{frame}
\frametitle{Summary}
\label{sec-8}

  In this tutorial, we have learnt to –

\begin{itemize}
\item Create lists.
\item Access lists using their index numbers.
\item Append elements to list using the function ``append``.
\item Delete Element from lists by specifying the index number of the
    element to be deleted in the ``del`` function.
\item Delete element from list by content using ``remove`` function.
\item Find out the list length using the ``len`` function.
\end{itemize}
\end{frame}
\begin{frame}
\frametitle{Evaluation}
\label{sec-9}


\begin{enumerate}
\item How do you create an empty list?
\vspace{8pt}
\item Can you have a list inside a list ?
\vspace{8pt}
\item How would you access the end of a list without finding its length?
\end{enumerate}
\end{frame}
\begin{frame}
\frametitle{Solutions}
\label{sec-10}


\begin{enumerate}
\item empty=[]
\vspace{8pt}
\item Yes\\
list\_in\_list=[2.3,[2,4,6],'string,'all datatypes can be there']
\vspace{8pt}
\item Using negative indices\\
nonempty = ['spam', `eggs', 100, 1.234]\\
     nonempty[-1]
\end{enumerate}
\end{frame}
\begin{frame}

 \begin{block}{}
  \begin{center}
  \textcolor{blue}{\Large THANK YOU!} 
  \end{center}
  \end{block}
\begin{block}{}
  \begin{center}
    For more Information, visit our website\\
    \url{http://fossee.in/}
  \end{center}  
  \end{block}
\end{frame}

\end{document}
