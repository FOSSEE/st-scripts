% Created 2011-06-14 Tue 13:44
\documentclass[presentation]{beamer}
\usepackage[latin1]{inputenc}
\usepackage[T1]{fontenc}
\usepackage{fixltx2e}
\usepackage{graphicx}
\usepackage{longtable}
\usepackage{float}
\usepackage{wrapfig}
\usepackage{soul}
\usepackage{textcomp}
\usepackage{marvosym}
\usepackage{wasysym}
\usepackage{latexsym}
\usepackage{amssymb}
\usepackage{hyperref}
\tolerance=1000
\usepackage[english]{babel} \usepackage{ae,aecompl}
\usepackage{mathpazo,courier,euler} \usepackage[scaled=.95]{helvet}
\usepackage{listings}
\lstset{language=Python, basicstyle=\ttfamily\bfseries,
commentstyle=\color{red}\itshape, stringstyle=\color{darkgreen},
showstringspaces=false, keywordstyle=\color{blue}\bfseries}
\providecommand{\alert}[1]{\textbf{#1}}

\title{}
\author{FOSSEE}
\date{}

\usetheme{Warsaw}\usecolortheme{default}\useoutertheme{infolines}\setbeamercovered{transparent}
\begin{document}











\begin{frame}

\begin{center}
\vspace{12pt}
\textcolor{blue}{\huge Getting started with Symbolics}
\end{center}
\vspace{18pt}
\begin{center}
\vspace{10pt}
\includegraphics[scale=0.95]{../images/fossee-logo.png}\\
\vspace{5pt}
\scriptsize Developed by FOSSEE Team, IIT-Bombay. \\ 
\scriptsize Funded by National Mission on Education through ICT\\
\scriptsize  MHRD,Govt. of India\\
\includegraphics[scale=0.30]{../images/iitb-logo.png}\\
\end{center}
\end{frame}
\begin{frame}
\frametitle{Objectives}
\label{sec-2}

 At the end of this tutorial, you will be able to,


\begin{itemize}
\item Define symbolic expressions in sage.
\item Use built-in constants and functions.
\item Perform Integration, differentiation using sage.
\item Define matrices.
\item Define Symbolic functions.
\item Simplify and solve symbolic expressions and functions.
\end{itemize}
\end{frame}
\begin{frame}
\frametitle{Pre-requisite}
\label{sec-3}

  Spoken tutorial on -

\begin{itemize}
\item Getting started with Sage Notebook.
\end{itemize}
\end{frame}
\begin{frame}
\frametitle{Exercise 1}
\label{sec-4}


\begin{itemize}
\item Define the following expression as symbolic
    expression in sage.
\begin{itemize}
\item x$^2$+y$^2$
\item y$^2$-4ax
\end{itemize}
\end{itemize}
  
\end{frame}
\begin{frame}[fragile]
\frametitle{Solution 1}
\label{sec-5}

\lstset{language=Python}
\begin{lstlisting}
var('x,y')
x^2+y^2

var('a,x,y')
y^2-4*a*x
\end{lstlisting}
\end{frame}
\begin{frame}
\frametitle{Exercise 2}
\label{sec-6}


\begin{itemize}
\item Find the values of the following constants upto 6 digits  precision
\begin{itemize}
\item pi$^2$
\item euler\_gamma$^2$
\end{itemize}
\end{itemize}

\begin{itemize}
\item Find the value of the following.
\begin{itemize}
\item sin(pi/4)
\item ln(23)
\end{itemize}
\end{itemize}
\end{frame}
\begin{frame}[fragile]
\frametitle{Solution 2}
\label{sec-7}

\lstset{language=Python}
\begin{lstlisting}
n(pi^2,digits=6)
n(sin(pi/4))
n(log(23,e))
\end{lstlisting}
\end{frame}
\begin{frame}
\frametitle{Exercise 3}
\label{sec-8}


\begin{itemize}
\item Define the piecewise function\\ 
   f(x)=3x+2 
   when x is in the closed interval 0 to 4\\
   f(x)=4x$^2$
   between 4 to 6.
\vspace{4pt}
\item Sum  of 1/(n$^2$-1) where n ranges from 1 to infinity.
\end{itemize}
\end{frame}
\begin{frame}[fragile]
\frametitle{Solution 3}
\label{sec-9}

\lstset{language=Python}
\begin{lstlisting}
var('x') 
h(x)=3*x+2 
g(x)= 4*x^2
f=Piecewise([[(0,4),h(x)],[(4,6),g(x)]],x)
f
\end{lstlisting}

\lstset{language=Python}
\begin{lstlisting}
var('n')
f=1/(n^2-1) 
sum(f(n), n, 1, oo)
\end{lstlisting}
\end{frame}
\begin{frame}
\frametitle{Exercise 4}
\label{sec-10}


\begin{itemize}
\item Differentiate the following.
\begin{itemize}
\item sin(x$^3$)+log(3x), to the second order
\item x$^5$*log(x$^7$), to the fourth order
\end{itemize}
\vspace{4pt}
\item Integrate the given expression
\begin{itemize}
\item x*sin(x$^2$)
\end{itemize}
\vspace{4pt}
\item Find x
\begin{itemize}
\item cos(x$^2$)-log(x)=0
\item Does the equation have a root between 1,2.
\end{itemize}
\end{itemize}
\end{frame}
\begin{frame}[fragile]
\frametitle{Solution 4}
\label{sec-11}

\lstset{language=Python}
\begin{lstlisting}
var('x')
f(x)= x^5*log(x^7) 
diff(f(x),x,5)

var('x')
integral(x*sin(x^2),x) 

var('x')
f=cos(x^2)-log(x)
find_root(f(x)==0,1,2)
\end{lstlisting}
\end{frame}
\begin{frame}
\frametitle{Exercise 5}
\label{sec-12}


\begin{itemize}
\item Find the determinant and inverse of 

      A=[[x,0,1][y,1,0][z,0,y]]
\end{itemize}
\end{frame}
\begin{frame}[fragile]
\frametitle{Solution 5}
\label{sec-13}

\lstset{language=Python}
\begin{lstlisting}
var('x,y,z')
A=matrix([[x,0,1],[y,1,0],[z,0,y]])
A.det()
A.inverse()
\end{lstlisting}
\end{frame}
\begin{frame}
\frametitle{Summary}
\label{sec-14}

In this tutorial, we have learnt to,


\begin{itemize}
\item Define symbolic expression and functions using the method ``var``.
\item Use built-in constants like pi,e,oo and functions like 
   sum,sin,cos,log,exp and many more.
\item Use <Tab> to see the documentation of a function.
\item Do simple calculus using functions like --
\begin{itemize}
\item diff()--to find a differential of a function
\item integral()--to integrate an expression
\item simplify--to simplify complicated expression.
\end{itemize}
\item Substitute values in expressions using ``substitute`` function.
\item Create symbolic matrices and perform operations on them like --
\begin{itemize}
\item det()--to find out the determinant of a matrix
\item inverse()--to find out the inverse of a matrix.
\end{itemize}
\end{itemize}
\end{frame}
\begin{frame}
\frametitle{Evaluation}
\label{sec-15}


\begin{enumerate}
\item How do you define a name `y' as a symbol?
\vspace{8pt}
\item Get the value of pi upto precision 5 digits using sage?
\vspace{8pt}
\item Find third order differential function of

   f(x)=sin(x$^2$)+exp(x$^3$)
\end{enumerate}
\end{frame}
\begin{frame}
\frametitle{Solutions}
\label{sec-16}


\begin{enumerate}
\item var(`y')
\vspace{8pt}
\item n(pi,5)
\vspace{8pt}
\item diff(f(x),x,3)
\end{enumerate}
\end{frame}
\begin{frame}

  \begin{block}{}
  \begin{center}
  \textcolor{blue}{\Large THANK YOU!} 
  \end{center}
  \end{block}
\begin{block}{}
  \begin{center}
    For more Information, visit our website\\
    \url{http://fossee.in/}
  \end{center}  
  \end{block}
\end{frame}

\end{document}