\documentclass[11pt,twocolumn]{article}
\usepackage{graphicx}
\newenvironment{enumcpt}{\begin{enumerate} \topsep 0pt \partopsep 0pt 
                        \parsep 0pt
                        \itemsep 0pt \leftmargin -1in \rightmargin 0pt
                        }{\end{enumerate}}
\begin{document}
\title{Instruction sheet for spoken tutorials for Sage}
\author{Fossee Team}
\maketitle
\section*{Introduction }
{This document is an instruction sheet for Sage spoken tutorials.}
\setlength{\columnsep}{15pt }
\begin{enumerate}
  \item You have been given a set of modules of Sage spoken tutorials.
  \item You will typically do one module at a time.
  \item These tutorials are meant for Linux environment.
  \item You should follow the spoken tutorials and do all the assignments mentioned in them.
   \item If you find it difficult at the first go, then consider listening to the video tutorial more than once and then re-do the assignments.
   \item You must go through the Spoken tutorials in the following modular order of sequence. 
  

\end{enumerate}

\section*{General Instructions}
{These instructions are to be followed for each module.}
\setlength{\columnsep}{15pt }
\begin{enumerate}
	  
	
	\item Please copy the \emph{Sage\_Workshop} folder onto your desktop.
     \item Right click on \emph{module-name.ogv}, point cursor on \emph{Open With} and select \emph{VLC Media Player}.
  \item Please follow the tutorial as shown in video.
  \item You can do a hands-on on the terminal simultaneously with the ongoing spoken tutorial by pausing the video as and when necessary.
  \item Open up a terminal by by pressing \emph{Ctrl-Alt-t} keys simultaneously.
  \item Pause the video wherever there is an assignment and only after finishing it, continue with the video.
  \item After finishing all the assignments please proceed to next module. 

\end{enumerate}


\section{Sage}
Go through the general instructions as stated above before starting each module.\\
This module introduces us to a mathematical software called \emph{Sage}.\\

Instructions for this module:\\
If Sage is not installed in your system then visit this website to know how to install it:
\begin{enumerate}
\item http://sagemath.org/doc/tutorial/introduction.html
\item http://www.sagemath.org/download-linux.html
\end{enumerate}
Then, follow the below set of instructions:
\begin{enumerate}
 \item Type \emph{sage} on your terminal to call the interpreter. 
 \item Type notebook on the sage prompt to start the notebook server.
 \item Enter the admin username and password to login. Use admin as the username and password would be what you entered when you started the notebook server. 
 \item If the server doesn't start then check the terminal and restart it again. 
\end{enumerate}

\subsection{Getting started with sage notebook.}

Tutorial required: \\getting\_started\_with\_sage\_notebook.ogv \\

This video teaches us what are sage notebooks, how to start a sage shell, create new worksheets, know about the menu options and cells in the worksheets, evaluate the cells, create and delete shells and navigate them, make annotations in the worksheet, usage of tab-completion, usage of code in the cells and usage of off-line help. 

\subsection{Getting started with symbolics.}

Tutorial required: \\ Getting\_started\_with\_symbolics.ogv \\

This video teaches us how to define symbolic expressions in sage, use built-in constants and functions, perform integration, differentiation using sage, define matrices, define symbolic functions, and simplifying symbolic expressions and functions. 
\subsection{Using Sage.}
Tutorial required:\\ using\_sage.ogv \\

This video teaches us how to use Sage in a range of things, know the functions used for calculus and learn about graph theory and number theory using Sage. 
\subsection{Sage for teaching.}
Tutorial required: \\ using\_sage\_to\_teach.ogv \\
This video teaches us how to use \textbf{@interact} feature of Sage for better documentation, and share, publish and edit Sage worksheets for collaborative learning. \\ \\

\end{document}
