% Created 2010-11-09 Tue 17:14
\documentclass[presentation]{beamer}
\usepackage[latin1]{inputenc}
\usepackage[T1]{fontenc}
\usepackage{fixltx2e}
\usepackage{graphicx}
\usepackage{longtable}
\usepackage{float}
\usepackage{wrapfig}
\usepackage{soul}
\usepackage{textcomp}
\usepackage{marvosym}
\usepackage{wasysym}
\usepackage{latexsym}
\usepackage{amssymb}
\usepackage{hyperref}
\tolerance=1000
\usepackage[english]{babel} \usepackage{ae,aecompl}
\usepackage{mathpazo,courier,euler} \usepackage[scaled=.95]{helvet}
\usepackage{listings}
\lstset{language=Python, basicstyle=\ttfamily\bfseries,
commentstyle=\color{red}\itshape, stringstyle=\color{darkgreen},
showstringspaces=false, keywordstyle=\color{blue}\bfseries}
\providecommand{\alert}[1]{\textbf{#1}}

\title{Sets}
\author{FOSSEE}
\date{}

\usetheme{Warsaw}\usecolortheme{default}\useoutertheme{infolines}\setbeamercovered{transparent}
\begin{document}

\maketitle









\begin{frame}
\frametitle{Outline}
\label{sec-1}

\begin{itemize}
\item Defining Sets
\item Operations on sets
\end{itemize}
\end{frame}
\begin{frame}
\frametitle{Question 1}
\label{sec-2}

  Given a list of marks, \texttt{marks = [20, 23, 22, 23, 20, 21, 23]} list
  all the duplicates
\end{frame}
\begin{frame}[fragile]
\frametitle{Solution 1}
\label{sec-3}

\lstset{language=Python}
\begin{lstlisting}
marks = [20, 23, 22, 23, 20, 21, 23] 
marks_set = set(marks)
for mark in marks_set:
    marks.remove(mark)

# we are now left with only duplicates
# in the list marks
duplicates = set(marks)
\end{lstlisting}
\end{frame}
\begin{frame}
\frametitle{Summary}
\label{sec-4}

  You should now be able to --
\begin{itemize}
\item make sets from lists
\item input sets directly
\item perform operations like union, intersection, symmetric difference
\item check if a subset of another
\item check containership, length and other properties similar to lists
\end{itemize}
\end{frame}
\begin{frame}
\frametitle{Thank you!}
\label{sec-5}

  \begin{block}{}
  \begin{center}
  This spoken tutorial has been produced by the
  \textcolor{blue}{FOSSEE} team, which is funded by the 
  \end{center}
  \begin{center}
    \textcolor{blue}{National Mission on Education through \\
      Information \& Communication Technology \\ 
      MHRD, Govt. of India}.
  \end{center}  
  \end{block}
\end{frame}

\end{document}
