% Created 2011-07-07 Thu 11:19
\documentclass[presentation]{beamer}
\usepackage[latin1]{inputenc}
\usepackage[T1]{fontenc}
\usepackage{fixltx2e}
\usepackage{graphicx}
\usepackage{longtable}
\usepackage{float}
\usepackage{wrapfig}
\usepackage{soul}
\usepackage{textcomp}
\usepackage{marvosym}
\usepackage{wasysym}
\usepackage{latexsym}
\usepackage{amssymb}
\usepackage{hyperref}
\tolerance=1000
\usepackage[english]{babel} \usepackage{ae,aecompl}
\usepackage{mathpazo,courier,euler} \usepackage[scaled=.95]{helvet}
\usepackage{listings}
\lstset{language=Python, basicstyle=\ttfamily\bfseries,
commentstyle=\color{red}\itshape, stringstyle=\color{darkgreen},
showstringspaces=false, keywordstyle=\color{blue}\bfseries}
\providecommand{\alert}[1]{\textbf{#1}}

\title{}
\author{FOSSEE}
\date{}

\usetheme{Warsaw}\usecolortheme{default}\useoutertheme{infolines}\setbeamercovered{transparent}
\begin{document}











\begin{frame}

\begin{center}
\vspace{12pt}
\textcolor{blue}{\huge Sets}
\end{center}
\vspace{18pt}
\begin{center}
\vspace{10pt}
\includegraphics[scale=0.95]{../images/fossee-logo.png}\\
\vspace{5pt}
\scriptsize Developed by FOSSEE Team, IIT-Bombay. \\ 
\scriptsize Funded by National Mission on Education through ICT\\
\scriptsize  MHRD,Govt. of India\\
\includegraphics[scale=0.30]{../images/iitb-logo.png}\\
\end{center}
\end{frame}
\begin{frame}
\frametitle{Objectives}
\label{sec-2}

  At the end of this tutorial, you will be able to, 


\begin{itemize}
\item Create sets from lists.
\item Perform union, intersection and symmetric difference operations.
\item Check if a set is a subset of other.
\item Understand various similarities with lists like length and containership.
\end{itemize}
\end{frame}
\begin{frame}
\frametitle{Pre-requisite}
\label{sec-3}

Spoken tutorial on -
\begin{itemize}
\item Getting started with Lists
\end{itemize}
\end{frame}
\begin{frame}
\frametitle{Exercise 1}
\label{sec-4}


\begin{itemize}
\item Given a list of marks,
\end{itemize}
 ~~~~~~~~~~~~\verb~marks = [20, 23, 22, 23, 20, 21, 23]~ \\
\vspace{8pt} 
 ~~~~~~~~List all the duplicates.
\end{frame}
\begin{frame}
\frametitle{Summary}
\label{sec-5}

 In this tutorial, we have learnt to,


\begin{itemize}
\item Make sets from lists.
\item Perform union, intersection and symmetric difference operations.
   by using the operators `|', `\&' and `\textasciicircum'  respectively.
\item Check if a set is a subset of other using the `<'and `<=' operators.
\item Understand various similarities with lists like length and containership.
\end{itemize}
\end{frame}
\begin{frame}
\frametitle{Evaluation}
\label{sec-6}


\begin{enumerate}
\item If \verb~a = [1, 1, 2, 3, 3, 5, 5, 8]~. What is set(a)?
\vspace{3pt}
\begin{itemize}
\item set([1, 1, 2, 3, 3, 5, 5, 8])
\item set([1, 2, 3, 5, 8])
\item set([1, 2, 3, 3, 5, 5])
\item Error
\end{itemize}
\vspace{6pt}
\item Given,  \verb~odd = set([1, 3, 5, 7, 9])~ and \verb~squares = set([1, 4, 9, 16])~. \\
    How do you find the symmetric difference of these two sets?
\vspace{15pt}
\item ``a'' is a set. how do you check if a variable ``b'' exists in ``a''?
\end{enumerate}
\end{frame}
\begin{frame}
\frametitle{Solutions}
\label{sec-7}


\begin{enumerate}
\item set([1, 2, 3, 5, 8])
\vspace{12pt}
\item odd \^{} squares
\vspace{12pt}
\item b in a
\end{enumerate}
\end{frame}
\begin{frame}

  \begin{block}{}
  \begin{center}
  \textcolor{blue}{\Large THANK YOU!} 
  \end{center}
  \end{block}
\begin{block}{}
  \begin{center}
    For more Information, visit our website\\
    \url{http://fossee.in/}
  \end{center}  
  \end{block}
\end{frame}

\end{document}