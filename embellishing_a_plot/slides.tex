% Created 2010-12-18 Sat 12:25
\documentclass[presentation]{beamer}
\usepackage[latin1]{inputenc}
\usepackage[T1]{fontenc}
\usepackage{fixltx2e}
\usepackage{graphicx}
\usepackage{longtable}
\usepackage{float}
\usepackage{wrapfig}
\usepackage{soul}
\usepackage{t1enc}
\usepackage{textcomp}
\usepackage{marvosym}
\usepackage{wasysym}
\usepackage{latexsym}
\usepackage{amssymb}
\usepackage{hyperref}
\tolerance=1000
\usepackage[english]{babel} \usepackage{ae,aecompl}
\usepackage{mathpazo,courier,euler} \usepackage[scaled=.95]{helvet}
\usepackage{listings}
\lstset{language=Python, basicstyle=\ttfamily\bfseries,
commentstyle=\color{red}\itshape, stringstyle=\color{darkgreen},
showstringspaces=false, keywordstyle=\color{blue}\bfseries}
\providecommand{\alert}[1]{\textbf{#1}}

\title{Embellishing a Plot}
\author{FOSSEE}
\date{}

\usetheme{Warsaw}\usecolortheme{default}\useoutertheme{infolines}\setbeamercovered{transparent}
\begin{document}

\maketitle









\begin{frame}
\frametitle{Outline}
\label{sec-1}

\begin{itemize}
\item Modifying the color, line style \& linewidth of a plot
\item Adding a title to the plot (with embedded \LaTeX{})
\item Labelling the axes
\item Annotating the plot
\item Setting the limits of axes.
\end{itemize}
\end{frame}
\begin{frame}
\frametitle{Question 1}
\label{sec-2}

  Plot sin(x) in blue colour and with linewidth as 3
\end{frame}
\begin{frame}[fragile]
\frametitle{Solution 1}
\label{sec-3}

\begin{verbatim}
In []: clf()
In []: plot(x, sin(x), 'b', linewidth=3)
\end{verbatim}
\end{frame}
\begin{frame}
\frametitle{Question 2}
\label{sec-4}

  Plot the sine curve with green filled circles.
\end{frame}
\begin{frame}[fragile]
\frametitle{Solution 2}
\label{sec-5}

\begin{verbatim}
In []: clf()
In []: plot(x, cos(x), 'go')
\end{verbatim}
\end{frame}
\begin{frame}
\frametitle{Question 3}
\label{sec-6}

  Plot the curve of x vs tan(x) in red dashed line and linewidth 3
\end{frame}
\begin{frame}[fragile]
\frametitle{Solution 3}
\label{sec-7}

\begin{verbatim}
In []: clf()
In []: plot(x, cos(x), 'r--')
\end{verbatim}
\end{frame}
\begin{frame}
\frametitle{Question 4}
\label{sec-8}

  Change the title of the figure such that the whole title is
  formatted in LaTex style
\end{frame}
\begin{frame}[fragile]
\frametitle{Solution 4}
\label{sec-9}

\begin{verbatim}
In []: title("$Parabolic function -x^2+4x-5$")
\end{verbatim}
\end{frame}
\begin{frame}
\frametitle{Question 5}
\label{sec-10}

  Set the x and y labels as ``x'' and ``f(x)'' in LaTex style.
\end{frame}
\begin{frame}[fragile]
\frametitle{Solution 5}
\label{sec-11}

\begin{verbatim}
In []: xlabel("$x$")
In []: yalbel("$f(x)$")
\end{verbatim}
\end{frame}
\begin{frame}
\frametitle{Question 6}
\label{sec-12}

  Make an annotation called ``root'' at the point (-4, 0). What happens
  to the first annotation?
\end{frame}
\begin{frame}[fragile]
\frametitle{Solution 6}
\label{sec-13}

\begin{verbatim}
In []: annotate("root", xy=(-4,0))
\end{verbatim}
\end{frame}
\begin{frame}
\frametitle{Question 7}
\label{sec-14}

  Set the limits of axes such that the area of interest is the
  rectangle (-1, -15) and (3, 0)
\end{frame}
\begin{frame}[fragile]
\frametitle{Solution 7}
\label{sec-15}

\begin{verbatim}
In []: xlim(-1, 3)
In []: ylim(-15, 0)
\end{verbatim}
\end{frame}
\begin{frame}
\frametitle{Summary}
\label{sec-16}

\begin{itemize}
\item Modifying the attributes of plot by passing additional arguments
\item How to add title
\item How to incorporate \LaTeX{} style formatting
\item How to label x and y axes
\item How to add annotations
\item How to set the limits of axes
\end{itemize}
\end{frame}
\begin{frame}
\frametitle{Thank you!}
\label{sec-17}

  \begin{block}{}
  \begin{center}
  This spoken tutorial has been produced by the
  \textcolor{blue}{FOSSEE} team, which is funded by the 
  \end{center}
  \begin{center}
    \textcolor{blue}{National Mission on Education through \\
      Information \& Communication Technology \\ 
      MHRD, Govt. of India}.
  \end{center}  
  \end{block}
\end{frame}

\end{document}
