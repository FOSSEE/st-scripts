% Created 2011-06-20 Mon 16:04
\documentclass[presentation]{beamer}
\usepackage[latin1]{inputenc}
\usepackage[T1]{fontenc}
\usepackage{fixltx2e}
\usepackage{graphicx}
\usepackage{longtable}
\usepackage{float}
\usepackage{wrapfig}
\usepackage{soul}
\usepackage{textcomp}
\usepackage{marvosym}
\usepackage{wasysym}
\usepackage{latexsym}
\usepackage{amssymb}
\usepackage{hyperref}
\tolerance=1000
\usepackage[english]{babel} \usepackage{ae,aecompl}
\usepackage{mathpazo,courier,euler} \usepackage[scaled=.95]{helvet}
\usepackage{listings}
\lstset{language=Python, basicstyle=\ttfamily\bfseries,
commentstyle=\color{red}\itshape, stringstyle=\color{darkgreen},
showstringspaces=false, keywordstyle=\color{blue}\bfseries}
\providecommand{\alert}[1]{\textbf{#1}}

\title{}
\author{FOSSEE}
\date{2010-09-14 Tue}

\usetheme{Warsaw}\usecolortheme{default}\useoutertheme{infolines}\setbeamercovered{transparent}
\begin{document}











\begin{frame}

\begin{center}
\vspace{12pt}
\textcolor{blue}{\huge Basic Datatypes and Operators}
\end{center}
\vspace{18pt}
\begin{center}
\vspace{10pt}
\includegraphics[scale=0.95]{../images/fossee-logo.png}\\
\vspace{5pt}
\scriptsize Developed by FOSSEE Team, IIT-Bombay. \\ 
\scriptsize Funded by National Mission on Education through ICT\\
\scriptsize  MHRD,Govt. of India\\
\includegraphics[scale=0.30]{../images/iitb-logo.png}\\
\end{center}
\end{frame}
\begin{frame}
\frametitle{Objectives}
\label{sec-2}

 At the end of the tutorial,you will be able to, 
\begin{itemize}

\item Datatypes in Python
\label{sec-2_1}%
\begin{itemize}

\item Numbers\\
\label{sec-2_1_1}%
\item Boolean\\
\label{sec-2_1_2}%
\item Sequence\\
\label{sec-2_1_3}%
\end{itemize} % ends low level

\item Operators in Python
\label{sec-2_2}%
\begin{itemize}

\item Arithmetic Operators\\
\label{sec-2_2_1}%
\item Boolean Operators\\
\label{sec-2_2_2}%
\end{itemize} % ends low level

\item Python Sequence Datatypes
\label{sec-2_3}%
\begin{itemize}

\item list\\
\label{sec-2_3_1}%
\item string\\
\label{sec-2_3_2}%
\item tuple\\
\label{sec-2_3_3}%
\end{itemize} % ends low level
\end{itemize} % ends low level
\end{frame}
\begin{frame}
\frametitle{Numbers}
\label{sec-3}


\begin{itemize}
\item int
\item float
\item complex
\end{itemize}
\end{frame}
\begin{frame}
\frametitle{Exercise 1}
\label{sec-4}


\begin{itemize}
\item Find the absolute value of 3+4j
\end{itemize}
\end{frame}
\begin{frame}
\frametitle{Exercise 2}
\label{sec-5}


\begin{itemize}
\item What is the datatype of number 999999999999999999? Is it
\end{itemize}
not int?
\end{frame}
\begin{frame}
\frametitle{Solution 2}
\label{sec-6}


\begin{itemize}
\item Long
\item Large integers numbers are internally stored in python as Long
    datatype.
\end{itemize}
\end{frame}
\begin{frame}
\frametitle{Exercise 3}
\label{sec-7}


\begin{itemize}
\item Using python find sqaure root of 3?
\end{itemize}
\end{frame}
\begin{frame}
\frametitle{Solution 3}
\label{sec-8}



\begin{itemize}
\item 3**0.5
\end{itemize}
\end{frame}
\begin{frame}
\frametitle{Exercise 4}
\label{sec-9}


\begin{itemize}
\item Is 3**1/2 and 3**0.5 same?
\end{itemize}
\end{frame}
\begin{frame}
\frametitle{Sequence Data types}
\label{sec-10}
\begin{itemize}

\item Properties\\
\label{sec-10_1}%
\begin{itemize}
\item Data in Sequence
\item Accessed using Index
\end{itemize}

\item Type\\
\label{sec-10_2}%
\begin{itemize}
\item list
\item String
\item Tuple
\end{itemize}

\end{itemize} % ends low level
\end{frame}
\begin{frame}
\frametitle{Exercise 5}
\label{sec-11}


\begin{itemize}
\item Check if 3 is an element of the list [1,7,5,3,4].\\ In case it is,
\end{itemize}
change it to 21.
\end{frame}
\begin{frame}
\frametitle{Exercise 6}
\label{sec-12}


\begin{itemize}
\item Convert the string \~{}``Elizabeth is queen of england''\~{} to \~{}``Elizabeth is
\end{itemize}
queen''\~{}
\end{frame}
\begin{frame}
\frametitle{Summary}
\label{sec-13}

  In this tutorial, we have learnt to,


\begin{itemize}
\item Understand the number Datatypes -- integer,float and complex.
\item Know the boolean datatype and operators -- +, *, /, **, \% .
\item use the sequence data types -- List,String and Tuple.
\item Slice sequences by using the row and column numbers.
\item Split and join a list using ``split()'' and ``join()'' function respectively.
\item Convert to string to tuple and vice-versa.
\end{itemize}
\end{frame}
\begin{frame}
\frametitle{Evaluation}
\label{sec-14}


\begin{enumerate}
\item What is the major diffence between tuples and lists?
\vspace{12pt}
\item Split this string on whitespaces
\begin{itemize}
\item string=``Split this string on whitespaces''
\end{itemize}
\end{enumerate}
\end{frame}
\begin{frame}
\frametitle{Solutions}
\label{sec-15}


\begin{enumerate}
\item Tuples are immutable while lists are not.
\vspace{12pt}
\item string.split()
\end{enumerate}
\end{frame}
\begin{frame}

  \begin{block}{}
  \begin{center}
  \textcolor{blue}{\Large THANK YOU!} 
  \end{center}
  \end{block}
\begin{block}{}
  \begin{center}
    For more Information, visit our website\\
    \url{http://fossee.in/}
  \end{center}  
  \end{block}
\end{frame}

\end{document}