% Created 2010-10-13 Wed 17:08
\documentclass[presentation]{beamer}
\usetheme{Warsaw}\useoutertheme{infolines}\usecolortheme{default}\setbeamercovered{transparent}
\usepackage[latin1]{inputenc}
\usepackage[T1]{fontenc}
\usepackage{graphicx}
\usepackage{longtable}
\usepackage{float}
\usepackage{wrapfig}
\usepackage{soul}
\usepackage{amssymb}
\usepackage{hyperref}


\title{Plotting Data }
\author{FOSSEE}
\date{2010-09-14 Tue}

\begin{document}

\maketitle






\begin{frame}
\frametitle{Tutorial Plan}
\label{sec-1}
\begin{itemize}

\item Datatypes in Python\\
\label{sec-1.1}%
\item Operators in Python\\
\label{sec-1.2}%
\end{itemize} % ends low level
\end{frame}
\begin{frame}
\frametitle{Numbers}
\label{sec-2}
\begin{itemize}

\item Integers\\
\label{sec-2.1}%
\item Float\\
\label{sec-2.2}%
\item Complex\\
\label{sec-2.3}%
\end{itemize} % ends low level
\end{frame}
\begin{frame}
\frametitle{Boolean}
\label{sec-3}
\begin{itemize}

\item True\\
\label{sec-3.1}%
\item False\\
\label{sec-3.2}%
\end{itemize} % ends low level
\end{frame}
\begin{frame}
\frametitle{Sequence Data types}
\label{sec-4}
\begin{itemize}

\item Data in Sequence\\
\label{sec-4.1}%
\item Accessed using Index
\label{sec-4.2}%
\begin{itemize}

\item list\\
\label{sec-4.2.1}%
\item String\\
\label{sec-4.2.2}%
\item Tuple\\
\label{sec-4.2.3}%
\end{itemize} % ends low level
\end{itemize} % ends low level
\end{frame}
\begin{frame}
\frametitle{All are Strings}
\label{sec-5}
\begin{itemize}

\item k='Single quote'\\
\label{sec-5.1}%
\item l="Double quote contain's single quote"\\
\label{sec-5.2}%
\item m='''"Contain's both"'''\\
\label{sec-5.3}%
\end{itemize} % ends low level
\end{frame}
\begin{frame}
\frametitle{Summary}
\label{sec-6}
\begin{itemize}

\item a=73\\
\label{sec-6.1}%
\item b=3.14\\
\label{sec-6.2}%
\item c=3+4j\\
\label{sec-6.3}%
\end{itemize} % ends low level
\end{frame}
\begin{frame}
\frametitle{Summary Contd.}
\label{sec-7}
\begin{itemize}

\item t=True\\
\label{sec-7.1}%
\item f=False\\
\label{sec-7.2}%
\item t and f\\
\label{sec-7.3}%
\end{itemize} % ends low level
\end{frame}
\begin{frame}
\frametitle{Summary Contd.}
\label{sec-8}
\begin{itemize}

\item l= [2,1,4,3]\\
\label{sec-8.1}%
\item s='hello'\\
\label{sec-8.2}%
\item tu=(1,2,3,4)\\
\label{sec-8.3}%
\end{itemize} % ends low level
\end{frame}
\begin{frame}
\frametitle{Summary Contd.}
\label{sec-9}
\begin{itemize}

\item tu[-1]\\
\label{sec-9.1}%
\item s[1:-1]\\
\label{sec-9.2}%
\end{itemize} % ends low level
\end{frame}
\begin{frame}
\frametitle{Summary Contd.}
\label{sec-10}
\begin{itemize}

\item Sorted(l)\\
\label{sec-10.1}%
\item reversed(s)\\
\label{sec-10.2}%
\end{itemize} % ends low level
\end{frame}

\end{document}
