% Created 2010-11-07 Sun 15:18
\documentclass[presentation]{beamer}
\usepackage[latin1]{inputenc}
\usepackage[T1]{fontenc}
\usepackage{fixltx2e}
\usepackage{graphicx}
\usepackage{longtable}
\usepackage{float}
\usepackage{wrapfig}
\usepackage{soul}
\usepackage{t1enc}
\usepackage{textcomp}
\usepackage{marvosym}
\usepackage{wasysym}
\usepackage{latexsym}
\usepackage{amssymb}
\usepackage{hyperref}
\tolerance=1000
\usepackage[english]{babel} \usepackage{ae,aecompl}
\usepackage{mathpazo,courier,euler} \usepackage[scaled=.95]{helvet}
\usepackage{listings}
\lstset{language=Python, basicstyle=\ttfamily\bfseries,
commentstyle=\color{red}\itshape, stringstyle=\color{darkgreen},
showstringspaces=false, keywordstyle=\color{blue}\bfseries}
\providecommand{\alert}[1]{\textbf{#1}}

\title{Getting started with arrays}
\author{FOSSEE}
\date{}

\usetheme{Warsaw}\usecolortheme{default}\useoutertheme{infolines}\setbeamercovered{transparent}
\begin{document}

\maketitle









\begin{frame}
\frametitle{Outline}
\label{sec-1}

\begin{itemize}
\item Arrays

\begin{itemize}
\item why arrays over lists
\end{itemize}

\item Creating arrays
\item Array operations
\end{itemize}
\end{frame}
\begin{frame}
\frametitle{Overview of Arrays}
\label{sec-2}

\begin{itemize}
\item Arrays are homogeneous data structures.

\begin{itemize}
\item elements have to the same data type
\end{itemize}

\item Arrays are faster compared to lists

\begin{itemize}
\item at least \emph{80-100 times} faster than lists
\end{itemize}

\end{itemize}
\end{frame}
\begin{frame}[fragile]
\frametitle{Creating Arrays}
\label{sec-3}

\begin{itemize}
\item Creating a 1-dimensional array
\end{itemize}

\begin{verbatim}
   In []: a1 = array([1, 2, 3, 4])
\end{verbatim}

  \texttt{[1, 2, 3, 4]} is a list.
\end{frame}
\begin{frame}[fragile]
\frametitle{Creating two-dimensional array}
\label{sec-4}

\begin{itemize}
\item Creating a 2-dimensional array
\end{itemize}

\begin{verbatim}
   In []: a2 = array([[1,2,3,4],[5,6,7,8]])
\end{verbatim}

  here we convert a list of lists to an array making a 2-d array.
\begin{itemize}
\item Easier method of creating array with consecutive elements.
\end{itemize}

\begin{verbatim}
   In []: ar = arange(1,9)
\end{verbatim}
\end{frame}
\begin{frame}[fragile]
\frametitle{\texttt{reshape()} method}
\label{sec-5}

\begin{itemize}
\item To reshape an array
\end{itemize}

\begin{verbatim}
   In []: ar.reshape(2, 4)
   In []: ar.reshape(4, 2)
   In []: ar = ar.reshape(2, 4)
\end{verbatim}
\end{frame}
\begin{frame}[fragile]
\frametitle{Creating \texttt{array} from \texttt{list}.}
\label{sec-6}

\begin{itemize}
\item \texttt{array()} method accepts list as argument
\item Creating a list
\begin{verbatim}
    In []: l1 = [1, 2, 3, 4]
\end{verbatim}

\item Creating an array
\begin{verbatim}
     In []: a3 = array(l1)
\end{verbatim}

\end{itemize}
\end{frame}
\begin{frame}
\frametitle{Exercise 1}
\label{sec-7}

  Create a 3-dimensional array of the order (2, 2, 4).
\end{frame}
\begin{frame}[fragile]
\frametitle{\texttt{.shape} of array}
\label{sec-8}

\begin{itemize}
\item \texttt{.shape}
    To find the shape of the array
\begin{verbatim}
     In []: a1.shape
\end{verbatim}

\item \texttt{.shape}
    returns a tuple of shape
\end{itemize}
\end{frame}
\begin{frame}
\frametitle{Exercise 2}
\label{sec-9}

  Find out the shape of the other arrays(a2, a3, ar) that we have created.
\end{frame}
\begin{frame}[fragile]
\frametitle{Homogeneous data}
\label{sec-10}

\begin{itemize}
\item All elements in array should be of same type
\begin{verbatim}
     In []: a4 = array([1,2,3,'a string'])
\end{verbatim}

\end{itemize}
\end{frame}
\begin{frame}[fragile]
\frametitle{Implicit type casting}
\label{sec-11}

\begin{verbatim}
    In []: a4
\end{verbatim}

    All elements are type casted to string type
\end{frame}
\begin{frame}
\frametitle{\texttt{identity()}, \texttt{zeros()} methods}
\label{sec-12}

\begin{itemize}
\item \texttt{identity(n)}
    Creates an identity matrix, a square matrix of order (n, n) with diagonal elements 1 and others 0.
\item \texttt{zeros((m, n))}
    Creates an \texttt{m X n} matrix with all elements 0.
\end{itemize}
\end{frame}
\begin{frame}
\frametitle{Learning exercise}
\label{sec-13}

\begin{itemize}
\item Find out about

\begin{itemize}
\item \texttt{zeros\_like()}
\item \texttt{ones()}
\item \texttt{ones\_like()}
\end{itemize}

\end{itemize}
\end{frame}
\begin{frame}
\frametitle{Array operations}
\label{sec-14}

\begin{itemize}
\item \texttt{a1 * 2}
    returns a new array with all elements of \texttt{a1} multiplied by \texttt{2}.

\begin{itemize}
\item Similarly \texttt{+}, \texttt{-} \& \texttt{/}.
\end{itemize}

\item \texttt{a1 + 2}
    returns a new array with all elements of \texttt{a1} summed with \texttt{2}.
\item \texttt{a1 += 2}
    adds \texttt{2} to all elements of array \texttt{a1}.

\begin{itemize}
\item Similarly \texttt{-=}, \texttt{*=} \& \texttt{/=}.
\end{itemize}

\item \texttt{a1 + a2}
    does elements-wise addition.

\begin{itemize}
\item Similarly \texttt{-}, \texttt{*} \& \texttt{/}.
\end{itemize}

\item \texttt{a1 * a2}
    does element-wise multiplication
\end{itemize}


  \textbf{Note} - array(A) * array(B) does element wise multiplication and not matrix multiplication
\end{frame}
\begin{frame}
\frametitle{Summary}
\label{sec-15}

  In this tutorial we covered,
\begin{itemize}
\item Basics of arrays
\item Creating arrays
\item Arrays from lists
\item Basic array operations
\end{itemize}
\end{frame}
\begin{frame}
\frametitle{Thank you!}
\label{sec-16}

  \begin{block}{}
  \begin{center}
  This spoken tutorial has been produced by the
  \textcolor{blue}{FOSSEE} team, which is funded by the 
  \end{center}
  \begin{center}
    \textcolor{blue}{National Mission on Education through \\
      Information \& Communication Technology \\ 
      MHRD, Govt. of India}.
  \end{center}  
  \end{block}
\end{frame}

\end{document}
