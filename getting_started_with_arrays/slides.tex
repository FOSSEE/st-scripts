% Created 2011-05-27 Fri 12:20
\documentclass[presentation]{beamer}
\usepackage[latin1]{inputenc}
\usepackage[T1]{fontenc}
\usepackage{fixltx2e}
\usepackage{graphicx}
\usepackage{longtable}
\usepackage{float}
\usepackage{wrapfig}
\usepackage{soul}
\usepackage{textcomp}
\usepackage{marvosym}
\usepackage{wasysym}
\usepackage{latexsym}
\usepackage{amssymb}
\usepackage{hyperref}
\tolerance=1000
\usepackage[english]{babel} \usepackage{ae,aecompl}
\usepackage{mathpazo,courier,euler} \usepackage[scaled=.95]{helvet}
\usepackage{listings}
\lstset{language=Python, basicstyle=\ttfamily\bfseries,
commentstyle=\color{red}\itshape, stringstyle=\color{darkgreen},
showstringspaces=false, keywordstyle=\color{blue}\bfseries}
\providecommand{\alert}[1]{\textbf{#1}}

\title{}
\author{FOSSEE}
\date{}

\usetheme{Warsaw}\usecolortheme{default}\useoutertheme{infolines}\setbeamercovered{transparent}
\begin{document}











\begin{frame}

\begin{center}
\vspace{12pt}
\textcolor{blue}{\huge Getting started with Arrays}
\end{center}
\vspace{18pt}
\begin{center}
\vspace{10pt}
\includegraphics[scale=0.95]{../images/fossee-logo.png}\\
\vspace{5pt}
\scriptsize Developed by FOSSEE Team, IIT-Bombay. \\ 
\scriptsize Funded by National Mission on Education through ICT\\
\scriptsize  MHRD,Govt. of India\\
\includegraphics[scale=0.30]{../images/iitb-logo.png}\\
\end{center}
\end{frame}
\begin{frame}
\frametitle{Objectives}
\label{sec-2}

  At the end of this tutorial, you will be able to, 


\begin{itemize}
\item Create arrays using data.
\item Create arrays from lists.
\item Perform basic array operations.
\item Create identity matrix.
\item Use functions zeros(), zeros\_like(), ones(), ones\_like()
\end{itemize}
\end{frame}
\begin{frame}
\frametitle{Pre-requisite}
\label{sec-3}

  Spoken tutorial on -

\begin{itemize}
\item Getting started with Lists.
\end{itemize}
\end{frame}
\begin{frame}
\frametitle{Overview of Arrays}
\label{sec-4}


\begin{itemize}
\item Arrays are homogeneous data structures.
\begin{itemize}
\item elements have to the same data type
\end{itemize}
\item Arrays are faster compared to lists
\begin{itemize}
\item at least \emph{80-100 times} faster than lists
\end{itemize}
\end{itemize}
\end{frame}
\begin{frame}[fragile]
\frametitle{\verb~.shape~ of array}
\label{sec-5}


\begin{itemize}
\item \verb~.shape~
    To find the shape of the array
\begin{verbatim}
     In []: a2.shape
\end{verbatim}

\item \verb~.shape~
    returns a tuple of shape
\end{itemize}
\end{frame}
\begin{frame}
\frametitle{Exercise 1}
\label{sec-6}

  Find out the shape of the other arrays(a1, a3, ar) that we have created.
\end{frame}
\begin{frame}
\frametitle{\verb~identity()~, \verb~zeros()~ methods}
\label{sec-7}


\begin{itemize}
\item \verb~identity(n)~
    Creates an identity matrix, a square matrix of order (n, n) with diagonal elements 1 and others 0.
\item \verb~zeros((m, n))~
    Creates an \verb~m X n~ matrix with all elements 0.
\end{itemize}
\end{frame}
\begin{frame}
\frametitle{Learning exercise}
\label{sec-8}

  Find out about

\begin{itemize}
\item \verb~zeros\_like()~
\item \verb~ones()~
\item \verb~ones\_like()~
\end{itemize}
\end{frame}
\begin{frame}
\frametitle{Summary}
\label{sec-9}

  In this tutorial, we have learnt to,

\begin{itemize}
\item Create an array using the ``array()`` function.
\item Convert a list to an array.
\item Perform some basic operations on arrays like addition,multiplication.
\item Use functions like
\begin{itemize}
\item .shape
\item arrange()
\item .reshape
\item zeros() \& zeros\_like()
\item ones() \& ones\_like()
\end{itemize}
\end{itemize}
\end{frame}
\begin{frame}
\frametitle{Evaluation}
\label{sec-10}


\begin{enumerate}
\item ``x = array([1, 2, 3], [5, 6, 7])`` is a valid statement
\begin{itemize}
\item True
\item False
\end{itemize}
\vspace{4pt}
\item What does the ``ones\_like()`` function do?
   
     (A) Returns an array of ones with the same shape and type as a
         given array.\\
     (B) Return a new array of given shape and type, filled with ones.
\vspace{6pt}     

     Read the statements and answer,
\begin{itemize}
\item Only statement A is correct.
\item Only statement B is correct.
\item Both statement A and B are correct.
\item Both statement A and B are incorrect.
\end{itemize}
\end{enumerate}
\end{frame}
\begin{frame}
\frametitle{Solutions}
\label{sec-11}


\begin{enumerate}
\item False\\
     x = array([[1, 2, 3], [5, 6, 7]])
\vspace{12pt}
\item Statement A - Returns an array of ones with the same shape and type as a
                  given array.
\end{enumerate}
\end{frame}
\begin{frame}

  \begin{block}{}
  \begin{center}
  \textcolor{blue}{\Large THANK YOU!} 
  \end{center}
  \end{block}
\begin{block}{}
  \begin{center}
    For more Information, visit our website\\
    \url{http://fossee.in/}
  \end{center}  
  \end{block}
\end{frame}

\end{document}