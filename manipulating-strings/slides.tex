% Created 2010-10-11 Mon 11:27
\documentclass[presentation]{beamer}
\usepackage[latin1]{inputenc}
\usepackage[T1]{fontenc}
\usepackage{fixltx2e}
\usepackage{graphicx}
\usepackage{longtable}
\usepackage{float}
\usepackage{wrapfig}
\usepackage{soul}
\usepackage{textcomp}
\usepackage{marvosym}
\usepackage{wasysym}
\usepackage{latexsym}
\usepackage{amssymb}
\usepackage{hyperref}
\tolerance=1000
\usepackage[english]{babel} \usepackage{ae,aecompl}
\usepackage{mathpazo,courier,euler} \usepackage[scaled=.95]{helvet}
\usepackage{listings}
\lstset{language=Python, basicstyle=\ttfamily\bfseries,
commentstyle=\color{red}\itshape, stringstyle=\color{darkgreen},
showstringspaces=false, keywordstyle=\color{blue}\bfseries}
\providecommand{\alert}[1]{\textbf{#1}}

\title{Manipulating strings}
\author{FOSSEE}
\date{}

\usetheme{Warsaw}\usecolortheme{default}\useoutertheme{infolines}\setbeamercovered{transparent}
\begin{document}

\maketitle









\begin{frame}
\frametitle{Outline}
\label{sec-1}

\begin{itemize}
\item Slicing strings to get sub-strings
\item Reversing strings
\item Replacing characters in strings.
\item Converting strings to upper or lower case
\item Joining a list of strings
\end{itemize}
\end{frame}
\begin{frame}
\frametitle{Question 1}
\label{sec-2}

  Obtain the sub-string excluding the first and last characters from
  the string \texttt{s}.
\end{frame}
\begin{frame}[fragile]
\frametitle{Solution 1}
\label{sec-3}

\lstset{language=Python}
\begin{lstlisting}
In []:  s[1:-1]
\end{lstlisting}
\end{frame}
\begin{frame}
\frametitle{Question 2}
\label{sec-4}

  Given a list week, week = \texttt{week = ["sun", "mon", "tue", "wed",   "thu", "fri", "sat"]}. Check if \texttt{s} is a valid name of a day of the
  week. Change the solution to this problem, to include forms like,
  SAT, SATURDAY, Saturday and Sat.
\end{frame}
\begin{frame}[fragile]
\frametitle{Solution 2}
\label{sec-5}

\lstset{language=Python}
\begin{lstlisting}
In []:  s in week
In []:  s.lower()[:3] in week
\end{lstlisting}
\end{frame}
\begin{frame}
\frametitle{Question 3}
\label{sec-6}

  Given \texttt{email} -- \texttt{info@fossee[dot]in}

  Replace the \texttt{[dot]} with \texttt{.} in \texttt{email}
\end{frame}
\begin{frame}[fragile]
\frametitle{Solution 3}
\label{sec-7}

\lstset{language=Python}
\begin{lstlisting}
email.replace('[dot], '.')
print email
\end{lstlisting}
\end{frame}
\begin{frame}
\frametitle{Question 4}
\label{sec-8}

  From the \texttt{email\_str} that we generated, change the separator to be a
  semicolon instead of a comma.
\end{frame}
\begin{frame}[fragile]
\frametitle{Solution 4}
\label{sec-9}

\lstset{language=Python}
\begin{lstlisting}
email_str = email_str.replace(",", ";")
\end{lstlisting}
\end{frame}
\begin{frame}
\frametitle{Summary}
\label{sec-10}

  You should now be able to --
\begin{itemize}
\item Slice strings and get sub-strings out of them
\item Reverse strings
\item Replace characters in strings.
\item Convert strings to upper or lower case
\item Join a list of strings
\end{itemize}
\end{frame}
\begin{frame}
\frametitle{Thank you!}
\label{sec-11}

  \begin{block}{}
  \begin{center}
  This spoken tutorial has been produced by the
  \textcolor{blue}{FOSSEE} team, which is funded by the 
  \end{center}
  \begin{center}
    \textcolor{blue}{National Mission on Education through \\
      Information \& Communication Technology \\ 
      MHRD, Govt. of India}.
  \end{center}  
  \end{block}
\end{frame}

\end{document}
