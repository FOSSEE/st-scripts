% Created 2010-10-10 Sun 21:15
\documentclass[presentation]{beamer}
\usepackage[latin1]{inputenc}
\usepackage[T1]{fontenc}
\usepackage{fixltx2e}
\usepackage{graphicx}
\usepackage{longtable}
\usepackage{float}
\usepackage{wrapfig}
\usepackage{soul}
\usepackage{textcomp}
\usepackage{marvosym}
\usepackage{wasysym}
\usepackage{latexsym}
\usepackage{amssymb}
\usepackage{hyperref}
\tolerance=1000
\usepackage[english]{babel} \usepackage{ae,aecompl}
\usepackage{mathpazo,courier,euler} \usepackage[scaled=.95]{helvet}
\usepackage{listings}
\lstset{language=Python, basicstyle=\ttfamily\bfseries,
commentstyle=\color{red}\itshape, stringstyle=\color{darkgreen},
showstringspaces=false, keywordstyle=\color{blue}\bfseries}
\providecommand{\alert}[1]{\textbf{#1}}

\title{Loops}
\author{FOSSEE}
\date{}

\usetheme{Warsaw}\usecolortheme{default}\useoutertheme{infolines}\setbeamercovered{transparent}
\begin{document}

\maketitle









\begin{frame}
\frametitle{Outline}
\label{sec-1}

\begin{itemize}
\item Loop while a condition is true.
\item Iterate over a sequence
\item Breaking out of loops.
\item Skipping iterations.
\end{itemize}
\end{frame}
\begin{frame}
\frametitle{Question 1}
\label{sec-2}

  Write a \texttt{while} loop to print the squares of all the even
  numbers below 10. 
\end{frame}
\begin{frame}[fragile]
\frametitle{Solution 1}
\label{sec-3}

\lstset{language=Python}
\begin{lstlisting}
In []: i = 2

In []:  while i<10:
 ....:     print i*i
 ....:     i += 2
\end{lstlisting}
\end{frame}
\begin{frame}
\frametitle{Question 2}
\label{sec-4}

  Write a \texttt{for} loop to print the squares of all the even numbers
  below 10.
\end{frame}
\begin{frame}[fragile]
\frametitle{Solution 2}
\label{sec-5}

\lstset{language=Python}
\begin{lstlisting}
In []: for n in range(2, 10, 2):
 ....:     print n*n
\end{lstlisting}
\end{frame}
\begin{frame}
\frametitle{Question 3}
\label{sec-6}

  Using the \texttt{continue} keyword modify the \texttt{for} loop to print the
  squares of even numbers below 10, to print the squares of only
  multiples of 4. (Do not modify the range function call.)
\end{frame}
\begin{frame}[fragile]
\frametitle{Solution 3}
\label{sec-7}

\lstset{language=Python}
\begin{lstlisting}
for n in range(2, 10, 2):
    if n%4:
        continue      
    print n*n
\end{lstlisting}
\end{frame}
\begin{frame}
\frametitle{Summary}
\label{sec-8}

  You should now be able to --
\begin{itemize}
\item use the \texttt{for} loop
\item use the \texttt{while} loop
\item Use \texttt{break}, \texttt{continue} and \texttt{pass} statements
\end{itemize}
\end{frame}
\begin{frame}
\frametitle{Thank you!}
\label{sec-9}

  \begin{block}{}
  \begin{center}
  This spoken tutorial has been produced by the
  \textcolor{blue}{FOSSEE} team, which is funded by the 
  \end{center}
  \begin{center}
    \textcolor{blue}{National Mission on Education through \\
      Information \& Communication Technology \\ 
      MHRD, Govt. of India}.
  \end{center}  
  \end{block}
\end{frame}

\end{document}
