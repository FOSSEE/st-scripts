\documentclass[17pt]{beamer}
\usepackage{amsmath}
\usepackage{framed}
\definecolor{Blue}{RGB}{0.16,0.32,0.75}
\setbeamercolor{structure}{fg=blue}
\usepackage{beamerthemesplit}




\definecolor{blue}{rgb}{0.16,0.32,0.75}
\setbeamercolor{structure}{fg=blue}
\author[FOSSEE]{}
\institute[IIT Bombay]{}
\date[]{}
% \setbeamercovered{transparent}

% theme split
\usepackage{verbatim}
\newenvironment{colorverbatim}[1][]%
{%
	\color{blue}
	\verbatim
}%
{%
	\endverbatim
}%

\usepackage{mathpazo,courier,euler}
\usepackage{listings}
\lstset{language=sh,
	basicstyle=\ttfamily\bfseries,
	showstringspaces=false,
	keywordstyle=\color{black}\bfseries}

% logo
\logo{\includegraphics[height=1.30 cm]{St-logo.png}}
\logo{\includegraphics[height=1.30 cm]{fossee-logo.png}
	
	\hspace{7.5cm}
	\includegraphics[scale=0.3]{fossee-logo.png}\\
	\hspace{281pt}
	\includegraphics[scale=0.08]{St-logo.png}}


\newcounter{saveenumi}
\newcommand{\seti}{\setcounter{saveenumi}{\value{enumi}}}
\newcommand{\conti}{\setcounter{enumi}{\value{saveenumi}}}

\begin{document}
	% sf family, bold font
	\sffamily \bfseries
	%\LARGE
	\title
	[Python for Scientific Computing]
	%\hspace{0.5cm}
	%\insertframenumber/\inserttotalframenumber]
	{\large Testing and debugging}
	\author
	[FOSSEE, IIT Bombay]
	{{\small Spoken Tutorial Project \\ http://spoken-tutorial.org \\ National Mission on Education  through ICT  \\ http://sakshat.ac.in } \\
		{\small Script: Thirumalesh H S}\\
		{\small Narrator: Kiran Kishore}\\
		{\small IIT Bombay}\\
		{\small 21 December 2015}}
	
	% slide 1
	\begin{frame}
		\titlepage
	\end{frame}
%%%%%%%%%%%%%%%%%%%%%%%%%%%%%%%%%%%%%%%%%%%%%%%%%%%%%%%%%%%%%%%%%%%%%%%%%%%%%%%%
\begin{frame}
\frametitle{Objectives}
\label{sec-2.1}

At the end of this tutorial, you should be able, \pause

\begin{itemize}
\item Understand what is software testing.\pause
\item Test simple functions for their functionality.\pause
\item Automate tests.
\end{itemize}
\end{frame}
%%%%%%%%%%%%%%%%%%%%%%%%%%%%%%%%%%%%%%%%%%%%%%%%%%%%%%%%%%%%%%%%%%%%%%%%%%%%%%%%
\begin{frame}
\frametitle{Objectives contd..}
\label{sec-2.2}

\begin{itemize}
\item Understand the need for coding style.\pause
\item Learn  some of the standards followed by the Python Community.\pause
\item Handle Errors and Exceptions.
\end{itemize}
\end{frame}
%%%%%%%%%%%%%%%%%%%%%%%%%%%%%%%%%%%%%%%%%%%%%%%%%%%%%%%%%%%%%%%%%%%%%%%%%%%%%%%%
\begin{frame}
\frametitle{System Specifications}\pause
\begin{itemize}
\item Ubuntu Linux 14.04\pause
\item \texttt{Python 2.7.6} \pause
\item \texttt{IPython 4.0.0}
\end{itemize}
\end{frame}
%%%%%%%%%%%%%%%%%%%%%%%%%%%%%%%%%%%%%%%%%%%%%%%%%%%%%%%%%%%%%%%%%%%%%%%%%%%%%%%%
\begin{frame}
	\frametitle{Pre-requisite}
	\label{sec-3}
	
	To practice this tutorial, you should know how to -\pause
	
	\begin{itemize}
		\item use functions\pause
	\end{itemize}
	If not, see the pre-requisite Python tutorials on {\color{blue}http://spoken-tutorial.org}
\end{frame}
%%%%%%%%%%%%%%%%%%%%%%%%%%%%%%%%%%%%%%%%%%%%%%%%%%%%%%%%%%%%%%%%%%%%%%%%%%%%%%%%
\begin{frame}[fragile]
	\frametitle{What is Software testing?}
	
	\begin{itemize}
		\item Software testing is an activity aimed at evaluating a program, and determining that it meets its required results.
	\end{itemize}
\end{frame}
%%%%%%%%%%%%%%%%%%%%%%%%%%%%%%%%%%%%%%%%%%%%%%%%%%%%%%%%%%%%%%%%%%%%%%%%%%%%%%%%
\begin{frame}[fragile]
\frametitle{gcd function}

 Create find\_gcd.py file with:\\

\begin{lstlisting}
def gcd(a, b):
    if b == 0:
          return a
     return gcd(b, a%b)   
\end{lstlisting}
\end{frame}
%%%%%%%%%%%%%%%%%%%%%%%%%%%%%%%%%%%%%%%%%%%%%%%%%%%%%%%%%%%%%%%%%%%%%%%%%%%%%%%%
\begin{frame}
\frametitle{Assignment 1}

\begin{itemize}
\item For the same inputs as gcd write automated tests for LCM.\pause
\item Use the data from the file lcmtestcases.txt
\end{itemize}
\end{frame}
%%%%%%%%%%%%%%%%%%%%%%%%%%%%%%%%%%%%%%%%%%%%%%%%%%%%%%%%%%%%%%%%%%%%%%%%%%%%%%%%

\begin{frame}
	\frametitle{Coding Style}
	
	\begin{itemize}
		\item A good program should be readable\pause
		\item Code is read more often than it is written. This is because, that way, other people can learn from it and extend and improve it. 
	\end{itemize}
\end{frame}
%%%%%%%%%%%%%%%%%%%%%%%%%%%%%%%%%%%%%%%%%%%%%%%%%%%%%%%%%%%%%%%%%%%%%%%%%%%%%%%%
\begin{frame}[fragile]
	\frametitle{Meaning full names}
	
	\texttt{mass = 10}\\
	\texttt{acceleration = 2}\\
	\texttt{force = mass * acceleration}\\
\end{frame}
%%%%%%%%%%%%%%%%%%%%%%%%%%%%%%%%%%%%%%%%%%%%%%%%%%%%%%%%%%%%%%%%%%%%%%%%%%%%%%%%
\begin{frame}
\frametitle{Code style}


\begin{itemize}
\item Four Space Indentation\pause
\item 79 characters limit on a line\pause
\item Funtions and methods should be seperated with two 
   blank lines\pause
\item Use Docstring to explain units of code performing specific task\pause
\item Use whitespace around operators and after punctuation.
\end{itemize}
\end{frame}
%%%%%%%%%%%%%%%%%%%%%%%%%%%%%%%%%%%%%%%%%%%%%%%%%%%%%%%%%%%%%%%%%%%%%%%%%%%%%%%%
\begin{frame}
\frametitle{Summary}
\label{sec-17.1}

 In this tutorial, we have learnt to, 
        
\begin{itemize}
\item Create simple tests for a function.\pause
\item Automate tests using many predefined test cases.\pause
\item Use the python coding standards.
\end{itemize}
\end{frame}
%%%%%%%%%%%%%%%%%%%%%%%%%%%%%%%%%%%%%%%%%%%%%%%%%%%%%%%%%%%%%%%%%%%%%%%%%%%%%%%%
\begin{frame}
\frametitle{Summary contd..}
\label{sec-17.1}
        
\begin{itemize}
\item Handle exception using \texttt{try} and \texttt{except}.\pause
\item Use \texttt{\%debug} for debugging on \texttt{ipython}.
\end{itemize}
\end{frame}
%%%%%%%%%%%%%%%%%%%%%%%%%%%%%%%%%%%%%%%%%%%%%%%%%%%%%%%%%%%%%%%%%%%%%%%%%%%%%%%%
\begin{frame}
\frametitle{Self assessment questions}
\label{sec-18.1}

\begin{enumerate}
\item What is proper indentation for python code according to style guidelines?\pause
	\begin{itemize}
	\item two space identation
	\item three space identation
	\item four Space Indentation
	\item no Indentation
\end{itemize}
\seti
\end{enumerate}
\end{frame}
%%%%%%%%%%%%%%%%%%%%%%%%%%%%%%%%%%%%%%%%%%%%%%%%%%%%%%%%%%%%%%%%%%%%%%%%%%%%%%%%
\begin{frame}
\frametitle{Self assessment questions}
\label{sec-18.2}

\begin{enumerate}
\conti
\item How do you start the debugger on ipython?\pause
	\begin{itemize}
	\item \texttt{debug}
	\item \texttt{\%debug}
	\item \texttt{\%debugger}
	\item \texttt{start debugger}
	\end{itemize}\pause
\end{enumerate}
\end{frame}
%%%%%%%%%%%%%%%%%%%%%%%%%%%%%%%%%%%%%%%%%%%%%%%%%%%%%%%%%%%%%%%%%%%%%%%%%%%%%%%%
\begin{frame}
\frametitle{Solutions}
\label{sec-19}

\begin{enumerate}
\item Four Space Indentation\pause
\vspace{12pt}
\item \texttt{\%debug}
\end{enumerate}
\end{frame}
%%%%%%%%%%%%%%%%%%%%%%%%%%%%%%%%%%%%%%%%%%%%%%%%%%%%%%%%%%%%%%%%%%%%%%%%%%%%%%%%
\begin{frame}
	\frametitle{Forum to answer questions}
	\begin{itemize}
		\item Do you have questions in THIS Spoken Tutorial?
		\item Choose the minute and second where you have the question.
		\item Explain your question briefly.
		\item Someone from the FOSSEE team will answer them. Please visit 
	\end{itemize}
	\begin{center}
		{\color{blue}{http://forums.spoken-tutorial.org/}}
	\end{center} 
\end{frame}
%%%%%%%%%%%%%%%%%%%%%%%%%%%%%%%%%%%%%%%%%%%%%%%%%%%%%%%%%%%%%%%%%%%%%%%%%%%%%%%%
\begin{frame}
	\frametitle{Forum to answer questions}
	\begin{itemize}
		\item Questions not related to the Spoken Tutorial?
		\item Do you have general / technical questions on the Software?
		\item Please visit the FOSSEE Forum
		\begin{center}
			{\color{blue}{http://forums.fossee.in/}}
		\end{center}
		\item Choose the Software and post your question.
	\end{itemize}
\end{frame}
%%%%%%%%%%%%%%%%%%%%%%%%%%%%%%%%%%%%%%%%%%%%%%%%%%%%%%%%%%%%%%%%%%%%%%%%%%%%%%%%
\begin{frame}
	\frametitle{Textbook Companion Project}
	\begin{itemize}
		\item The FOSSEE team coordinates coding of solved examples of popular
		books 
		\item We give honorarium and certificate to those who do this
	\end{itemize}
	For more details, please visit this site:
	\begin{center}
		{\color{blue}{http://tbc-python.fossee.in/}}
	\end{center}
\end{frame}
%%%%%%%%%%%%%%%%%%%%%%%%%%%%%%%%%%%%%%%%%%%%%%%%%%%%%%%%%%%%%%%%%%%%%%%%%%%%%%%%
\begin{frame}
	\frametitle{Acknowledgements}
	\begin{itemize}
		\item Spoken Tutorial Project is a part of the Talk to a Teacher  project 
		\item It is supported by the National Mission on Education through  ICT, MHRD, Government of India 
		\item More information on this Mission is available at: \\{\color{blue}\url{http://spoken-tutorial.org/NMEICT-Intro}}
	\end{itemize}
\end{frame}
%%%%%%%%%%%%%%%%%%%%%%%%%%%%%%%%%%%%%%%%%%%%%%%%%%%%%%%%%%%%%%%%%%%%%%%%%%%%%%%%
\begin{frame}
	
	\begin{block}{}
		\begin{center}
			\textcolor{blue}{\Large THANK YOU!} 
		\end{center}
	\end{block}
	\begin{block}{}
		\begin{center}
			For more Information, visit our website\\
			{http://fossee.in/}
		\end{center}  
	\end{block}
\end{frame}

\end{document}
