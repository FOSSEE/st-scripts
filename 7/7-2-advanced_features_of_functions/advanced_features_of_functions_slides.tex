\documentclass[17pt]{beamer}
\usepackage{amsmath}
\usepackage[english]{babel}
\usepackage{framed}
\definecolor{Blue}{RGB}{0.16,0.32,0.75}
\setbeamercolor{structure}{fg=blue}
\usepackage{beamerthemesplit}
\definecolor{blue}{rgb}{0.16,0.32,0.75}
\setbeamercolor{structure}{fg=blue}
\author[FOSSEE]{}
\institute[IIT Bombay]{}
\date[]{}
% \setbeamercovered{transparent}

% theme split
\usepackage{verbatim}
\newenvironment{colorverbatim}[1][]%
{%
\color{blue}
\verbatim
}%
{%
\endverbatim
}%

\usepackage{mathpazo,courier,euler}
\usepackage{listings}
\lstset{language=sh,
    basicstyle=\ttfamily\bfseries,
  showstringspaces=false,
  keywordstyle=\color{black}\bfseries}

% logo
\logo{\includegraphics[height=1.30 cm]{St-logo.png}}
\logo{\includegraphics[height=1.30 cm]{fossee-logo.png}

\hspace{7.5cm}
\includegraphics[scale=0.3]{fossee-logo.png}\\
\hspace{281pt}
\includegraphics[scale=0.08]{St-logo.png}}


\newcounter{saveenumi}
\newcommand{\seti}{\setcounter{saveenumi}{\value{enumi}}}
\newcommand{\conti}{\setcounter{enumi}{\value{saveenumi}}}

\begin{document}
% sf family, bold font
\sffamily \bfseries
%\LARGE
\title
[Python for Scientific Computing]
%\hspace{0.3cm}
%\insertframenumber/\inserttotalframenumber]
{\large Features of Functions}
\author
[FOSSEE, IIT BOMBAY]
{{\small Spoken Tutorial Project \\ http://spoken-tutorial.org \\ National Mission on Education  through ICT  \\ http://sakshat.ac.in } \\[0.1cm]
{\small Script: Aditya Palaparthy }\\
{\small Narration: Kiran K}\\
{\small IIT Bombay} \\ [0.1cm]
{\small  10 December 2015}}
% slide 1
\begin{frame}
   \titlepage
\end{frame}
%%%%%%%%%%%%%%%%%%%%%%%%%%%%%%%%%%%%%%%%%%%%%%%%%%%%%%%%%%%%%%%%%%%%%%%%%%%%%%%%
\begin{frame}
\frametitle{Objectives}
\label{sec-2.1}

  In this tutorial we will learn, \pause
\begin{itemize}
\item Assign default values to arguments, when defining functions.\pause
\item Define and call functions with keyword arguments.
\end{itemize}
\end{frame}
%%%%%%%%%%%%%%%%%%%%%%%%%%%%%%%%%%%%%%%%%%%%%%%%%%%%%%%%%%%%%%%%%%%%%%%%%%%%%%%%
\begin{frame}
\frametitle{Objectives}
\label{sec-2.2}
\begin{itemize}
\item Learn some of the built-in functions available in Python standard 
    library and the scientific computing libraries.
\end{itemize}
\end{frame}
%%%%%%%%%%%%%%%%%%%%%%%%%%%%%%%%%%%%%%%%%%%%%%%%%%%%%%%%%%%%%%%%%%%%%%%%%%%%%%%%
\begin{frame}
\frametitle{System Specifications}\pause
\begin{itemize}
\item Ubuntu Linux 14.04\pause
\item \texttt{Python 2.7.6} \pause
\item \texttt{IPython 4.0.0}
\end{itemize}
\end{frame}
%%%%%%%%%%%%%%%%%%%%%%%%%%%%%%%%%%%%%%%%%%%%%%%%%%%%%%%%%%%%%%%%%%%%%%%%%%%%%%%%
\begin{frame}
\frametitle{Pre-requisite}
\label{sec-3}
To practise this tutorial, you should know how to 
\begin{itemize}
\item run basic Python commands on the ipython console.
\item use functions
\end{itemize}
If not, see the pre-requisite Python tutorials on {\color{blue}http://spoken-tutorial.org}
\end{frame}
%%%%%%%%%%%%%%%%%%%%%%%%%%%%%%%%%%%%%%%%%%%%%%%%%%%%%%%%%%%%%%%%%%%%%%%%%%%%%%%%


\begin{frame}[fragile]
\frametitle{Functions}
\label{sec-4.1}
\begin{itemize}
\lstset{language=Python}

\item s.strip() strips on spaces. 
\item s.strip('@') strips the string of '@' 
symbols.

\item plot(x, y) plots x v/s y using default 
line style. 
\item plot(x, y, 'o') plots x v/s y with 
circle markers. 
\end{itemize}
\end{frame}

\begin{frame}[fragile]
\frametitle{Functions}
\label{sec-4.2}
\begin{itemize}
\lstset{language=Python}
\item linspace(0,2*pi,100) returns 100 pts between 0 and 2pi

\item linspace(0,2*pi) returns 50 pts between 0 and 2pi
\end{itemize}
\end{frame}

\begin{frame}
\frametitle{Exercise 1}
\label{sec-5}

\begin{itemize}
\item Redefine the function \texttt{welcome}, by interchanging it's
  arguments.
\vspace{3pt}  
  Place the \texttt{name} argument with it's default value of
  ``World'' before the \texttt{greet} argument.
\end{itemize}
\end{frame}

\begin{frame}[fragile]
\frametitle{Solution 1}
\label{sec-6}
\begin{itemize}
\item 
\lstset{language=Python}
\begin{small}
\begin{lstlisting}
def welcome(name="World", greet):
    print greet, name
\end{lstlisting}
\end{small}\pause
\item  We get an error that reads, \\
  \begin{small}
  \verb~SyntaxError: non-default argument~
  \verb~follows default argument~.
  \end{small}\pause
\item When defining a function all the
  argument with default values should come at the end.

\end{itemize}
\end{frame}

\begin{frame}
\frametitle{Exercise2}
\label{sec-8}

\begin{itemize}
\item Redefine the function \texttt{welcome} with a default value of
  ``Hello'' to the \texttt{greet} argument.\\ 
  Then, call the function without any arguments.
\end{itemize}
\end{frame}

\begin{frame}[fragile]
\frametitle{Keyword arguments examples}
\label{sec-9}

\lstset{language=Python}
\begin{small}
\begin{lstlisting}
legend(['sin(2y)'], loc = 'center')

plot(y, sin(y), 'g', linewidth = 2)

annotate('local max', xy = (1.5, 1))

pie(science.values(),      
            labels = science.keys())
\end{lstlisting}
\end{small}
\end{frame}

\begin{frame}[fragile]
\frametitle{Built-in functions}
\label{sec-10}

\lstset{language=Python}
\begin{small}
\begin{lstlisting}
Math functions - abs, sin, ....
 
Plot functions - plot, bar, pie ...
 
Boolean functions - and, or, not ...
\end{lstlisting}
\end{small}
\end{frame}

\begin{frame}[fragile]
\frametitle{Classes of functions}
\label{sec-11}

\lstset{language=Python}
\begin{small}
\begin{lstlisting}
- pylab
  - plot, bar, contour, boxplot, 
    errorbar, log, polar, quiver, 
    semilog

- scipy (modules)
  - fftpack, stats, linalg, ndimage, 
    signal, optimize, integrate
\end{lstlisting}
\end{small}
\end{frame}

\begin{frame}
\frametitle{Summary}
\label{sec-12}

 In this tutorial, we have learnt to, 

\begin{itemize}
\item Define functions with default arguments.\pause
\item Call functions using keyword arguments.\pause
\item Use the range of functions available in the Python standard library 
   and the Scientific Computing related packages.
\end{itemize}
\end{frame}

\begin{frame}[fragile]
\frametitle{Evaluation}
\label{sec-13.1}

\begin{enumerate}
\item All arguments of a function cannot have default values.
 True or False?\pause
\vspace{3pt} 
\item The following is a valid function definition. True or False? \pause
\lstset{language=Python}
\begin{footnotesize}
\begin{lstlisting}
def separator(count=40, char, show=False):
     if show:
          print char * count
     return char * count
\end{lstlisting}
\end{footnotesize}
\seti
\end{enumerate}
\end{frame}

\begin{frame}[fragile]
\frametitle{Evaluation}
\label{sec-13.2}

\begin{enumerate}
\conti
\item When calling a function,
\begin{itemize}
\item the arguments should always be in the order in which they are defined.\pause
\item only keyword arguments can be in any order, but should be called
     at the beginning.\pause
\item only keyword arguments can be in any order, but should be called at the end.
\end{itemize}
\end{enumerate}
\end{frame}

\begin{frame}
\frametitle{Solutions}
\label{sec-14}

\begin{enumerate}
\item False\pause
\vspace{12pt}
\item False\pause
\vspace{12pt}
\item Only keyword arguments can be in any order, 
   but should be called at the end.
\end{enumerate}
\end{frame}
%%%%%%%%%%%%%%%%%%%%%%%%%%%%%%%%%%%%%%%%%%%%%%%%%%%%%%%%%%%%%%%%%%%%%%%%%%%%%%%%
\begin{frame}
\frametitle{Forum to answer questions}
\begin{itemize}
\item Do you have questions in THIS Spoken Tutorial?
\item Choose the minute and second where you have the question.
\item Explain your question briefly.
\item Someone from the FOSSEE team will answer them. Please visit 
\end{itemize}
\begin{center}
{\color{blue}{http://forums.spoken-tutorial.org/}}
 \end{center} 
\end{frame}
%%%%%%%%%%%%%%%%%%%%%%%%%%%%%%%%%%%%%%%%%%%%%%%%%%%%%%%%%%%%%%%%%%%%%%%%%%%%%%%%
\begin{frame}
\frametitle{Forum to answer questions}
\begin{itemize}
\item Questions not related to the Spoken Tutorial?
\item Do you have general / technical questions on the Software?
\item Please visit the FOSSEE Forum
\begin{center}
{\color{blue}{http://forums.fossee.in/}}
 \end{center}
\item Choose the Software and post your question.
\end{itemize}
\end{frame}
%%%%%%%%%%%%%%%%%%%%%%%%%%%%%%%%%%%%%%%%%%%%%%%%%%%%%%%%%%%%%%%%%%
\begin{frame}
\frametitle{Textbook Companion Project}
\begin{itemize}
\item The FOSSEE team coordinates coding of solved examples of popular
  books 
\item We give honorarium and certificate to those who do this
\end{itemize}
For more details, please visit this site:
\begin{center}
{\color{blue}{http://tbc-python.fossee.in/}}
\end{center}
\end{frame}
%%%%%%%%%%%%%%%%%%%%%%%%%%%%%%%%%%%%%%%%%%%%%%%%%%%%%%%%%%%%%%%%%%%%%%%%%%%%%%%%
\begin{frame}
\frametitle{Acknowledgements}
\begin{itemize}
\item Spoken Tutorial Project is a part of the Talk to a Teacher  project 
\item It is supported by the National Mission on Education through  ICT, MHRD, Government of India 
\item More information on this Mission is available at: \\{\color{blue}\url{http://spoken-tutorial.org/NMEICT-Intro}}
\end{itemize}
\end{frame}
%%%%%%%%%%%%%%%%%%%%%%%%%%%%%%%%%%%%%%%%%%%%%%%%%%%%%%%%%%%%%%%%%%%%%%%%%%%%%%%%
\begin{frame}

  \begin{block}{}
  \begin{center}
  \textcolor{blue}{\Large THANK YOU!} 
  \end{center}
  \end{block}
\begin{block}{}
  \begin{center}
    For more Information, visit our website\\
    {http://fossee.in/}
  \end{center}  
  \end{block}
\end{frame}

\end{document}
