% Created 2010-12-18 Sat 12:23
\documentclass[presentation]{beamer}
\usepackage[latin1]{inputenc}
\usepackage[T1]{fontenc}
\usepackage{fixltx2e}
\usepackage{graphicx}
\usepackage{longtable}
\usepackage{float}
\usepackage{wrapfig}
\usepackage{soul}
\usepackage{t1enc}
\usepackage{textcomp}
\usepackage{marvosym}
\usepackage{wasysym}
\usepackage{latexsym}
\usepackage{amssymb}
\usepackage{hyperref}
\tolerance=1000
\usepackage[english]{babel} \usepackage{ae,aecompl}
\usepackage{mathpazo,courier,euler} \usepackage[scaled=.95]{helvet}
\usepackage{listings}
\lstset{language=Python, basicstyle=\ttfamily\bfseries,
commentstyle=\color{red}\itshape, stringstyle=\color{darkgreen},
showstringspaces=false, keywordstyle=\color{blue}\bfseries}
\providecommand{\alert}[1]{\textbf{#1}}

\title{Getting Started -- \texttt{ipython}}
\author{FOSSEE}
\date{}

\usetheme{Warsaw}\usecolortheme{default}\useoutertheme{infolines}\setbeamercovered{transparent}
\begin{document}


\begin{frame}
  \frametitle{}
    \begin{center}
      \textcolor{blue}{Getting Started -- \texttt{ipython}}
    \end{center}
    \begin{center}
      \includegraphics[scale=0.25]{../images/iitb-logo.png}\\
      Developed by FOSSEE Team, IIT-Bombay. \\ 
      Funded by National Mission on Education through ICT

      MHRD, Govt. of India
    \end{center}
\end{frame}

\begin{frame}
\frametitle{Objectives}
\label{sec-1}
At the end of this tutorial, you will be able to --
\begin{itemize}
\item invoke the \texttt{ipython} interpreter
\item quit the \texttt{ipython} interpreter
\item navigate the \texttt{ipython} session history 
\item use tab-completion 
\item look-up documentation of functions
\item interrupt incomplete or incorrect commands
\end{itemize}
\end{frame}

\begin{frame}
\frametitle{Question 1}
\label{sec-2}
\begin{enumerate}
\item Type \texttt{ab} and hit tab to see what happens.
\item Next, just type \texttt{a} and hit tab to see what happens.
\end{enumerate}
\end{frame}

\begin{frame}
\frametitle{Solution 1}
\label{sec-3}
\begin{enumerate}
\item \texttt{ab} tab completes to \texttt{abs}
\item \texttt{a<tab>} gives us a list of all the commands starting
  with a.
\end{enumerate}
\end{frame}

\begin{frame}
\frametitle{Question 2}
\label{sec-4}

  Look-up the documentation of \texttt{round} and see how to use it.
\end{frame}
\begin{frame}
\frametitle{Solution 2}
\label{sec-5}

  \texttt{round?}
\end{frame}
\begin{frame}[fragile]
\frametitle{Question 3}
\label{sec-6}

  Check the output of
\begin{verbatim}
round(2.48)
round(2.48, 1)
round(2.48, 2)

round(2.484)
round(2.484, 1)
round(2.484, 2)
\end{verbatim}
  Look-up the documentation of \texttt{round} and see how to use it.
\end{frame}
\begin{frame}
\frametitle{Solution 3}
\label{sec-7}

  We get 2.0, 2.5 and 2.48, which are what we expect. 
\end{frame}
\begin{frame}
\frametitle{Question 4}
\label{sec-8}
\begin{enumerate}
\item Try typing \texttt{round(2.484}, and hit enter. and then cancel
  the command using Ctrl-C.
\item Then, type the command, \texttt{round(2.484, 2)} 
\end{enumerate}
\end{frame}
\begin{frame}[fragile]
\frametitle{Solution 4}
\label{sec-9}

\begin{verbatim}
round(2.484 
^C

round(2.484, 2)
\end{verbatim}
\end{frame}
\begin{frame}
\frametitle{Summary}
\label{sec-10}
In this tutorial, we have learnt to --
\begin{itemize}
\item invoke the IPython interpreter using \texttt{ipython} command 
\item quit \texttt{ipython} using \texttt{Ctrl-D}
\item navigate the history using arrow keys
\item use tab-completion to work faster
\item look up documentation of functions using \texttt{?}
\item send keyboard interrupts using \texttt{Ctrl-C}
\end{itemize}
\end{frame}
\begin{frame}
\frametitle{Evaluation}
\label{sec-11}

\begin{itemize}
\item \texttt{ipython} is a programming language similar to Python  \\
 True or False\\ 
 \vspace*{20pt}
 \item   Which key combination quits \texttt{ipython}?\\
 \vspace*{10pt}
  * Ctrl + C\\
  * Ctrl + D\\
  * Alt + C\\
  * Alt + D\\
\end{itemize} 
\end{frame}
\begin{frame}
\frametitle{Evaluation}
\label{sec-12}

\begin{itemize}
\item Which character is used at the end of a command, in \texttt{ipython} to display the documentation\\ 
\vspace*{10pt}
 * \_\\
 * ?\\
 * !\\
 * \&\\ 
\end{itemize}
\end{frame}
\begin{frame}
\frametitle{Solutions}
\label{sec-13}

\begin{itemize}
\item False
\vspace*{10pt}
\item Ctrl + C
\vspace*{10pt}
\item ?
\end{itemize}
\end{frame}
\begin{frame}
\frametitle{Acknowledgement}
\label{sec-14}

  \begin{block}{}
  \begin{center}
  \textcolor{blue}{\Large THANK YOU!} 
  \end{center}
  \end{block}
\begin{block}{}
  \begin{center}
    For more Information, visit our website\\
    \url{http://fossee.in/}
  \end{center}  
  \end{block}
\end{frame}

\end{document}
