% Created 2011-05-24 Tue 11:37
\documentclass[presentation]{beamer}
\usepackage[utf8]{inputenc}
\usepackage[T1]{fontenc}
\usepackage{fixltx2e}
\usepackage{graphicx}
\usepackage{longtable}
\usepackage{float}
\usepackage{wrapfig}
\usepackage{soul}
\usepackage{textcomp}
\usepackage{marvosym}
\usepackage{wasysym}
\usepackage{latexsym}
\usepackage{amssymb}
\usepackage{hyperref}
\tolerance=1000
\usepackage[english]{babel} \usepackage{ae,aecompl}
\usepackage{mathpazo,courier,euler} \usepackage[scaled=.95]{helvet}
\usepackage{listings}
\lstset{language=Python, basicstyle=\ttfamily\bfseries,
commentstyle=\color{red}\itshape, stringstyle=\color{darkgreen},
showstringspaces=false, keywordstyle=\color{blue}\bfseries}
\providecommand{\alert}[1]{\textbf{#1}}

\title{}
\author{FOSSEE}
\date{}

\usetheme{Warsaw}\usecolortheme{default}\useoutertheme{infolines}\setbeamercovered{transparent}
\begin{document}











\begin{frame}

\begin{center}
\vspace{12pt}
\textcolor{blue}{\huge Getting started with \texttt{ipython}}
\end{center}
\vspace{18pt}
\begin{center}
\vspace{10pt}
\includegraphics[scale=0.95]{../images/fossee-logo.png}\\
\vspace{5pt}
\scriptsize Developed by FOSSEE Team, IIT-Bombay. \\ 
\scriptsize Funded by National Mission on Education through ICT\\
\scriptsize  MHRD,Govt. of India\\
\includegraphics[scale=0.30]{../images/iitb-logo.png}\\
\end{center}
\end{frame}
\begin{frame}
\frametitle{Objectives}
\label{sec-2}

  At the end of this tutorial, you will be able to, 

\begin{itemize}
\item invoke the \verb~ipython~ interpreter.
\item quit the \verb~ipython~ interpreter.
\item navigate in the history of \verb~ipython~.
\item use tab-completion.
\item look-up documentation of functions.
\item interrupt incomplete or incorrect commands.
\end{itemize}
\end{frame}
\begin{frame}
\frametitle{Exercise 1}
\label{sec-3}


\begin{enumerate}
\item Type \texttt{ab} and hit tab to see what happens.
\item Next, just type \texttt{a} and hit tab to see what happens.
\end{enumerate}
\end{frame}
\begin{frame}
\frametitle{Exercise 2}
\label{sec-4}

  Look-up the documentation of \texttt{round} and see how to use it
\end{frame}
\begin{frame}[fragile]
\frametitle{Exercise 3}
\label{sec-5}

  Check the output of
\lstset{language=Python}
\begin{lstlisting}
round(2.48)
round(2.48, 1)
round(2.48, 2)

round(2.484)
round(2.484, 1)
round(2.484, 2)
\end{lstlisting}
\end{frame}
\begin{frame}
\frametitle{Solution 3}
\label{sec-6}

  We get 2.0, 2.5 and 2.48, which are what we expect. 
\end{frame}
\begin{frame}
\frametitle{Exercise 4}
\label{sec-7}


\begin{enumerate}
\item Type \texttt{round(2.484}, and hit enter and then cancel the command
     using Ctrl-C.
\item Then, type the command, \texttt{round(2.484, 2)} and resume
     the video.
\end{enumerate}
\end{frame}
\begin{frame}
\frametitle{Summary}
\label{sec-8}

  In this tutorial, we have learnt to –

\begin{itemize}
\item invoke the \verb~ipython~ interpreter by typing ipython.
\item quit the \verb~ipython~ interpreter by using \verb~Ctrl-d~.
\item navigate in the history of \verb~ipython~ by using the arrow keys.
\item use tab-completionto work faster.
\item see the documentation of functions using question mark.
\item interrupt using \verb~Ctrl-c~ when we make an error.
\end{itemize}
\end{frame}
\begin{frame}
\frametitle{Evaluation}
\label{sec-9}



\begin{enumerate}
\item ``ipython`` is a programming language similar to Python.
\begin{itemize}
\item True or False
\end{itemize}
\item Which key combination quits ``ipython``?
\begin{itemize}
\item Ctrl + C
\item Ctrl + D
\item Alt + C
\item Alt + D
\end{itemize}
\item Which character is used at the end of a command, in Ipython to
     display the documentation.
\begin{itemize}
\item under score (\_)
\item question mark (?)
\item exclamation mark (!)
\item ampersand (\&)
\end{itemize}
\end{enumerate}
\end{frame}
\begin{frame}
\frametitle{Solutions}
\label{sec-10}

   

\begin{enumerate}
\item False
\item Ctrl + D
\item question mark (?)
\end{enumerate}
\end{frame}
\begin{frame}

  \begin{block}{}
  \begin{center}
  \textcolor{blue}{\Large THANK YOU!} 
  \end{center}
  \end{block}
\begin{block}{}
  \begin{center}
    For more Information, visit our website\\
    \url{http://fossee.in/}
  \end{center}  
  \end{block}
\end{frame}

\end{document}