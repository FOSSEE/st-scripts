\documentclass[17pt]{beamer}
\usepackage{amsmath}
\usepackage{framed}
\definecolor{Blue}{RGB}{0.16,0.32,0.75}
\setbeamercolor{structure}{fg=blue}
\usepackage{beamerthemesplit}
\definecolor{blue}{rgb}{0.16,0.32,0.75}
\setbeamercolor{structure}{fg=blue}
\author[FOSSEE]{}
\institute[IIT Bombay]{}
\date[]{}
% \setbeamercovered{transparent}

% theme split
\usepackage{verbatim}
\newenvironment{colorverbatim}[1][]%
{%
\color{blue}
\verbatim
}%
{%
\endverbatim
}%

\usepackage{mathpazo,courier,euler}
\usepackage{listings}
\lstset{language=sh,
    basicstyle=\ttfamily\bfseries,
  showstringspaces=false,
  keywordstyle=\color{black}\bfseries}

% logo
\logo{\includegraphics[height=1.30 cm]{St-logo.png}}
\logo{\includegraphics[height=1.30 cm]{fossee-logo.png}

\hspace{7.5cm}
\includegraphics[scale=0.3]{fossee-logo.png}\\
\hspace{281pt}
\includegraphics[scale=0.08]{St-logo.png}}


\newcounter{saveenumi}
\newcommand{\seti}{\setcounter{saveenumi}{\value{enumi}}}
\newcommand{\conti}{\setcounter{enumi}{\value{saveenumi}}}

\begin{document}
% sf family, bold font
\sffamily \bfseries
%\LARGE
\title
[Python for Scientific Computing]
%\hspace{0.5cm}
%\insertframenumber/\inserttotalframenumber]
{Parsing Data}
\author
[FOSSEE, IIT BOMBAY]
{{\small Spoken Tutorial Project \\ http://spoken-tutorial.org \\ National Mission on Education  through ICT  \\ http://sakshat.ac.in } \\[0.1cm]
{\small  Script:Aditya Palaparthy}\\ 
{\small Narration: Kiran K}\\
{\small IIT Bombay} \\ [0.1cm]
{\small  23 October 2015}}
% slide 1
\begin{frame}
   \titlepage
\end{frame}
%%%%%%%%%%%%%%%%%%%%%%%%%%%%%%%%%%%%%%%%%%%%%%%%%%%%%%%%%%%%%%%%%%%%%%%%%%%%%%%%
\begin{frame}
\frametitle{Objectives}
\label{sec-2}
At the end of this tutorial, we will learn to-  \pause
\begin{itemize}
\item Split a string using a delimiter.\pause
\item Remove the whitespace around the string.\pause
\item Convert the datatypes of variables from one type to other.
\end{itemize}
\end{frame}
%%%%%%%%%%%%%%%%%%%%%%%%%%%%%%%%%%%%%%%%%%%%%%%%%%%%%%%%%%%%%%%%%%%%%%%%%%%%%%%%
\begin{frame}
\frametitle{System Specifications}\pause
\begin{itemize}
\item Ubuntu Linux 14.04\pause
\item \texttt{Python 2.7.6} \pause
\item \texttt{IPython 4.0.0}
\end{itemize}
\end{frame}
%%%%%%%%%%%%%%%%%%%%%%%%%%%%%%%%%%%%%%%%%%%%%%%%%%%%%%%%%%%%%%%%%%%%%%%%%%%%%%%%
\begin{frame}
\frametitle{Pre-requisite}
\label{sec-3}

To practise this tutorial, you should know how to
\begin{itemize}
\item run basic Python commands on the  ipython console
\item use lists.
\end{itemize}
If not, see the pre-requisite Python tutorials on {\color{blue} http://spoken-tutorial.org.}
\end{frame}
%%%%%%%%%%%%%%%%%%%%%%%%%%%%%%%%%%%%
\begin{frame}
\frametitle{Data set}
\begin{itemize}
\item File sslc.txt contains data as:
\item A;015163;JOSEPH RAJ S;083;042;47;00;72;244;;;
\end{itemize}

\end{frame}
%%%%%%%%%%%%%%%%%%%%%%%%%%%%%%%%%%%%
\begin{frame}
\frametitle{Data set}
\label{sec-4}
  The following are the fields in any given line.
\begin{itemize}
\begin{small}
\item Region Code which is `A'
\item Roll Number 015163
\item Name JOSEPH RAJ S
\item Marks of 5 subjects: -- English 083 -- 
     Hindi 042 -- Maths 47 --
     Science 00 -- Social 72
\item Total marks 244
\end{small}
\end{itemize}
\end{frame}
%%%%%%%%%%%%%%%%%%%%%%%%%%%%%%%%%%
\begin{frame}
\frametitle{Exercise 1}
\label{sec-5}
Split the variable line using a space as argument  \\
Is it same as splitting without an argument ?

\end{frame}
%%%%%%%%%%%%%%%%%%%%%%%%%%%%%%%%%%%

\begin{frame}
\frametitle{Exercise 2}
\label{sec-8}

  What happens if you do \texttt{int("1.25")}
\end{frame}

\begin{frame}
\frametitle{Summary}
\label{sec-9.1}
In this tutorial, we have learnt to,\pause
\begin{itemize}
\item Tokenize a string using various delimiters like semi-colons.\pause
\item Split a data separated by delimiters by using the function \texttt{split()}.\pause
\item Get rid of extra white spaces around using the \texttt{strip()} function.\pause
\end{itemize}
\end{frame}

\begin{frame}
\frametitle{Summary}
\label{sec-9.2}
\begin{itemize}
\item Convert datatypes of numbers from one type to another.\pause
\item Parse input data and perform computations on it.
\end{itemize}
\end{frame}

\begin{frame}
\frametitle{Evaluation}
\label{sec-10.1}

\begin{enumerate}
\item How do you split the string ``Guido;Rossum;Python'' to get the words.\pause
\item How will you remove the extra whitespace in this sentence\\
     ``~~~~~~Hello~~~~World~~~~~~``
\end{enumerate}
\end{frame}

\begin{frame}
\frametitle{Evaluation}
\label{sec-10.2}
\begin{enumerate}    
\setcounter{enumi}{2}
\item What does int(``20.0'') produce.\pause
\begin{itemize}
\item 20
\item 20.0
\item Error
\item ``20''
\end{itemize}
\end{enumerate}
\end{frame}

\begin{frame}
\frametitle{Solutions}
\label{sec-11}


\begin{enumerate}
\item line.split(';')\pause
\item ''~~~~~~Hello~~~~World~~~~~~''.strip()\pause
\item Error
\end{enumerate}
\end{frame}

%%%%%%%%%%%%%%%%%%%%%%%%%%%%%%%%%%%%%%%%%%%%%%%%%%%%%%%%%%%%%%%%%%%%%%%%%%%%%%%%
\begin{frame}
\frametitle{Forum to answer questions}
\begin{itemize}
\item Do you have questions in THIS Spoken Tutorial?
\item Choose the minute and second where you have the question.
\item Explain your question briefly.
\item Someone from the FOSSEE team will answer them. Please visit 
\end{itemize}
\begin{center}
{\color{blue}{http://forums.spoken-tutorial.org/}}
 \end{center} 
\end{frame}
%%%%%%%%%%%%%%%%%%%%%%%%%%%%%%%%%%%%%%%%%%%%%%%%%%%%%%%%%%%%%%%%%%%%%%%%%%%%%%%%
\begin{frame}
\frametitle{Forum to answer questions}
\begin{itemize}
\item Questions not related to the Spoken Tutorial?
\item Do you have general / technical questions on the Software?
\item Please visit the FOSSEE Forum
\begin{center}
{\color{blue}{http://forums.fossee.in/}}
 \end{center}
\item Choose the Software and post your question.
\end{itemize}
\end{frame}
%%%%%%%%%%%%%%%%%%%%%%%%%%%%%%%%%%%%%%%%%%%%%%%%%%%%%%%%%%%%%%%%%%

\begin{frame}
\frametitle{Textbook Companion Project}
\begin{itemize}
\item The FOSSEE team coordinates coding of solved examples of popular
  books 
\item We give honorarium and certificate to those who do this
\end{itemize}
For more details, please visit this site:
\begin{center}
{\color{blue}{http://tbc-python.fossee.in/}}
\end{center}
\end{frame}
%%%%%%%%%%%%%%%%%%%%%%%%%%%%%%%%%%%%%%%%%%%%%%%%%%%%%%%%%%%%%%%%%%%%%%%%%%%%%%%%
\begin{frame}
\frametitle{Acknowledgements}
\begin{itemize}
\item Spoken Tutorial Project is a part of the Talk to a Teacher  project 
\item It is supported by the National Mission on Education through  ICT, MHRD, Government of India 
\item More information on this Mission is available at: \\{\color{blue}\url{http://spoken-tutorial.org/NMEICT-Intro}}
\end{itemize}
\end{frame}
%%%%%%%%%%%%%%%%%%%%%%%%%%%%%%%%%%%%%%%%%%%%%%%%%%%%%%%%%%%%%%%%%%%%%%%%%%%%%%%%
\begin{frame}

  \begin{block}{}
  \begin{center}
  \textcolor{blue}{\Large THANK YOU!} 
  \end{center}
  \end{block}
\begin{block}{}
  \begin{center}
    For more Information, visit our website\\
    {http://fossee.in/}
  \end{center}  
  \end{block}
\end{frame}


\end{document}
