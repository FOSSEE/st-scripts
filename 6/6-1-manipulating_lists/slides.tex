\documentclass[17pt]{beamer}
\usepackage{amsmath}
\usepackage{framed}
\definecolor{Blue}{RGB}{0.16,0.32,0.75}
\setbeamercolor{structure}{fg=blue}
\usepackage{beamerthemesplit}
\definecolor{blue}{rgb}{0.16,0.32,0.75}
\setbeamercolor{structure}{fg=blue}
\author[FOSSEE]{}
\institute[IIT Bombay]{}
\date[]{}
% \setbeamercovered{transparent}

% theme split
\usepackage{verbatim}
\newenvironment{colorverbatim}[1][]%
{%
\color{blue}
\verbatim
}%
{%
\endverbatim
}%

\usepackage{mathpazo,courier,euler}
\usepackage{listings}
\lstset{language=sh,
    basicstyle=\ttfamily\bfseries,
  showstringspaces=false,
  keywordstyle=\color{black}\bfseries}

% logo
\logo{\includegraphics[height=1.30 cm]{St-logo.png}}
\logo{\includegraphics[height=1.30 cm]{fossee-logo.png}

\hspace{7.5cm}
\includegraphics[scale=0.3]{fossee-logo.png}\\
\hspace{281pt}
\includegraphics[scale=0.08]{St-logo.png}}


\newcounter{saveenumi}
\newcommand{\seti}{\setcounter{saveenumi}{\value{enumi}}}
\newcommand{\conti}{\setcounter{enumi}{\value{saveenumi}}}

\begin{document}
% sf family, bold font
\sffamily \bfseries
%\LARGE
\title
[Python for Scientific Computing]
%\hspace{0.5cm}
%\insertframenumber/\inserttotalframenumber]
{\large Mainpulating Lists}
\author
[FOSSEE, IIT BOMBAY]
{{\small Spoken Tutorial Project \\ http://spoken-tutorial.org \\ National Mission on Education  through ICT  \\ http://sakshat.ac.in } \\[0.1cm]
{\small  Script: Aditya Palaparthy}\\
{\small Narration : Kiran K}\\
{\small IIT Bombay} \\ [0.1cm]
{\small  23 October 2015}}
% slide 1
\begin{frame}
   \titlepage
\end{frame}
%%%%%%%%%%%%%%%%%%%%%%%%%%%%%%%%%%%%%%%%%%%%%%%%%%%%%%%%%%%%%%%%%%%%%%%%%%%%%%%%
\begin{frame}
\frametitle{Objectives}
\label{sec-2}

  In this tutorial we will learn, \pause


\begin{itemize}
\item Learn the details of slicing and striding of lists.\pause
\item Sort and reverse lists.
\end{itemize}
\end{frame}
%%%%%%%%%%%%%%%%%%%%%%%%%%%%%%%%%%%%%%%%%%%%%%%%%%%%%%%%%%%%%%%%%%%%%%%%%%%%%%%%
\begin{frame}
\frametitle{System Specifications}\pause
\begin{itemize}
\item Ubuntu Linux 14.04\pause
\item \texttt{Python 2.7.6} \pause
\item \texttt{IPython 4.0.0}
\end{itemize}
\end{frame}
%%%%%%%%%%%%%%%%%%%%%%%%%%%%%%%%%%%%%%%%%%%%%%%%%%%%%%%%%%%%%%%%%%%%%%%%%%%%%%%%
\begin{frame}
\frametitle{Pre-requisites}
To practise this tutorial,you should know how to
\begin{itemize}
\item run basic Python commands on the ipython console 
\item use lists
\end{itemize}
If not, see the pre-requisite Python tutorials on
{\color{blue}http://spoken-tutorial.org}
\end{frame}

\begin{frame}
\frametitle{Exercise 1}
\label{sec-4}

  Obtain the primes less than 10, from the list  \texttt{primes}. 
\end{frame}

\begin{frame}[fragile]
\frametitle{Slicing}
\label{sec-5}

\lstset{language=Python}
\begin{lstlisting}
p[start:stop]
\end{lstlisting}

\begin{itemize}
\item Returns all elements of \texttt{p} between \texttt{start} and \texttt{stop}
\item The element with index equal to \texttt{stop} is \textbf{not} included.
\end{itemize}
\end{frame}

\begin{frame}
\frametitle{Exercise 2}
\label{sec-6}

  Obtain all the multiples of three from the list \texttt{num}.
\end{frame}

\begin{frame}[fragile]
\frametitle{Solution 2}
\label{sec-7}

\lstset{language=Python}
\begin{lstlisting}
num[::3]
\end{lstlisting}
\end{frame}

\begin{frame}[fragile]
\frametitle{Exercise 3}
\label{sec-8}

  Given a list of marks of students in an examination, obtain a list
  with marks in descending order.
\lstset{language=Python}
\begin{small}
\begin{lstlisting}
marks = [99, 67, 47, 100, 50, 75, 62]
\end{lstlisting}
\end{small}
\end{frame}

\begin{frame}
\frametitle{Summary}
\label{sec-9}

  In this tutorial, we have learnt to,


\begin{itemize}
\item Obtain parts of lists using slicing and striding.\pause
\item Sort lists using the \texttt{sort} method.\pause
\item Use the method \texttt{reverse} to reverse the lists.
\end{itemize}
\end{frame}

\begin{frame}
\frametitle{Evaluation}
\label{sec-10.1}

\begin{enumerate}
\item Given the list primes,\\ \texttt{primes = [2, 3, 5, 7, 11, 13, 17, 19, 23,
   29]} \pause\\ How do you obtain the last 4 primes? \pause
\vspace{11pt}   
\item Given a list, p, of unknown length, obtain the first 3 (or all, if
   there are fewer) characters of it.
\seti
\end{enumerate}
\end{frame}


\begin{frame}
\frametitle{Solutions}
\label{sec-11}


\begin{enumerate}
\item \texttt{primes[-4:]}\pause
\vspace{8pt}
\item \texttt{p[:3]}\pause
\vspace{8pt}
\end{enumerate}
\end{frame}

%%%%%%%%%%%%%%%%%%%%%%%%%%%%%%%%%%%%%%%%%%%%%%%%%%%%%%%%%%%%%%%%%%%%%%%%%%%%%%%%
\begin{frame}
\frametitle{Forum to ask questions}
\begin{itemize}
\item Do you have questions in THIS Spoken Tutorial?
\item Choose the minute and second where you have the question.
\item Explain your question briefly.
\item Someone from the FOSSEE team will answer them. Please visit 
\end{itemize}
\begin{center}
{\color{blue}{http://forums.spoken-tutorial.org/}}
 \end{center} 
\end{frame}
%%%%%%%%%%%%%%%%%%%%%%%%%%%%%%%%%%%%%%%%%%%%%%%%%%%%%%%%%%%%%%%%%%%%%%%%%%%%%%%%
\begin{frame}
\frametitle{Forum to answer questions}
\begin{itemize}
\item Questions not related to the Spoken Tutorial?
\item Do you have general / technical questions on the Software?
\item Please visit the FOSSEE Forum
\begin{center}
{\color{blue}{http://forums.fossee.in/}}
 \end{center}
\item Choose the Software and post your question.
\end{itemize}
\end{frame}
%%%%%%%%%%%%%%%%%%%%%%%%%%%%%%%%%%%%%%%%%%%%%%%%%%%%%%%%%%%%%%%%%%%%%%%%%%%%%%%%
\begin{frame}
\frametitle{Textbook Companion Project}
\begin{itemize}
\item The FOSSEE team coordinates coding of solved examples of popular
  books 
\item We give honorarium and certificate to those who do this
\end{itemize}
For more details, please visit this site:
\begin{center}
{\color{blue}{http://tbc-python.fossee.in/}}
\end{center}
\end{frame}
%%%%%%%%%%%%%%%%%%%%%%%%%%%%%%%%%%%%%%%%%%%%%%%%%%%%%%%%%%%%%%%%%%%%%%%%%%%%%%%%
\begin{frame}
\frametitle{Acknowledgements}
\begin{itemize}
\item Spoken Tutorial Project is a part of the Talk to a Teacher  project 
\item It is supported by the National Mission on Education through  ICT, MHRD, Government of India 
\item More information on this Mission is available at: \\{\color{blue}\url{http://spoken-tutorial.org/NMEICT-Intro}}
\end{itemize}
\end{frame}
%%%%%%%%%%%%%%%%%%%%%%%%%%%%%%%%%%%%%%%%%%%%%%%%%%%%%%%%%%%%%%%%%%%%%%%%%%%%%%%%
\begin{frame}

  \begin{block}{}
  \begin{center}
  \textcolor{blue}{\Large THANK YOU!} 
  \end{center}
  \end{block}
\begin{block}{}
  \begin{center}
    For more Information, visit our website\\
    {http://fossee.in/}
  \end{center}  
  \end{block}
\end{frame}

\end{document}
