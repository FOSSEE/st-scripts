\documentclass[17pt]{beamer}
\usepackage{amsmath}
\usepackage{framed}
\definecolor{Blue}{RGB}{0.16,0.32,0.75}
\setbeamercolor{structure}{fg=blue}
\usepackage{beamerthemesplit}




\definecolor{blue}{rgb}{0.16,0.32,0.75}
\setbeamercolor{structure}{fg=blue}
\author[FOSSEE]{}
\institute[IIT Bombay]{}
\date[]{}
% \setbeamercovered{transparent}

% theme split
\usepackage{verbatim}
\newenvironment{colorverbatim}[1][]%
{%
\color{blue}
\verbatim
}%
{%
\endverbatim
}%

\usepackage{mathpazo,courier,euler}
\usepackage{listings}
\lstset{language=sh,
    basicstyle=\ttfamily\bfseries,
  showstringspaces=false,
  keywordstyle=\color{black}\bfseries}

% logo
\logo{\includegraphics[height=1.30 cm]{St-logo.png}}
\logo{\includegraphics[height=1.30 cm]{fossee-logo.png}

\hspace{7.5cm}
\includegraphics[scale=0.3]{fossee-logo.png}\\
\hspace{281pt}
\includegraphics[scale=0.08]{St-logo.png}}


\newcounter{saveenumi}
\newcommand{\seti}{\setcounter{saveenumi}{\value{enumi}}}
\newcommand{\conti}{\setcounter{enumi}{\value{saveenumi}}}

\begin{document}
% sf family, bold font
\sffamily \bfseries
%\LARGE
\title
[Python for Scientific Computing]
%\hspace{0.5cm}
%\insertframenumber/\inserttotalframenumber]
{\large Manipulating strings}
\author
[FOSSEE, IIT Bombay]
{{\small Spoken Tutorial Project \\ http://spoken-tutorial.org \\ National Mission on Education  through ICT  \\ http://sakshat.ac.in } \\
{\small Script: Thirumalesh H S}\\
{\small Narrator: Kiran Kishore}\\
{\small IIT Bombay}\\
{\small 19 November 2015}}

% slide 1
\begin{frame}
   \titlepage
\end{frame}

%%%%%%%%%%%%%%%%%%%%%%%%%%%%%%%%%%%%%%%%%%%%%%%%%%%%%%%%%%%%%%%%%%%%%%%%%%%%%%%%
\begin{frame}
\frametitle{Objectives}

  At the end of this tutorial, you will be able to, \pause


\begin{itemize}
\item Slice strings and get sub-strings out of them.\pause
\item Reverse strings.
\end{itemize}
\end{frame}
%%%%%%%%%%%%%%%%%%%%%%%%%%%%%%%%%%%%%%%%%%%%%%%%%%%%%%%%%%%%%%%%%%%%%%%%%%%%%%%%
\begin{frame}
\frametitle{Objectives}

\begin{itemize}
\item Replace characters in strings.\pause
\item Convert strings to upper or lower case.\pause
\item Join a list elements to form a string.
\end{itemize}
\end{frame}
%%%%%%%%%%%%%%%%%%%%%%%%%%%%%%%%%%%%%%%%%%%%%%%%%%%%%%%%%%%%%%%%%%%%%%%%%%%%%%%%

\begin{frame}
\frametitle{System Specifications}\pause
\begin{itemize}
\item Ubuntu Linux 14.04\pause
\item \texttt{Python 2.7.6} \pause
\item \texttt{IPython 4.0.0}
\end{itemize}
\end{frame}
%%%%%%%%%%%%%%%%%%%%%%%%%%%%%%%%%%%%%%%%%%%%%%%%%%%%%%%%%%%%%%%%%%%%%%%%%%%%%%%%
\begin{frame}
\frametitle{Pre-requisite}
\label{sec-3}

  To practice this tutorial, you should know how to -\pause

\begin{itemize}
\item use strings, lists and basic datatypes\pause
\end{itemize}
If not, see the pre-requisite Python tutorials on {\color{blue}http://spoken-tutorial.org}
\end{frame}
%%%%%%%%%%%%%%%%%%%%%%%%%%%%%%%%%%%%%%%%%%%%%%%%%%%%%%%%%%%%%%%%%%%%%%%%%%%%%%%%
\begin{frame}
\frametitle{string slicing}

\begin{itemize}
\item String slicing allows us to obtain sub-strings.\pause

\item Syntax:\\
string\_name[start: stop: [step]]
\end{itemize}
\end{frame}
%%%%%%%%%%%%%%%%%%%%%%%%%%%%%%%%%%%%%%%%%%%%%%%%%%%%%%%%%%%%%%%%%%%%%%%%%%%%%%%%
\begin{frame}
\frametitle{Assignment 1}

\begin{itemize}
\item Obtain the sub-string excluding the first and last characters from
  the string \texttt{s}.
\end{itemize}
\end{frame}
%%%%%%%%%%%%%%%%%%%%%%%%%%%%%%%%%%%%%%%%%%%%%%%%%%%%%%%%%%%%%%%%%%%%%%%%%%%%%%%%
\begin{frame}
\frametitle{Assignment 2}
\label{sec-6}

\begin{itemize}
\item Check if each element in list ['SATURDAY', 'Saturday'] is present in the list week. 
\end{itemize}  
\end{frame}
%%%%%%%%%%%%%%%%%%%%%%%%%%%%%%%%%%%%%%%%%%%%%%%%%%%%%%%%%%%%%%%%%%%%%%%%%%%%%%%%
\begin{frame}
\frametitle{Assignment 3}
\label{sec-7}


\begin{itemize}
\item Replace the [dot] with symbol '.' in email
\end{itemize}
\end{frame}
%%%%%%%%%%%%%%%%%%%%%%%%%%%%%%%%%%%%%%%%%%%%%%%%%%%%%%%%%%%%%%%%%%%%%%%%%%%%%%%%
\begin{frame}
\frametitle{Assignment 4}
\label{sec-8}


\begin{itemize}
\item From the email\_str that we generated, change the separator to be a semicolon instead of a comma.

\end{itemize}
\end{frame}
%%%%%%%%%%%%%%%%%%%%%%%%%%%%%%%%%%%%%%%%%%%%%%%%%%%%%%%%%%%%%%%%%%%%%%%%%%%%%%%%
\begin{frame}
\frametitle{Summary}
\label{sec-9.1}

  In this tutorial, we have learnt to, 

\begin{itemize}
\item Obtain sub-strings\pause 
\item Reversing strings by using the index numbers
\end{itemize}
\end{frame}
%%%%%%%%%%%%%%%%%%%%%%%%%%%%%%%%%%%%%%%%%%%%%%%%%%%%%%%%%%%%%%%%%%%%%%%%%%%%%%%%
\begin{frame}
\frametitle{Summary}
\label{sec-9.2}

\begin{itemize}
\item Use following functions -\pause
\begin{itemize}
\item \texttt{upper()} -- to obtain the upper case of a string\pause
\item \texttt{lower()} -- to obtain the lower case of a string\pause
\item \texttt{replace()} -- to replace a character by another one\pause
\item \texttt{.join()} -- to join a list of strings with an operator
\end{itemize}
\end{itemize}
\end{frame}
%%%%%%%%%%%%%%%%%%%%%%%%%%%%%%%%%%%%%%%%%%%%%%%%%%%%%%%%%%%%%%%%%%%%%%%%%%%%%%%%
\begin{frame}
\frametitle{Self assessment questions}
\label{sec-10}


\begin{enumerate}
\item Given a string \texttt{s = ``this is a string''}, how will you change it
   to \texttt{''this isn't a list''}\pause
\vspace{12pt}   
\item Given the string \texttt{''F.R.I.E.N.D.S''} in \texttt{s}, obtain the string \texttt{''friends''}.
\end{enumerate}
\end{frame}
%%%%%%%%%%%%%%%%%%%%%%%%%%%%%%%%%%%%%%%%%%%%%%%%%%%%%%%%%%%%%%%%%%%%%%%%%%%%%%%%
\begin{frame}[fragile]
\frametitle{Solutions of self assessment questions}
\label{sec-11}


\begin{enumerate}
\begin{small}
\item \lstset{language=Python}
\begin{lstlisting}
s = s.replace("string", "list")
s = s.replace("is", "isn't")
s
\end{lstlisting}\pause
\vspace{8pt}
\item \texttt{s[::2].lower()}
\end{small}
\end{enumerate}
\end{frame}
%%%%%%%%%%%%%%%%%%%%%%%%%%%%%%%%%%%%%%%%%%%%%%%%%%%%%%%%%%%%%%%%%%%%%%%%%%%%%%%%
\begin{frame}
\frametitle{Forum to answer questions}
\begin{itemize}
\item Do you have questions in THIS Spoken Tutorial?
\item Choose the minute and second where you have the question.
\item Explain your question briefly.
\item Someone from the FOSSEE team will answer them. Please visit 
\end{itemize}
\begin{center}
{\color{blue}{http://forums.spoken-tutorial.org/}}
 \end{center} 
\end{frame}
%%%%%%%%%%%%%%%%%%%%%%%%%%%%%%%%%%%%%%%%%%%%%%%%%%%%%%%%%%%%%%%%%%%%%%%%%%%%%%%%
\begin{frame}
\frametitle{Forum to answer questions}
\begin{itemize}
\item Questions not related to the Spoken Tutorial?
\item Do you have general / technical questions on the Software?
\item Please visit the FOSSEE Forum
\begin{center}
{\color{blue}{http://forums.fossee.in/}}
 \end{center}
\item Choose the Software and post your question.
\end{itemize}
\end{frame}
%%%%%%%%%%%%%%%%%%%%%%%%%%%%%%%%%%%%%%%%%%%%%%%%%%%%%%%%%%%%%%%%%%%%%%%%%%%%%%%%
\begin{frame}
\frametitle{Textbook Companion Project}
\begin{itemize}
\item The FOSSEE team coordinates coding of solved examples of popular
  books 
\item We give honorarium and certificate to those who do this
\end{itemize}
For more details, please visit this site:
\begin{center}
{\color{blue}{http://tbc-python.fossee.in/}}
\end{center}
\end{frame}
%%%%%%%%%%%%%%%%%%%%%%%%%%%%%%%%%%%%%%%%%%%%%%%%%%%%%%%%%%%%%%%%%%%%%%%%%%%%%%%%
\begin{frame}
\frametitle{Acknowledgements}
\begin{itemize}
\item Spoken Tutorial Project is a part of the Talk to a Teacher  project 
\item It is supported by the National Mission on Education through  ICT, MHRD, Government of India 
\item More information on this Mission is available at: \\{\color{blue}\url{http://spoken-tutorial.org/NMEICT-Intro}}
\end{itemize}
\end{frame}
%%%%%%%%%%%%%%%%%%%%%%%%%%%%%%%%%%%%%%%%%%%%%%%%%%%%%%%%%%%%%%%%%%%%%%%%%%%%%%%%
\begin{frame}

  \begin{block}{}
  \begin{center}
  \textcolor{blue}{\Large THANK YOU!} 
  \end{center}
  \end{block}
\begin{block}{}
  \begin{center}
    For more Information, visit our website\\
    {http://fossee.in/}
  \end{center}  
  \end{block}
\end{frame}

\end{document}
