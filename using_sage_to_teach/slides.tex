% Created 2011-06-16 Thu 12:19
\documentclass[presentation]{beamer}
\usepackage[latin1]{inputenc}
\usepackage[T1]{fontenc}
\usepackage{fixltx2e}
\usepackage{graphicx}
\usepackage{longtable}
\usepackage{float}
\usepackage{wrapfig}
\usepackage{soul}
\usepackage{textcomp}
\usepackage{marvosym}
\usepackage{wasysym}
\usepackage{latexsym}
\usepackage{amssymb}
\usepackage{hyperref}
\tolerance=1000
\usepackage[english]{babel} \usepackage{ae,aecompl}
\usepackage{mathpazo,courier,euler} \usepackage[scaled=.95]{helvet}
\usepackage{listings}
\lstset{language=Python, basicstyle=\ttfamily\bfseries,
commentstyle=\color{red}\itshape, stringstyle=\color{darkgreen},
showstringspaces=false, keywordstyle=\color{blue}\bfseries}
\providecommand{\alert}[1]{\textbf{#1}}

\title{}
\author{FOSSEE}
\date{}

\usetheme{Warsaw}\usecolortheme{default}\useoutertheme{infolines}\setbeamercovered{transparent}
\begin{document}











\begin{frame}

\begin{center}
\vspace{12pt}
\textcolor{blue}{\huge Using Sage to teach}
\end{center}
\vspace{18pt}
\begin{center}
\vspace{10pt}
\includegraphics[scale=0.95]{../images/fossee-logo.png}\\
\vspace{5pt}
\scriptsize Developed by FOSSEE Team, IIT-Bombay. \\ 
\scriptsize Funded by National Mission on Education through ICT\\
\scriptsize  MHRD,Govt. of India\\
\includegraphics[scale=0.30]{../images/iitb-logo.png}\\
\end{center}
\end{frame}
\begin{frame}
\frametitle{Objectives}
\label{sec-2}

 At the end of this tutorial, you will be able to,


\begin{itemize}
\item Use ``@interact'' feature of SAGE for better demonstration.
\item Share, publish and edit SAGE worksheets for collaborative learning.
\end{itemize}
\end{frame}
\begin{frame}
\frametitle{Pre-requisite}
\label{sec-3}

  Spoken tuorial on -

\begin{itemize}
\item Getting started with Sage.
\item Getting started with Symbolics.
\end{itemize}
\end{frame}
\begin{frame}
\frametitle{Exercise 1}
\label{sec-4}


\begin{itemize}
\item Plot the sine curve and vary its frequency using the ``@interact'' feature.
\end{itemize}
\end{frame}
\begin{frame}
\frametitle{Exercise 2}
\label{sec-5}


\begin{itemize}
\item Take a string as input from user and circular shift it to the left and
  vary the shift length using a slider.
\end{itemize}
\end{frame}
\begin{frame}
\frametitle{Summary}
\label{sec-6}

  In this tutorial,we have learnt to,


\begin{itemize}
\item Use interactive feaures of SAGE using ``@interact''.
\item Publish our work.
\item Edit a copy of one of the published worksheets.
\item Share the worksheets with fellow users.
\end{itemize}
\end{frame}
\begin{frame}
\frametitle{Evaluation}
\label{sec-7}


\begin{enumerate}
\item Which default argument, when used with ``@interact'' gives a slider 
    starting at 0 and ending in 10.
\begin{itemize}
\item (0..11)
\item range(0, 11)
\item~[0, 1, 2, 3, 4, 5, 6, 7, 8, 9, 10]~
\item (0..10)
\end{itemize}
\vspace{5pt}
\item What is the input widget resulted by using ``n = [2, 4, 5, 9]'' in the
    default arguments along with ``@interact''.
\begin{itemize}
\item input field
\item set of buttons
\item slider
\item None
\end{itemize}
\end{enumerate}
\end{frame}
\begin{frame}
\frametitle{Solutions}
\label{sec-8}


\begin{enumerate}
\item (0..10)
\vspace{12pt}
\item Set of buttons
\end{enumerate}
\end{frame}
\begin{frame}

  \begin{block}{}
  \begin{center}
  \textcolor{blue}{\Large THANK YOU!} 
  \end{center}
  \end{block}
\begin{block}{}
  \begin{center}
    For more Information, visit our website\\
    \url{http://fossee.in/}
  \end{center}  
  \end{block}
\end{frame}

\end{document}