% Created 2011-06-21 Tue 12:19
\documentclass[presentation]{beamer}
\usepackage[latin1]{inputenc}
\usepackage[T1]{fontenc}
\usepackage{fixltx2e}
\usepackage{graphicx}
\usepackage{longtable}
\usepackage{float}
\usepackage{wrapfig}
\usepackage{soul}
\usepackage{textcomp}
\usepackage{marvosym}
\usepackage{wasysym}
\usepackage{latexsym}
\usepackage{amssymb}
\usepackage{hyperref}
\tolerance=1000
\usepackage[english]{babel} \usepackage{ae,aecompl}
\usepackage{mathpazo,courier,euler} \usepackage[scaled=.95]{helvet}
\usepackage{listings}
\lstset{language=Python, basicstyle=\ttfamily\bfseries,
commentstyle=\color{red}\itshape, stringstyle=\color{darkgreen},
showstringspaces=false, keywordstyle=\color{blue}\bfseries}
\providecommand{\alert}[1]{\textbf{#1}}

\title{}
\author{FOSSEE}
\date{}

\usetheme{Warsaw}\usecolortheme{default}\useoutertheme{infolines}\setbeamercovered{transparent}
\begin{document}











\begin{frame}

\begin{center}
\vspace{12pt}
\textcolor{blue}{\huge Input/Output}
\end{center}
\vspace{18pt}
\begin{center}
\vspace{10pt}
\includegraphics[scale=0.95]{../images/fossee-logo.png}\\
\vspace{5pt}
\scriptsize Developed by FOSSEE Team, IIT-Bombay. \\ 
\scriptsize Funded by National Mission on Education through ICT\\
\scriptsize  MHRD,Govt. of India\\
\includegraphics[scale=0.30]{../images/iitb-logo.png}\\
\end{center}
\end{frame}
\begin{frame}
\frametitle{Objectives}
\label{sec-2}

At the end of this tutorial,you will be able to, 


\begin{itemize}
\item Print some value.
\item Print using modifiers.
\item Take input from user.
\item Display a prompt to the user before taking the input.
\end{itemize}
   
\end{frame}
\begin{frame}
\frametitle{Exercise 1}
\label{sec-3}


\begin{itemize}
\item What happens when you do\\
  \verb~print ``x is \%d, y is \%f''  \%(x, y)~
\end{itemize}
\end{frame}
\begin{frame}
\frametitle{Exercise 2}
\label{sec-4}


\begin{itemize}
\item Enter the number 5.6 as input and store it in a variable called
  \verb~c~.
\end{itemize}
\end{frame}
\begin{frame}
\frametitle{Exercise 3}
\label{sec-5}


\begin{itemize}
\item What happens when you do not enter anything and hit enter.
\end{itemize}
\end{frame}
\begin{frame}
\frametitle{Exercise 4}
\label{sec-6}


\begin{itemize}
\item How do you display a prompt and let the user enter input in a new line.
\end{itemize}
\end{frame}
\begin{frame}
\frametitle{Summary}
\label{sec-7}

 In this tutorial, we have learnt to,


\begin{itemize}
\item Use the print statement.
\item Use the modifiers \%d, \%f, \%s in the print statement.
\item Take input from user by using ``raw\_input()''.
\item Display a prompt to the user before taking the input by passing 
    a string as an argument to ``raw\_input''.
\end{itemize}
\end{frame}
\begin{frame}
\frametitle{Evaluation}
\label{sec-8}


\begin{enumerate}
\item ``a = raw\_input()'' and user enters ``2.5''.\\
   What is the type of a?
\begin{itemize}
\item str
\item int
\item float
\item char
\end{itemize}
\vspace{5pt}
\item ``a = 2'' and ``b = 4.5''. 
   What is the result of the following action.\\
  \verb~print ``a is \%d and b is \%2.1f'' \%(b,a)~ 
\begin{itemize}
\item a is 2 and b is 4.5
\item a is 4 and b is 2
\item a is 4 and b is 2.0
\item a is 4.5 and b is 2
\end{itemize}
\end{enumerate}
\end{frame}
\begin{frame}
\frametitle{Solutions}
\label{sec-9}


\begin{enumerate}
\item str
\vspace{12pt}
\item a is 4 and b is 2.0
\end{enumerate}
\end{frame}
\begin{frame}

  \begin{block}{}
  \begin{center}
  \textcolor{blue}{\Large THANK YOU!} 
  \end{center}
  \end{block}
\begin{block}{}
  \begin{center}
    For more Information, visit our website\\
    \url{http://fossee.in/}
  \end{center}  
  \end{block}
\end{frame}

\end{document}