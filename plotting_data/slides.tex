% Created 2011-05-06 Fri 17:36
\documentclass[presentation]{beamer}
\usepackage[utf8]{inputenc}
\usepackage[T1]{fontenc}
\usepackage{fixltx2e}
\usepackage{graphicx}
\usepackage{longtable}
\usepackage{float}
\usepackage{wrapfig}
\usepackage{soul}
\usepackage{textcomp}
\usepackage{marvosym}
\usepackage{wasysym}
\usepackage{latexsym}
\usepackage{amssymb}
\usepackage{hyperref}
\tolerance=1000
\usepackage[english]{babel} \usepackage{ae,aecompl}
\usepackage{mathpazo,courier,euler} \usepackage[scaled=.95]{helvet}
\usepackage{listings}
\lstset{language=Python, basicstyle=\ttfamily\bfseries,
commentstyle=\color{red}\itshape, stringstyle=\color{darkgreen},
showstringspaces=false, keywordstyle=\color{blue}\bfseries}
\providecommand{\alert}[1]{\textbf{#1}}

\title{}
\author{FOSSEE}
\date{}

\usetheme{Warsaw}\usecolortheme{default}\useoutertheme{infolines}\setbeamercovered{transparent}
\begin{document}











\begin{frame}

\begin{center}
\textcolor{blue}{Plotting Data}
\end{center}
 \begin{center}
\includegraphics[scale=0.25]{../images/iitb-logo.png}\\
Developed by FOSSEE Team, IIT-Bombay. \\ 
Funded by National Mission on Education through ICT

MHRD, Govt. of India
\end{center}
\end{frame}
\begin{frame}
\frametitle{Objectives}
\label{sec-2}

  At the end of this tutorial, you will be able to,

\begin{itemize}
\item Define a list of numbers.
\item Perform elementwise squaring of the list.
\item Plot data points.
\item Plot errorbars.
\end{itemize}
  
\end{frame}
\begin{frame}
\frametitle{Simple Pendulum Data}
\label{sec-3}




\begin{center}
\begin{tabular}{rr}
   L  &     T  \\
 0.1  &  0.69  \\
 0.2  &  0.90  \\
 0.3  &  1.19  \\
 0.4  &  1.30  \\
 0.5  &  1.47  \\
 0.6  &  1.58  \\
 0.7  &  1.77  \\
 0.8  &  1.83  \\
 0.9  &  1.94  \\
\end{tabular}
\end{center}


  
\end{frame}
\begin{frame}
\frametitle{Question 1}
\label{sec-4}

  Plot the given experimental data with large dots.
      
  
\end{frame}
\begin{frame}
\frametitle{Question 1 Data}
\label{sec-5}


    
  

\begin{center}
\begin{tabular}{rr}
    L  &     T  \\
 0.08  &  0.04  \\
 0.09  &  0.08  \\
 0.07  &  0.03  \\
 0.05  &  0.05  \\
 0.06  &  0.03  \\
 0.00  &  0.03  \\
 0.06  &  0.04  \\
 0.06  &  0.07  \\
 0.01  &  0.08  \\
\end{tabular}
\end{center}


    
\end{frame}
\begin{frame}
\frametitle{Question 2}
\label{sec-6}

  Plot the given experimental data with small dots.     
\end{frame}
\begin{frame}
\frametitle{Question 2 Data}
\label{sec-7}




\begin{center}
\begin{tabular}{rr}
    P  &     D  \\
 1.48  &  0.68  \\
 2.96  &  0.89  \\
 4.44  &  1.18  \\
 5.92  &  1.29  \\
 7.40  &  1.46  \\
 8.88  &  1.57  \\
 10.3  &  1.76  \\
 11.8  &  1.82  \\
 13.3  &  1.93  \\
\end{tabular}
\end{center}


  
\end{frame}
\begin{frame}
\frametitle{Summary}
\label{sec-8}

  In this tutorial, we have learnt to –

\begin{itemize}
\item Declare a sequence of numbers using the function ``array``.
\item Perform elemtwise squaring using the ``square`` function.
\item Use the various options available for plotting like dots,lines.
\item Plot experimental data such that we can also represent error by using the
    ``errorbar()`` function.
\end{itemize}
\end{frame}
\begin{frame}
\frametitle{Evaluation}
\label{sec-9}


\begin{enumerate}
\item Square the following sequence.
\begin{itemize}
\item distance\_values=[2.1,4.6,8.72,9.03]
\end{itemize}
\item Plot L v/s T in red plusses.
\end{enumerate}
\end{frame}
\begin{frame}
\frametitle{Solutions}
\label{sec-10}


\begin{enumerate}
\item square(distance\_values)
\item plot(L,T,'r+')
\end{enumerate}
\end{frame}
\begin{frame}
\frametitle{Acknowledgement}
\label{sec-11}

  \begin{block}{}
  \begin{center}
  \textcolor{blue}{\Large THANK YOU!} 
  \end{center}
  \end{block}
\begin{block}{}
  \begin{center}
    For more Information, visit our website\\
    \url{http://fossee.in/}
  \end{center}  
  \end{block}                                                                                                                                             
\end{frame}

\end{document}