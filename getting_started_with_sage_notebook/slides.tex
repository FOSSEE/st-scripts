% Created 2010-11-11 Thu 02:40
\documentclass[presentation]{beamer}
\usepackage[latin1]{inputenc}
\usepackage[T1]{fontenc}
\usepackage{fixltx2e}
\usepackage{graphicx}
\usepackage{longtable}
\usepackage{float}
\usepackage{wrapfig}
\usepackage{soul}
\usepackage{textcomp}
\usepackage{marvosym}
\usepackage{wasysym}
\usepackage{latexsym}
\usepackage{amssymb}
\usepackage{hyperref}
\tolerance=1000
\usepackage[english]{babel} \usepackage{ae,aecompl}
\usepackage{mathpazo,courier,euler} \usepackage[scaled=.95]{helvet}
\usepackage{listings}
\lstset{language=Python, basicstyle=\ttfamily\bfseries,
commentstyle=\color{red}\itshape, stringstyle=\color{darkgreen},
showstringspaces=false, keywordstyle=\color{blue}\bfseries}
\providecommand{\alert}[1]{\textbf{#1}}

\title{Getting started -- Sage}
\author{FOSSEE}
\date{}

\usetheme{Warsaw}\usecolortheme{default}\useoutertheme{infolines}\setbeamercovered{transparent}
\begin{document}

\maketitle









\begin{frame}
\frametitle{Outline}
\label{sec-1}

\begin{itemize}
\item Know what Sage and Sage notebook are.
\item Be able to start a Sage shell or notebook
\item Be able to start using the notebook
\item Be able to create new worksheets
\item Know about the menu options available
\item Know about the cells in the worksheet
\item Be able to evaluate cells, create and delete cells, navigate them.
\item Be able to make annotations in the worksheet
\item Be able to use tab completion.
\item Be able to use code from other languages in the cells.
\end{itemize}
\end{frame}
\begin{frame}
\frametitle{What is Sage?}
\label{sec-2}

\begin{itemize}
\item free, open-source mathematical software.
\item can do a lot of math for you, including, but not limited to

\begin{itemize}
\item algebra
\item geometry
\item cryptography
\item graph theory
\end{itemize}

\item can be used as aid in teaching and research
\end{itemize}
\end{frame}
\begin{frame}
\frametitle{Summary}
\label{sec-3}

\begin{itemize}
\item What is Sage
\item How to start Sage shell
\item What is Sage notebook
\item How to start the Sage notebook
\item How to create accounts and start using the notebook
\item How to create new worksheets
\item The menus available on the notebook
\item About cells in the worksheet
\item Methods to evaluate the cell, create new cells, delete the cells
    and navigate around the cells
\item To make annotations in the worksheet
\item Tab completions
\item And embedding code of other scripting languages in the cells
\end{itemize}
\end{frame}
\begin{frame}
\frametitle{Thank you!}
\label{sec-4}

  \begin{block}{}
  \begin{center}
  This spoken tutorial has been produced by the
  \textcolor{blue}{FOSSEE} team, which is funded by the 
  \end{center}
  \begin{center}
    \textcolor{blue}{National Mission on Education through \\
      Information \& Communication Technology \\ 
      MHRD, Govt. of India}.
  \end{center}  
  \end{block}
\end{frame}

\end{document}
