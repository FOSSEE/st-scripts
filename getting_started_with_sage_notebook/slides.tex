% Created 2011-06-13 Mon 14:38
\documentclass[presentation]{beamer}
\usepackage[latin1]{inputenc}
\usepackage[T1]{fontenc}
\usepackage{fixltx2e}
\usepackage{graphicx}
\usepackage{longtable}
\usepackage{float}
\usepackage{wrapfig}
\usepackage{soul}
\usepackage{textcomp}
\usepackage{marvosym}
\usepackage{wasysym}
\usepackage{latexsym}
\usepackage{amssymb}
\usepackage{hyperref}
\tolerance=1000
\usepackage[english]{babel} \usepackage{ae,aecompl}
\usepackage{mathpazo,courier,euler} \usepackage[scaled=.95]{helvet}
\usepackage{listings}
\lstset{language=Python, basicstyle=\ttfamily\bfseries,
commentstyle=\color{red}\itshape, stringstyle=\color{darkgreen},
showstringspaces=false, keywordstyle=\color{blue}\bfseries}
\providecommand{\alert}[1]{\textbf{#1}}

\title{}
\author{FOSSEE}
\date{}

\usetheme{Warsaw}\usecolortheme{default}\useoutertheme{infolines}\setbeamercovered{transparent}
\begin{document}











\begin{frame}

\begin{center}
\vspace{12pt}
\textcolor{blue}{\huge Getting started with Sage}
\end{center}
\vspace{18pt}
\begin{center}
\vspace{10pt}
\includegraphics[scale=0.95]{../images/fossee-logo.png}\\
\vspace{5pt}
\scriptsize Developed by FOSSEE Team, IIT-Bombay. \\ 
\scriptsize Funded by National Mission on Education through ICT\\
\scriptsize  MHRD,Govt. of India\\
\includegraphics[scale=0.30]{../images/iitb-logo.png}\\
\end{center}
\end{frame}
\begin{frame}
\frametitle{Objectives}
\label{sec-2}

  At the end of this tutorial, you will be able to,


\begin{itemize}
\item Know what Sage and Sage notebook are.
\item Start a Sage shell or notebook.
\item Create new worksheets.
\item Know about the menu options available and the cells in the worksheet.
\item Evaluate cells, create and delete cells, navigate them.
\item Make annotations in the worksheet.
\item Use tab completion.
\item Use code from other languages in the cells.
\item Use the offline help available.
\end{itemize}
\end{frame}
\begin{frame}
\frametitle{What is Sage?}
\label{sec-3}


\begin{itemize}
\item free, open-source mathematical software.
\item can do a lot of math for you, including, but not limited to
\begin{itemize}
\item algebra
\item geometry
\item cryptography
\item graph theory
\end{itemize}
\item can be used as aid in teaching and research
\end{itemize}
\end{frame}
\begin{frame}
\frametitle{Installing Sage}
\label{sec-4}

   Visit the page\\
\vspace{8pt}   
   \url{http://sagemath.org/doc/tutorial/introduction.html}\\
\vspace{5pt}   
   \url{http://sagemath.org/doc/tutorial/introduction.html}\\ 
\vspace{8pt}   
   for the tutorial on how to install Sage.
\end{frame}
\begin{frame}
\frametitle{Summary}
\label{sec-5}

 In thus tutorial, we have learnt to, 


\begin{itemize}
\item Know about Sage and sage notebook.
\item Start Sage shell  and sage notebook.
\item Create accounts and start using the notebook.
\item Create new worksheets.
\item Access the menus available on the notebook.
\item Evaluate cells in the worksheet.
\item Create new cells, delete the cells.
     and navigate around the cells.
\item Make annotations in the worksheet.
\item Use tab completions.
\item Embed code of other scripting languages in the cells.
\end{itemize}
\end{frame}
\begin{frame}
\frametitle{Evaluation}
\label{sec-6}


\begin{enumerate}
\item Each cell in a sage worksheet displays the result of only the last
     operation.\\
     True or False.
\vspace{12pt}
\item How do you evaluate a cell using the keyboard keys?
\begin{itemize}
\item Shift key along with enter key
\item Control key along with enter key
\item Alt key along with enter key
\end{itemize}
\end{enumerate}
\end{frame}
\begin{frame}
\frametitle{Solutions}
\label{sec-7}


\begin{enumerate}
\item True
\vspace{12pt}
\item Shift key along with enter key
\end{enumerate}
\end{frame}
\begin{frame}

  \begin{block}{}
  \begin{center}
  \textcolor{blue}{\Large THANK YOU!} 
  \end{center}
  \end{block}
\begin{block}{}
  \begin{center}
    For more Information, visit our website\\
    \url{http://fossee.in/}
  \end{center}  
  \end{block}
\end{frame}

\end{document}