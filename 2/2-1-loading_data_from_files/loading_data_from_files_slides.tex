\documentclass[17pt]{beamer}
\usepackage{amsmath}
\usepackage{framed}
\definecolor{Blue}{RGB}{0.16,0.32,0.75}
\setbeamercolor{structure}{fg=blue}
\usepackage{beamerthemesplit}




\definecolor{blue}{rgb}{0.16,0.32,0.75}
\setbeamercolor{structure}{fg=blue}
\author[FOSSEE]{}
\institute[IIT Bombay]{}
\date[]{}
% \setbeamercovered{transparent}

% theme split
\usepackage{verbatim}
\newenvironment{colorverbatim}[1][]%
{%
\color{blue}
\verbatim
}%
{%
\endverbatim
}%

\usepackage{mathpazo,courier,euler}
\usepackage{listings}
\lstset{language=sh,
    basicstyle=\ttfamily\bfseries,
  showstringspaces=false,
  keywordstyle=\color{black}\bfseries}

% logo
\logo{\includegraphics[height=1.30 cm]{St-logo.png}}
\logo{\includegraphics[height=1.30 cm]{fossee-logo.png}

\hspace{7.5cm}
\includegraphics[scale=0.3]{fossee-logo.png}\\
\hspace{281pt}
\includegraphics[scale=0.08]{St-logo.png}}


\newcounter{saveenumi}
\newcommand{\seti}{\setcounter{saveenumi}{\value{enumi}}}
\newcommand{\conti}{\setcounter{enumi}{\value{saveenumi}}}

\begin{document}
% sf family, bold font
\sffamily \bfseries
%\LARGE
\title
[Loading data from files]
%\hspace{0.5cm}
%\insertframenumber/\inserttotalframenumber]
{Loading data from files}
\author
[Usha Vishwanathan]
{{\small Spoken Tutorial Project \\ http://spoken-tutorial.org \\ National Mission on Education  through ICT  \\ http://sakshat.ac.in } \\[0.1cm]
{\small  Usha Vishwanathan}\\
{\small IIT Bombay} \\ [0.1cm]
{\small  16 October 2015}}

%%%%%%%%%%%%%%%%%%%%%%%%%%%%%%%%%%%%%%%%%%%%%%%%%%%%%%%%%%%%%%%%%%%%%%%
\begin{frame}
 \titlepage
\end{frame}
%%%%%%%%%%%%%%%%%%%%%%%%%%%%%%%%%%%%%%%%%%%%%%%%%%%%%%%%%%%%%%%%%%%%%%%
\begin{frame}
  \frametitle{Objectives}
  In this tutorial, you will learn to, \\
  Read data from files, which contain data in:
\begin{itemize}
\item Single column format \\ 
\item Multiple columns separated by spaces or other delimiters
  \end{itemize}
\end{frame}
%%%%%%%%%%%%%%%%%%%%%%%%%%%%%%%%%%%%%%%%%%%%%%%%%%%%%%%%%%%%%%%%%%%%%%%
\begin{frame}
\frametitle{System Specifications}\pause
\begin{itemize}
\item Ubunutu 14.04\pause
\item \texttt{Python 2.7.6} \pause
\item \texttt{IPython 3.0.0}
\end{itemize}
\end{frame}
%%%%%%%%%%%%%%%%%%%%%%%%%%%%%%%%%%%%%%%%%%%%%%%%%%%%%%%%%%%%%%%%%%%%%%%
\begin{frame}
  \frametitle{Assignment 1}

Read the data from the file \verb~pendulum\_semicolon.txt~. This file contains data in two columns. These columns are separated by semi-colons.\\
 Use the IPython help to see how to do this.
    
\end{frame}
%%%%%%%%%%%%%%%%%%%%%%%%%%%%%%%%%%%%%%%%%%%%%%%%%%%%%%%%%%%%%%%%%%%%%%%
\begin{frame}
  \frametitle{Summary}  
 To Read data from files using the \verb~loadtxt()~ command, the data can be in,
  \begin{itemize}
   \item A single column 
   \item Multiple columns format separated by spaces or other delimiters   
  \end{itemize}
\end{frame}

%%%%%%%%%%%%%%%%%%%%%%%%%%%%%%%%%%%%%%%%%%%%%%%%%%%%%%%%%%%%%%%%%%%%%%%
\begin{frame}
  \frametitle{Evaluation}
    \begin{itemize}
     \item \verb~loadtxt()~ can read data only from a file with one column.
     True or False?
     \item Given a file ``data.txt`` with three columns of data separated by spaces. Read it into 3 separate simple sequences.
    \end{itemize}
\end{frame}

%%%%%%%%%%%%%%%%%%%%%%%%%%%%%%%%%%%%%%%%%%%%%%%%%%%%%%%%%%%%%%%%%%%%%%%
\begin{frame}[fragile]
  \frametitle{Evaluation..}
  \begin{itemize}
  \item Given a file ``data.txt`` with three columns of data separated by ``:''. Read it into 3 separate simple sequences.
  \end{itemize}
\end{frame}
%%%%%%%%%%%%%%%%%%%%%%%%%%%%%%%%%%%%%%%%%%%%%%%%%%%%%%%%%%%%%%%%%%%%%%%
\begin{frame}
  \frametitle{Solutions}
    \begin{itemize}
    \item False
    \item x = loadtxt(``data.txt'', unpack=True)
    \item x = loadtxt(``data.txt'', unpack=True, delimiter='':'')
  \end{itemize} 
\end{frame}
%%%%%%%%%%%%%%%%%%%%%%%%%%%%%%%%%%%%%%%%%%%%%%%%%%%%%%%%%%%%%%%%%%%%%%%
\begin{frame}
\frametitle{About the Spoken Tutorial Project}
\begin{itemize}
\item Watch the video available at {\color{blue}http://spoken-tutorial.org /What\_is\_a\_Spoken\_Tutorial}
\item It summarises the Spoken Tutorial project \pause
\item If you do not have good bandwidth, you can download and watch it
\end{itemize}
\end{frame}
%%%%%%%%%%%%%%%%%%%%%%%%%%%%%%%%%%%%%%%%%%%%%%%%%%%%%%%%%%%%%%%%%%%%%%%%%%%%%%%%
\begin{frame}
\frametitle{Spoken Tutorial Workshops}The Spoken Tutorial Project Team 
\begin{itemize}
\item Conducts workshops using spoken tutorials 
\item Gives certificates to those who pass an online test 
\item For more details, please write to \\ \hspace {0.5cm}{\color{blue}contact@spoken-tutorial.org}
\end{itemize}
\end{frame}
%%%%%%%%%%%%%%%%%%%%%%%%%%%%%%%%%%%%%%%%%%%%%%%%%%%%%%%%%%%%%%%%%%%%%%%%%%%%%%%%
\begin{frame}
\frametitle{Forum to answer questions}
\begin{itemize}
\item Do you have questions in THIS Spoken Tutorial?
\item Choose the minute and second where you have the question.
\item Explain your question briefly.
\item Someone from the FOSSEE team will answer them. Please visit 
\end{itemize}
\begin{center}
{\color{blue}{http://forums.spoken-tutorial.org/}}
 \end{center} 
\end{frame}
%%%%%%%%%%%%%%%%%%%%%%%%%%%%%%%%%%%%%%%%%%%%%%%%%%%%%%%%%%%%%%%%%%%%%%%%%%%%%%%%
\begin{frame}
\frametitle{Forum to answer questions}
\begin{itemize}
\item Questions not related to the Spoken Tutorial?
\item Do you have general / technical questions on the Software?
\item Please visit the FOSSEE Forum
\begin{center}
{\color{blue}{http://forums.fossee.in/}}
 \end{center}
\item Choose the Software and post your question.
\end{itemize}
\end{frame}
%%%%%%%%%%%%%%%%%%%%%%%%%%%%%%%%%%%%%%%%%%%%%%%%%%%%%%%%%%%%%%%%%%%%%%%%%%%%%%%%
\begin{frame}
\frametitle{Textbook Companion Project}
\begin{itemize}
\item The FOSSEE team coordinates coding of solved examples of popular
  books 
\item We give honorarium and certificate to those who do this
\end{itemize}
For more details, please visit this site:
\begin{center}
{\color{blue}{http://tbc-python.fossee.in/}}
\end{center}
\end{frame}
%%%%%%%%%%%%%%%%%%%%%%%%%%%%%%%%%%%%%%%%%%%%%%%%%%%%%%%%%%%%%%%%%%%%%%%%%%%%%%%%
\begin{frame}
\frametitle{Acknowledgements}
\begin{itemize}
\item Spoken Tutorial Project is a part of the Talk to a Teacher  project 
\item It is supported by the National Mission on Education through  ICT, MHRD, Government of India 
\item More information on this Mission is available at: \\{\color{blue}\url{http://spoken-tutorial.org/NMEICT-Intro}}
\end{itemize}
\end{frame}
%%%%%%%%%%%%%%%%%%%%%%%%%%%%%%%%%%%%%%%%%%%%%%%%%%%%%%%%%%%%%%%%%%%%%%%%%%%%%%%%
\begin{frame}

  \begin{block}{}
  \begin{center}
  \textcolor{blue}{\Large THANK YOU!} 
  \end{center}
  \end{block}
\begin{block}{}
  \begin{center}
    For more Information, visit our website\\
    {http://fossee.in/}
  \end{center}  
  \end{block}
\end{frame}

\end{document}
