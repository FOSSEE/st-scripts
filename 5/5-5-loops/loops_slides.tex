\documentclass[17pt]{beamer}
\usepackage{amsmath}
\usepackage{framed}
\definecolor{Blue}{RGB}{0.16,0.32,0.75}
\setbeamercolor{structure}{fg=blue}
\usepackage{beamerthemesplit}




\definecolor{blue}{rgb}{0.16,0.32,0.75}
\setbeamercolor{structure}{fg=blue}
\author[FOSSEE]{}
\institute[IIT Bombay]{}
\date[]{}
% \setbeamercovered{transparent}

% theme split
\usepackage{verbatim}
\newenvironment{colorverbatim}[1][]%
{%
\color{blue}
\verbatim
}%
{%
\endverbatim
}%

\usepackage{mathpazo,courier,euler}
\usepackage{listings}
\lstset{language=sh,
    basicstyle=\ttfamily\bfseries,
  showstringspaces=false,
  keywordstyle=\color{black}\bfseries}

% logo
\logo{\includegraphics[height=1.30 cm]{St-logo.png}}
\logo{\includegraphics[height=1.30 cm]{fossee-logo.png}

\hspace{7.5cm}
\includegraphics[scale=0.3]{fossee-logo.png}\\
\hspace{281pt}
\includegraphics[scale=0.08]{St-logo.png}}


\newcounter{saveenumi}
\newcommand{\seti}{\setcounter{saveenumi}{\value{enumi}}}
\newcommand{\conti}{\setcounter{enumi}{\value{saveenumi}}}

\begin{document}
% sf family, bold font
\sffamily \bfseries
%\LARGE
\title
[Python for Scientific Computing]
%\hspace{0.5cm}
%\insertframenumber/\inserttotalframenumber]
{\large Loops}
\author
[FOSSEE, IIT Bombay]
{{\small Spoken Tutorial Project \\ http://spoken-tutorial.org \\ National Mission on Education  through ICT  \\ http://sakshat.ac.in } \\
{\small Script: Thirumalesh H S}\\
{\small Narrator: Kiran Kishore}\\
{\small IIT Bombay}\\
{\small 19 November 2015}}

% slide 1
\begin{frame}
   \titlepage
\end{frame}

%%%%%%%%%%%%%%%%%%%%%%%%%%%%%%%%%%%%%%%%%%%%%%%%%%%%%%%%%%%%%%%%%%%%%%%%%%%%%%%%
\begin{frame}
\frametitle{Objectives}
\label{sec-2}

  At the end of this tutorial, you will be able to, \pause


\begin{itemize}
\item use the \texttt{for} loop.\pause
\item use the \texttt{while} loop.\pause
\item Use \texttt{break}, \texttt{continue} and \texttt{pass} statements in the loops.
\end{itemize}
\end{frame}
%%%%%%%%%%%%%%%%%%%%%%%%%%%%%%%%%%%%%%%%%%%%%%%%%%%%%%%%%%%%%%%%%%%%%%%%%%%%%%%%
\begin{frame}
\frametitle{System Specifications}\pause
\begin{itemize}
\item Ubuntu Linux 14.04\pause
\item \texttt{Python 2.7.6} \pause
\item \texttt{IPython 4.0.0}
\end{itemize}
\end{frame}
%%%%%%%%%%%%%%%%%%%%%%%%%%%%%%%%%%%%%%%%%%%%%%%%%%%%%%%%%%%%%%%%%%%%%%%%%%%%%%%%
\begin{frame}
\frametitle{Pre-requisite}
\label{sec-3}

  To practice this tutorial, you should know how to -\pause

\begin{itemize}
\item use for statements\pause
\item conditionals
\end{itemize}
If not, see the pre-requisite Python tutorials on {\color{blue}http://spoken-tutorial.org}
\end{frame}
%%%%%%%%%%%%%%%%%%%%%%%%%%%%%%%%%%%%%%%%%%%%%%%%%%%%%%%%%%%%%%%%%%%%%%%%%%%%%%%%
\begin{frame}[fragile]
\frametitle{while loop}

Syntax:

\lstset{language=Python}
\begin{lstlisting}
while condition:
   True block statements
   update condition variable
\end{lstlisting}  
\end{frame}
%%%%%%%%%%%%%%%%%%%%%%%%%%%%%%%%%%%%%%%%%%%%%%%%%%%%%%%%%%%%%%%%%%%%%%%%%%%%%%%%
\begin{frame}
\frametitle{Assignment 1}
\label{sec-4}

  Write a \texttt{while} loop to print the squares of all the even
  numbers below 10. 
\end{frame}
%%%%%%%%%%%%%%%%%%%%%%%%%%%%%%%%%%%%%%%%%%%%%%%%%%%%%%%%%%%%%%%%%%%%%%%%%%%%%%%%
\begin{frame}
\frametitle{Assignment 2}
\label{sec-5}

  Write a \texttt{for} loop to print the squares of all the even numbers
  below 10.
\end{frame}
%%%%%%%%%%%%%%%%%%%%%%%%%%%%%%%%%%%%%%%%%%%%%%%%%%%%%%%%%%%%%%%%%%%%%%%%%%%%%%%%
\begin{frame}
\frametitle{Summary}
\label{sec-7}

 In this tutorial, we have learnt to,

\begin{itemize}
\item Iterate over a sequence using \texttt{for} and \texttt{while} loops.\pause
\item Break out of loops using \texttt{break} statement. \pause
\item Skip iterations using \texttt{continue} statement.\pause
\item Use the \texttt{pass} statement in a loop.
\end{itemize}
\end{frame}
%%%%%%%%%%%%%%%%%%%%%%%%%%%%%%%%%%%%%%%%%%%%%%%%%%%%%%%%%%%%%%%%%%%%%%%%%%%%%%%%
\begin{frame}
\frametitle{Self assessment questions}
\label{sec-8}


\begin{enumerate}
\item Given \texttt{range(1,4)} \\ Write a code to print only the number 1.\pause
\vspace{12pt}
\item Which statement do you use to skip iterations.\pause
\begin{itemize}
\item break
\item pass
\item continue
\end{itemize}
\end{enumerate}
\end{frame}
%%%%%%%%%%%%%%%%%%%%%%%%%%%%%%%%%%%%%%%%%%%%%%%%%%%%%%%%%%%%%%%%%%%%%%%%%%%%%%%%
\begin{frame}[fragile]
\frametitle{Solution of self assessment questions}
\label{sec-9}


\begin{enumerate}
\item \lstset{language=Python}
\begin{lstlisting}
for i in range(1,4):
    print i
    break
\end{lstlisting}
\vspace{8pt}
\item \texttt{continue}
\end{enumerate}
\end{frame}

%%%%%%%%%%%%%%%%%%%%%%%%%%%%%%%%%%%%%%%%%%%%%%%%%%%%%%%%%%%%%%%%%%%%%%%%%%%%%%%%
\begin{frame}
\frametitle{Forum to answer questions}
\begin{itemize}
\item Do you have questions in THIS Spoken Tutorial?
\item Choose the minute and second where you have the question.
\item Explain your question briefly.
\item Someone from the FOSSEE team will answer them. Please visit 
\end{itemize}
\begin{center}
{\color{blue}{http://forums.spoken-tutorial.org/}}
 \end{center} 
\end{frame}
%%%%%%%%%%%%%%%%%%%%%%%%%%%%%%%%%%%%%%%%%%%%%%%%%%%%%%%%%%%%%%%%%%%%%%%%%%%%%%%%
\begin{frame}
\frametitle{Forum to answer questions}
\begin{itemize}
\item Questions not related to the Spoken Tutorial?
\item Do you have general / technical questions on the Software?
\item Please visit the FOSSEE Forum
\begin{center}
{\color{blue}{http://forums.fossee.in/}}
 \end{center}
\item Choose the Software and post your question.
\end{itemize}
\end{frame}
%%%%%%%%%%%%%%%%%%%%%%%%%%%%%%%%%%%%%%%%%%%%%%%%%%%%%%%%%%%%%%%%%%%%%%%%%%%%%%%%
\begin{frame}
\frametitle{Textbook Companion Project}
\begin{itemize}
\item The FOSSEE team coordinates coding of solved examples of popular
  books 
\item We give honorarium and certificate to those who do this
\end{itemize}
For more details, please visit this site:
\begin{center}
{\color{blue}{http://tbc-python.fossee.in/}}
\end{center}
\end{frame}
%%%%%%%%%%%%%%%%%%%%%%%%%%%%%%%%%%%%%%%%%%%%%%%%%%%%%%%%%%%%%%%%%%%%%%%%%%%%%%%%
\begin{frame}
\frametitle{Acknowledgements}
\begin{itemize}
\item Spoken Tutorial Project is a part of the Talk to a Teacher  project 
\item It is supported by the National Mission on Education through  ICT, MHRD, Government of India 
\item More information on this Mission is available at: \\{\color{blue}\url{http://spoken-tutorial.org/NMEICT-Intro}}
\end{itemize}
\end{frame}
%%%%%%%%%%%%%%%%%%%%%%%%%%%%%%%%%%%%%%%%%%%%%%%%%%%%%%%%%%%%%%%%%%%%%%%%%%%%%%%%
\begin{frame}

  \begin{block}{}
  \begin{center}
  \textcolor{blue}{\Large THANK YOU!} 
  \end{center}
  \end{block}
\begin{block}{}
  \begin{center}
    For more Information, visit our website\\
    {http://fossee.in/}
  \end{center}  
  \end{block}
\end{frame}

\end{document}
