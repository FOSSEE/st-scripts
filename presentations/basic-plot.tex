%%%%%%%%%%%%%%%%%%%%%%%%%%%%%%%%%%%%%%%%%%%%%%%%%%%%%%%%%%%%%%%%%%%%%%%%%%%%%%%%
%Tutorial slides on Python.
%
% Author: FOSSEE 
% Copyright (c) 2009, FOSSEE, IIT Bombay
%%%%%%%%%%%%%%%%%%%%%%%%%%%%%%%%%%%%%%%%%%%%%%%%%%%%%%%%%%%%%%%%%%%%%%%%%%%%%%%%

\documentclass[14pt,compress]{beamer}
%\documentclass[draft]{beamer}
%\documentclass[compress,handout]{beamer}
%\usepackage{pgfpages} 
%\pgfpagesuselayout{2 on 1}[a4paper,border shrink=5mm]

% Modified from: generic-ornate-15min-45min.de.tex
\mode<presentation>
{
  \usetheme{Warsaw}
  \useoutertheme{infolines}
  \setbeamercovered{transparent}
}

\usepackage[english]{babel}
\usepackage[latin1]{inputenc}
%\usepackage{times}
\usepackage[T1]{fontenc}

% Taken from Fernando's slides.
\usepackage{ae,aecompl}
\usepackage{mathpazo,courier,euler}
\usepackage[scaled=.95]{helvet}

\definecolor{darkgreen}{rgb}{0,0.5,0}

\usepackage{listings}
\lstset{language=Python,
    basicstyle=\ttfamily\bfseries,
    commentstyle=\color{red}\itshape,
  stringstyle=\color{darkgreen},
  showstringspaces=false,
  keywordstyle=\color{blue}\bfseries}

%%%%%%%%%%%%%%%%%%%%%%%%%%%%%%%%%%%%%%%%%%%%%%%%%%%%%%%%%%%%%%%%%%%%%%
% Macros
\setbeamercolor{emphbar}{bg=blue!20, fg=black}
\newcommand{\emphbar}[1]
{\begin{beamercolorbox}[rounded=true]{emphbar} 
      {#1}
 \end{beamercolorbox}
}
\newcounter{time}
\setcounter{time}{0}
\newcommand{\inctime}[1]{\addtocounter{time}{#1}{\tiny \thetime\ m}}

\newcommand{\typ}[1]{\lstinline{#1}}

\newcommand{\kwrd}[1]{ \texttt{\textbf{\color{blue}{#1}}}  }

%%% This is from Fernando's setup.
% \usepackage{color}
% \definecolor{orange}{cmyk}{0,0.4,0.8,0.2}
% % Use and configure listings package for nicely formatted code
% \usepackage{listings}
% \lstset{
%    language=Python,
%    basicstyle=\small\ttfamily,
%    commentstyle=\ttfamily\color{blue},
%    stringstyle=\ttfamily\color{orange},
%    showstringspaces=false,
%    breaklines=true,
%    postbreak = \space\dots
% }

%%%%%%%%%%%%%%%%%%%%%%%%%%%%%%%%%%%%%%%%%%%%%%%%%%%%%%%%%%%%%%%%%%%%%%
% Title page
\title[Basic Plotting]{Python for Scientific Computing : Basic Plotting}

\author[FOSSEE] {FOSSEE}

\institute[IIT Bombay] {Department of Aerospace Engineering\\IIT Bombay}
\date{}

% DOCUMENT STARTS
\begin{document}

\begin{frame}
  \maketitle
\end{frame}

\begin{frame}
  \frametitle{About the Session}
  \begin{block}{Intended Audience}
  \begin{itemize}
       \item Engg., Mathematics and Science teachers.
       \item Interested students from similar streams.
  \end{itemize}
  \end{block}  

  \begin{block}{Goal: Successful participants will be able to}
    \begin{itemize}
      \item Use Python as a basic Plotting tool.
    \end{itemize}
  \end{block}
\end{frame}

\begin{frame}
\frametitle{Checklist}
   \begin{itemize}
    \item IPython
    \item Pylab
  \end{itemize}
\end{frame}

%% \begin{frame}[fragile]
%% \frametitle{Starting up \ldots}
%% \begin{block}{}
%% \begin{verbatim}
%%   $ ipython -pylab  
%% \end{verbatim}
%% \end{block}
%% \begin{lstlisting}     
%%   In []: print "Hello, World!"
%%   Hello, World!
%% \end{lstlisting}
%% Exiting
%% \begin{lstlisting}     
%%   In []: ^D(Ctrl-D)
%%   Do you really want to exit([y]/n)? y
%% \end{lstlisting}
%% \end{frame}

\begin{frame}[fragile]
  \frametitle{Summary}
  \begin{block}{IPython}
    \begin{itemize}
    \item Starting and Quiting.
    \item AutoCompletion
    \item Help
    \end{itemize}
  \end{block}
  \begin{block}{Plotting}
    \begin{itemize}    
    \item Creating simple plots.
    \item Adding labels and legends.
    \item Annotating plots.
    \item Changing the looks: size, linewidth, colors
    \end{itemize}  
  \end{block}
\end{frame}

\begin{frame}
  \frametitle{Thank you!}  
  \begin{block}{}
    This is first tutorial from the series of
    \begin{center}      
      \textcolor{blue}{'Python for Scientific Computing'}.
    \end{center}
  \end{block}
  \begin{block}{}
  It is part of \textcolor{blue}{FOSSEE} project funded by:
  \begin{center}
    \textcolor{blue}{NME through ICT from MHRD, Govt. of India}.
  \end{center}  
  \end{block}
\end{frame}

\end{document}

