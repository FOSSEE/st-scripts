%%%%%%%%%%%%%%%%%%%%%%%%%%%%%%%%%%%%%%%%%%%%%%%%%%%%%%%%%%%%%%%%%%%%%%%%%%%%%%%%
%Tutorial slides on Python.
%
% Author: FOSSEE 
% Copyright (c) 2009, FOSSEE, IIT Bombay
%%%%%%%%%%%%%%%%%%%%%%%%%%%%%%%%%%%%%%%%%%%%%%%%%%%%%%%%%%%%%%%%%%%%%%%%%%%%%%%%

\documentclass[14pt,compress]{beamer}
%\documentclass[draft]{beamer}
%\documentclass[compress,handout]{beamer}
%\usepackage{pgfpages} 
%\pgfpagesuselayout{2 on 1}[a4paper,border shrink=5mm]

% Modified from: generic-ornate-15min-45min.de.tex
\mode<presentation>
{
  \usetheme{Warsaw}
  \useoutertheme{infolines}
  \setbeamercovered{transparent}
}

\usepackage[english]{babel}
\usepackage[latin1]{inputenc}
%\usepackage{times}
\usepackage[T1]{fontenc}

% Taken from Fernando's slides.
\usepackage{ae,aecompl}
\usepackage{mathpazo,courier,euler}
\usepackage[scaled=.95]{helvet}

\definecolor{darkgreen}{rgb}{0,0.5,0}

\usepackage{listings}
\lstset{language=Python,
    basicstyle=\ttfamily\bfseries,
    commentstyle=\color{red}\itshape,
  stringstyle=\color{darkgreen},
  showstringspaces=false,
  keywordstyle=\color{blue}\bfseries}

%%%%%%%%%%%%%%%%%%%%%%%%%%%%%%%%%%%%%%%%%%%%%%%%%%%%%%%%%%%%%%%%%%%%%%
% Macros
\setbeamercolor{emphbar}{bg=blue!20, fg=black}
\newcommand{\emphbar}[1]
{\begin{beamercolorbox}[rounded=true]{emphbar} 
      {#1}
 \end{beamercolorbox}
}
\newcounter{time}
\setcounter{time}{0}
\newcommand{\inctime}[1]{\addtocounter{time}{#1}{\tiny \thetime\ m}}

\newcommand{\typ}[1]{\lstinline{#1}}

\newcommand{\kwrd}[1]{ \texttt{\textbf{\color{blue}{#1}}}  }

% Title page
\title{Python for Scientific Computing : Least Square Fit}

\author[FOSSEE] {FOSSEE}

\institute[IIT Bombay] {Department of Aerospace Engineering\\IIT Bombay}
\date{}

% DOCUMENT STARTS
\begin{document}

\begin{frame}
  \maketitle
\end{frame}

\begin{frame}
  \frametitle{About the Session}
  \begin{block}{Goal}
Finding least square fit of given data-set
  \end{block}
  \begin{block}{Checklist}
    \begin{itemize}
    \item pendulum.txt
  \end{itemize}
  \end{block}
\end{frame}

\begin{frame}[fragile]
\frametitle{Solution of equations}
Consider,
  \begin{align*}
    3x + 2y - z  & = 1 \\
    2x - 2y + 4z  & = -2 \\
    -x + \frac{1}{2}y -z & = 0
  \end{align*}
Solution:
  \begin{align*}
    x & = 1 \\
    y & = -2 \\
    z & = -2
  \end{align*}
\end{frame}

\begin{frame}[fragile]
\frametitle{Scipy Methods - \typ{roots}}
\begin{itemize}
\item Calculates the roots of polynomials
\item To calculate the roots of $x^2-5x+6$ 
\end{itemize}
\begin{lstlisting}
  In []: coeffs = [1, -5, 6]
  In []: roots(coeffs)
  Out[]: array([3., 2.])
\end{lstlisting}
\vspace*{-.2in}
\begin{center}
\includegraphics[height=1.6in, interpolate=true]{data/roots}    
\end{center}
\end{frame}

\begin{frame}[fragile]
\frametitle{Scipy Methods - \typ{fsolve}}
\begin{small}
\begin{lstlisting}
  In []: from scipy.optimize import fsolve
\end{lstlisting}
\end{small}
\begin{itemize}
\item Finds the roots of a system of non-linear equations
\item Input arguments - Function and initial estimate
\item Returns the solution
\end{itemize}
\end{frame}

\begin{frame}[fragile]
  \frametitle{Summary}
  \begin{block}{}
    \begin{itemize}
    \item Solving linear equations
    \item Finding roots
    \item Roots of non linear equations
    \item Basics of Functions
    \end{itemize}
  \end{block}    
\end{frame}

\begin{frame}
  \frametitle{Thank you!}  
  \begin{block}{}
  This session is part of \textcolor{blue}{FOSSEE} project funded by:
  \begin{center}
    \textcolor{blue}{NME through ICT from MHRD, Govt. of India}.
  \end{center}  
  \end{block}
\end{frame}

\end{document}
