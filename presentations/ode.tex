%%%%%%%%%%%%%%%%%%%%%%%%%%%%%%%%%%%%%%%%%%%%%%%%%%%%%%%%%%%%%%%%%%%%%%%%%%%%%%%%
%Tutorial slides on Python.
%
% Author: FOSSEE 
% Copyright (c) 2009, FOSSEE, IIT Bombay
%%%%%%%%%%%%%%%%%%%%%%%%%%%%%%%%%%%%%%%%%%%%%%%%%%%%%%%%%%%%%%%%%%%%%%%%%%%%%%%%

\documentclass[14pt,compress]{beamer}
%\documentclass[draft]{beamer}
%\documentclass[compress,handout]{beamer}
%\usepackage{pgfpages} 
%\pgfpagesuselayout{2 on 1}[a4paper,border shrink=5mm]

% Modified from: generic-ornate-15min-45min.de.tex
\mode<presentation>
{
  \usetheme{Warsaw}
  \useoutertheme{infolines}
  \setbeamercovered{transparent}
}

\usepackage[english]{babel}
\usepackage[latin1]{inputenc}
%\usepackage{times}
\usepackage[T1]{fontenc}

% Taken from Fernando's slides.
\usepackage{ae,aecompl}
\usepackage{mathpazo,courier,euler}
\usepackage[scaled=.95]{helvet}

\definecolor{darkgreen}{rgb}{0,0.5,0}

\usepackage{listings}
\lstset{language=Python,
    basicstyle=\ttfamily\bfseries,
    commentstyle=\color{red}\itshape,
  stringstyle=\color{darkgreen},
  showstringspaces=false,
  keywordstyle=\color{blue}\bfseries}

%%%%%%%%%%%%%%%%%%%%%%%%%%%%%%%%%%%%%%%%%%%%%%%%%%%%%%%%%%%%%%%%%%%%%%
% Macros
\setbeamercolor{emphbar}{bg=blue!20, fg=black}
\newcommand{\emphbar}[1]
{\begin{beamercolorbox}[rounded=true]{emphbar} 
      {#1}
 \end{beamercolorbox}
}
\newcounter{time}
\setcounter{time}{0}
\newcommand{\inctime}[1]{\addtocounter{time}{#1}{\tiny \thetime\ m}}

\newcommand{\typ}[1]{\lstinline{#1}}

\newcommand{\kwrd}[1]{ \texttt{\textbf{\color{blue}{#1}}}  }

% Title page
\title{Python for Scientific Computing: Ordinary Differential Equation}

\author[FOSSEE] {FOSSEE}

\institute[IIT Bombay] {Department of Aerospace Engineering\\IIT Bombay}
\date{}

% DOCUMENT STARTS
\begin{document}

\begin{frame}
  \maketitle
\end{frame}

\begin{frame}
  \frametitle{About the Session}
  \begin{block}{Goal}
Solving ordinary differential equations.
  \end{block}
  \begin{block}{Prerequisite}
    \begin{itemize}
    \item Understanding of Arrays.
    \item functions and lists
    \end{itemize}    
  \end{block}
\end{frame}

\begin{frame}[fragile]
\frametitle{Solving ODEs using SciPy}
\begin{itemize}
\item Let's consider the spread of an epidemic in a population
\item $\frac{dy}{dt} = ky(L-y)$ gives the spread of the disease
\item L is the total population.
\item Use L = 250000, k = 0.00003, y(0) = 250
\end{itemize}
\end{frame}

\begin{frame}[fragile]
\frametitle{ODEs - Simple Pendulum}
We shall use the simple ODE of a simple pendulum. 
\begin{equation*}
\ddot{\theta} = -\frac{g}{L}sin(\theta)
\end{equation*}
\begin{itemize}
\item This equation can be written as a system of two first order ODEs
\end{itemize}
\begin{align}
\dot{\theta} &= \omega \\
\dot{\omega} &= -\frac{g}{L}sin(\theta) \\
 \text{At}\ t &= 0 : \nonumber \\
 \theta = \theta_0(10^o)\quad & \&\quad  \omega = 0\ (Initial\ values)\nonumber 
\end{align}
\end{frame}

\begin{frame}[fragile]
  \frametitle{Summary}
  \begin{block}{}
    Solving ordinary differential equations
  \end{block}
\end{frame}

\begin{frame}
  \frametitle{Thank you!}  
  \begin{block}{}
  This session is part of \textcolor{blue}{FOSSEE} project funded by:
  \begin{center}
    \textcolor{blue}{NME through ICT from MHRD, Govt. of India}.
  \end{center}  
  \end{block}
\end{frame}

\end{document}
