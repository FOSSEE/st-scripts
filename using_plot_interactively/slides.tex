% Created 2011-05-04 Wed 11:44
\documentclass[presentation]{beamer}
\usepackage[latin1]{inputenc}
\usepackage[T1]{fontenc}
\usepackage{fixltx2e}
\usepackage{graphicx}
\usepackage{longtable}
\usepackage{float}
\usepackage{wrapfig}
\usepackage{soul}
\usepackage{textcomp}
\usepackage{marvosym}
\usepackage{wasysym}
\usepackage{latexsym}
\usepackage{amssymb}
\usepackage{hyperref}
\tolerance=1000
\usepackage[english]{babel} \usepackage{ae,aecompl}
\usepackage{mathpazo,courier,euler} \usepackage[scaled=.95]{helvet}
\usepackage{listings}
\lstset{language=Python, basicstyle=\ttfamily\bfseries,
commentstyle=\color{red}\itshape, stringstyle=\color{darkgreen},
showstringspaces=false, keywordstyle=\color{blue}\bfseries}
\providecommand{\alert}[1]{\textbf{#1}}

\title{}
\author{FOSSEE}
\date{}

\usetheme{Warsaw}\usecolortheme{default}\useoutertheme{infolines}\setbeamercovered{transparent}
\begin{document}











\begin{frame}

\begin{center}
\textcolor{blue}{Using plot Interactively}
\end{center}
\begin{center}
\includegraphics[scale=0.25]{../images/iitb-logo.png}\\
Developed by FOSSEE Team, IIT-Bombay. \\ 
Funded by National Mission on Education through ICT

MHRD, Govt. of India
\end{center}
\end{frame}
\begin{frame}
\frametitle{Objectives}
\label{sec-2}

  At the end of this tutorial, you will be able to, 

\begin{itemize}
\item Create simple plots of mathematical functions.
\item Use the Figure window to study plots better.
\end{itemize}
\end{frame}
\begin{frame}
\frametitle{Error if Ipython not installed}
\label{sec-3}
\begin{itemize}

\item `ERROR: matplotlib could NOT be imported!  Starting normal IPython.`\\
\label{sec-3_1}%
\end{itemize} % ends low level
\end{frame}
\begin{frame}
\frametitle{Plot UI}
\label{sec-4}

   \includegraphics[height=0.12in, interpolate=true]{buttons}

\begin{itemize}
\item Save
\item Zoom
\item Move axis
\item Back and Forward Button
\item Home
\end{itemize}
\end{frame}
\begin{frame}
\frametitle{Question1}
\label{sec-5}

  Plot (sin(x)*sin(x))/x.

\begin{enumerate}
\item Save the plot by the sinsquarebyx.pdf in pdf format.
\item Zoom and find the maxima.
\item Bring it back to initial position.
\end{enumerate}
\end{frame}
\begin{frame}
\frametitle{Summary}
\label{sec-6}

  In this tutorial,we have learnt to-

\begin{itemize}
\item Start Ipython with pylab.
\item Use the linspace function to create `num` equally spaced points in a region.
\item Find the length of sequnces using len function.
\item Plot mathematical functions using plot.
\item Clear drawing area using clf.
\item Plott mathematical functions using plot.
\item Use the UI of plot
\begin{itemize}
\item Save
\item Zoom
\item Move axis
\item Back and Forward Button
\item Home
\end{itemize}
\end{itemize}
 
\end{frame}
\begin{frame}
\frametitle{Evaluation}
\label{sec-7}


\begin{enumerate}
\item Create 100 equally spaced points between -pi/2 and pi/2?
\item How do you clear a figure in ipython?
\item How do find the length of a sequen
\end{enumerate}
\end{frame}
\begin{frame}
\frametitle{Solutions\ldots{}}
\label{sec-8}


\begin{enumerate}
\item linspace(-pi/2,pi/2,100)
\item clf()
\item len(sequence\_name)
\end{enumerate}
\end{frame}
\begin{frame}
\frametitle{Acknowledgement\ldots{}}
\label{sec-9}

 \begin{block}{}
  \begin{center}
  \textcolor{blue}{\Large THANK YOU!} 
  \end{center}
  \end{block}
\begin{block}{}
  \begin{center}
    For more Information, visit our website\\
    \url{http://fossee.in/}
  \end{center}  
  \end{block}
\end{frame}

\end{document}