% Created 2010-11-07 Sun 16:18
\documentclass[presentation]{beamer}
\usepackage[latin1]{inputenc}
\usepackage[T1]{fontenc}
\usepackage{fixltx2e}
\usepackage{graphicx}
\usepackage{longtable}
\usepackage{float}
\usepackage{wrapfig}
\usepackage{soul}
\usepackage{t1enc}
\usepackage{textcomp}
\usepackage{marvosym}
\usepackage{wasysym}
\usepackage{latexsym}
\usepackage{amssymb}
\usepackage{hyperref}
\tolerance=1000
\usepackage[english]{babel} \usepackage{ae,aecompl}
\usepackage{mathpazo,courier,euler} \usepackage[scaled=.95]{helvet}
\usepackage{listings}
\lstset{language=Python, basicstyle=\ttfamily\bfseries,
commentstyle=\color{red}\itshape, stringstyle=\color{darkgreen},
showstringspaces=false, keywordstyle=\color{blue}\bfseries}
\providecommand{\alert}[1]{\textbf{#1}}

\title{Matrices}
\author{FOSSEE}
\date{}

\usetheme{Warsaw}\usecolortheme{default}\useoutertheme{infolines}\setbeamercovered{transparent}
\begin{document}

\maketitle









\begin{frame}
\frametitle{Outline}
\label{sec-1}

\begin{itemize}
\item Creating Matrices

\begin{itemize}
\item using direct data
\item converting a list
\end{itemize}

\item Matrix operations
\item Inverse of matrix
\item Determinant of matrix
\item Eigen values and Eigen vectors of matrices
\item Norm of matrix
\item Singular Value Decomposition of matrices
\end{itemize}
\end{frame}
\begin{frame}[fragile]
\frametitle{Creating a matrix}
\label{sec-2}

\begin{itemize}
\item Creating a matrix using direct data
\end{itemize}

\begin{verbatim}
   In []: m1 = array([1, 2, 3, 4])
\end{verbatim}

\begin{itemize}
\item Creating a matrix using lists
\end{itemize}

\begin{verbatim}
   In []: l1 = [[1,2,3,4],[5,6,7,8]]
   In []: m2 = array(l1)
\end{verbatim}
\end{frame}
\begin{frame}[fragile]
\frametitle{Exercise 1}
\label{sec-3}

  Create a (2, 4) matrix \texttt{m3}
\begin{verbatim}
   m3 = [[5,  6,  7,  8],
         [9, 10, 11, 12]]
\end{verbatim}
\end{frame}
\begin{frame}[fragile]
\frametitle{Solution 1}
\label{sec-4}

\begin{itemize}
\item m3 can be created as,
\end{itemize}

\begin{verbatim}
   In []: m3 = array([[5,6,7,8],[9,10,11,12]])
\end{verbatim}
\end{frame}
\begin{frame}[fragile]
\frametitle{Matrix operations}
\label{sec-5}

\begin{itemize}
\item Element-wise addition (both matrix should be of order \texttt{mXn})
\begin{verbatim}
     In []: m3 + m2
\end{verbatim}

\item Element-wise subtraction (both matrix should be of order \texttt{mXn})
\begin{verbatim}
     In []: m3 - m2
\end{verbatim}

\end{itemize}
\end{frame}
\begin{frame}[fragile]
\frametitle{Matrix Multiplication}
\label{sec-6}

\begin{itemize}
\item Element-wise multiplication using \texttt{m3 * m2}
\begin{verbatim}
     In []: m3 * m2
\end{verbatim}

\item Matrix Multiplication using \texttt{dot(m3, m2)}
\begin{verbatim}
     In []: dot(m3, m2)
     Out []: ValueError: objects are not aligned
\end{verbatim}

\end{itemize}
\end{frame}
\begin{frame}[fragile]
\frametitle{Matrix Multiplication (cont'd)}
\label{sec-7}

\begin{itemize}
\item Create two compatible matrices of order \texttt{nXm} and \texttt{mXr}
\begin{verbatim}
     In []: m1.shape
\end{verbatim}


\begin{itemize}
\item matrix m1 is of order \texttt{1 X 4}
\end{itemize}

\item Creating another matrix of order \texttt{4 X 2}
\begin{verbatim}
     In []: m4 = array([[1,2],[3,4],[5,6],[7,8]])
\end{verbatim}

\item Matrix multiplication
\begin{verbatim}
     In []: dot(m1, m4)
\end{verbatim}

\end{itemize}
\end{frame}
\begin{frame}
\frametitle{Recall from \texttt{array}}
\label{sec-8}

\begin{itemize}
\item The functions

\begin{itemize}
\item \texttt{identity(n)} - 
      creates an identity matrix of order \texttt{nXn}
\item \texttt{zeros((m,n))} - 
      creates a matrix of order \texttt{mXn} with 0's
\item \texttt{zeros\_like(A)} - 
      creates a matrix with 0's similar to the shape of matrix \texttt{A}
\item \texttt{ones((m,n))}
      creates a matrix of order \texttt{mXn} with 1's
\item \texttt{ones\_like(A)}
      creates a matrix with 1's similar to the shape of matrix \texttt{A}
\end{itemize}

\end{itemize}

  Can also be used with matrices
\end{frame}
\begin{frame}[fragile]
\frametitle{More matrix operations}
\label{sec-9}

  Transpose of a matrix
\begin{verbatim}
   In []: m4.T
\end{verbatim}
\end{frame}
\begin{frame}[fragile]
\frametitle{Exercise 2 : Frobenius norm \& inverse}
\label{sec-10}

  Find out the Frobenius norm of inverse of a \texttt{4 X 4} matrix.
\begin{verbatim}
   
\end{verbatim}

  The matrix is
\begin{verbatim}
   m5 = arange(1,17).reshape(4,4)
\end{verbatim}

\begin{itemize}
\item Inverse of A,

\begin{itemize}
\item $A^{-1} = inv(A)$
\end{itemize}

\item Frobenius norm is defined as,

\begin{itemize}
\item $||A||_F = [\sum_{i,j} abs(a_{i,j})^2]^{1/2}$
\end{itemize}

\end{itemize}
\end{frame}
\begin{frame}[fragile]
\frametitle{Exercise 3 : Infinity norm}
\label{sec-11}

  Find the infinity norm of the matrix \texttt{im5}
\begin{verbatim}
   
\end{verbatim}

\begin{itemize}
\item Infinity norm is defined as,
       $max([\sum_{i} abs(a_{i})^2])$
\end{itemize}
\end{frame}
\begin{frame}[fragile]
\frametitle{\texttt{norm()} method}
\label{sec-12}

\begin{itemize}
\item Frobenius norm
\begin{verbatim}
     In []: norm(im5)
\end{verbatim}

\item Infinity norm
\begin{verbatim}
     In []: norm(im5, ord=inf)
\end{verbatim}

\end{itemize}
\end{frame}
\begin{frame}[fragile]
\frametitle{Determinant}
\label{sec-13}

  Find out the determinant of the matrix m5
\begin{verbatim}
   
\end{verbatim}

\begin{itemize}
\item determinant can be found out using

\begin{itemize}
\item \texttt{det(A)} - returns the determinant of matrix \texttt{A}
\end{itemize}

\end{itemize}
\end{frame}
\begin{frame}[fragile]
\frametitle{eigen values \& eigen vectors}
\label{sec-14}

  Find out the eigen values and eigen vectors of the matrix \texttt{m5}.
\begin{verbatim}
   
\end{verbatim}

\begin{itemize}
\item eigen values and vectors can be found out using
\begin{verbatim}
     In []: eig(m5)
\end{verbatim}

    returns a tuple of \emph{eigen values} and \emph{eigen vectors}
\item \emph{eigen values} in tuple

\begin{itemize}
\item \texttt{In []: eig(m5)[0]}
\end{itemize}

\item \emph{eigen vectors} in tuple

\begin{itemize}
\item \texttt{In []: eig(m5)[1]}
\end{itemize}

\item Computing \emph{eigen values} using \texttt{eigvals()}
\begin{verbatim}
     In []: eigvals(m5)
\end{verbatim}

\end{itemize}
\end{frame}
\begin{frame}[fragile]
\frametitle{Singular Value Decomposition (\texttt{svd})}
\label{sec-15}

    $M = U \Sigma V^*$
\begin{itemize}
\item U, an \texttt{mXm} unitary matrix over K.
\item $\Sigma$
        , an \texttt{mXn} diagonal matrix with non-negative real numbers on diagonal.
\item $V^*$
        , an \texttt{nXn} unitary matrix over K, denotes the conjugate transpose of V.
\item SVD of matrix \texttt{m5} can be found out as,
\end{itemize}

\begin{verbatim}
     In []: svd(m5)
\end{verbatim}
\end{frame}
\begin{frame}
\frametitle{Summary}
\label{sec-16}

\begin{itemize}
\item Matrices

\begin{itemize}
\item creating matrices
\end{itemize}

\item Matrix operations
\item Inverse (\texttt{inv()})
\item Determinant (\texttt{det()})
\item Norm (\texttt{norm()})
\item Eigen values \& vectors (\texttt{eig(), eigvals()})
\item Singular Value Decomposition (\texttt{svd()})
\end{itemize}
\end{frame}
\begin{frame}
\frametitle{Thank you!}
\label{sec-17}

  \begin{block}{}
  \begin{center}
  This spoken tutorial has been produced by the
  \textcolor{blue}{FOSSEE} team, which is funded by the 
  \end{center}
  \begin{center}
    \textcolor{blue}{National Mission on Education through \\
      Information \& Communication Technology \\ 
      MHRD, Govt. of India}.
  \end{center}  
  \end{block}

  
\end{frame}

\end{document}
