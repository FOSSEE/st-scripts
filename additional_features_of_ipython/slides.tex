% Created 2011-05-19 Thu 13:48
\documentclass[presentation]{beamer}
\usepackage[utf8]{inputenc}
\usepackage[T1]{fontenc}
\usepackage{fixltx2e}
\usepackage{graphicx}
\usepackage{longtable}
\usepackage{float}
\usepackage{wrapfig}
\usepackage{soul}
\usepackage{textcomp}
\usepackage{marvosym}
\usepackage{wasysym}
\usepackage{latexsym}
\usepackage{amssymb}
\usepackage{hyperref}
\tolerance=1000
\usepackage[english]{babel} \usepackage{ae,aecompl}
\usepackage{mathpazo,courier,euler} \usepackage[scaled=.95]{helvet}
\usepackage{listings}
\lstset{language=Python, basicstyle=\ttfamily\bfseries,
commentstyle=\color{red}\itshape, stringstyle=\color{darkgreen},
showstringspaces=false, keywordstyle=\color{blue}\bfseries}
\providecommand{\alert}[1]{\textbf{#1}}

\title{}
\author{FOSSEE}
\date{}

\usetheme{Warsaw}\usecolortheme{default}\useoutertheme{infolines}\setbeamercovered{transparent}
\begin{document}











\begin{frame}

\begin{center}
\vspace{12pt}
\textcolor{blue}{\huge Additional features of \texttt{ipython}}
\end{center}
\vspace{18pt}
\begin{center}
\vspace{10pt}
\includegraphics[scale=0.95]{../images/fossee-logo.png}\\
\vspace{5pt}
\scriptsize Developed by FOSSEE Team, IIT-Bombay. \\ 
\scriptsize Funded by National Mission on Education through ICT\\
\scriptsize  MHRD,Govt. of India\\
\includegraphics[scale=0.30]{../images/iitb-logo.png}\\
\end{center}
\end{frame}
\begin{frame}
\frametitle{Objectives}
\label{sec-2}

  At the end of this tutorial, you will be able to,
 

\begin{itemize}
\item Retrieve your ipython history.
\item View a part of the history.
\item Save a part of your history to a file.
\item Run a script from within ipython.
\end{itemize}
\end{frame}
\begin{frame}
\frametitle{Pre-requisite}
\label{sec-3}

  Spoken tuorial on -

\begin{itemize}
\item Embellishing a Plot
\end{itemize}
\end{frame}
\begin{frame}
\frametitle{Exercise 1}
\label{sec-4}

  Read through the documentation of \texttt{\%hist} and find out how to
  list all the commands between 5 and 10.
\end{frame}
\begin{frame}
\frametitle{Exercise 2}
\label{sec-5}

  Change the label on y-axis to ``y'' and save the lines of code
  accordingly
\end{frame}
\begin{frame}
\frametitle{Exercise 3}
\label{sec-6}

  Use \texttt{\%hist} and \texttt{\%save} and create a script that has the function show()
  in it.Run the script to produce the plot and display the same.
\end{frame}
\begin{frame}
\frametitle{Exercise 4}
\label{sec-7}

  Run the script without using the -i option. Do you find any
  difference?
\end{frame}
\begin{frame}
\frametitle{Solution 4}
\label{sec-8}

  We see that it raises \verb~NameError~ saying the name \verb~linspace~ is not
  found.
\end{frame}
\begin{frame}
\frametitle{Summary}
\label{sec-9}

  In this tutorial, we have learnt to –

\begin{itemize}
\item Retrieve the history using \texttt{\%hist} command.
\item View only a part of history by passing an argument to \%hist.
\item Save the required lines of code in required order using \%save command.
\item Use \%run -i command to run the saved script.
\end{itemize}
\end{frame}
\begin{frame}
\frametitle{Evaluation}
\label{sec-10}


\begin{enumerate}
\item How do you retrieve the recent 5 commands
\begin{itemize}
\item ``\%hist``
\item ``\%hist -5``
\item ``\%hist 5``
\item ``\%hist 5-10``
\end{itemize}
\item How do you save the lines 2 3 4 5 7 9 10 11
\begin{itemize}
\item ``\%save filepath 2-5 7 9-11``
\item ``\%save filepath 2-11``
\item ``\%save filepath``
\item ``\%save 2-5 7 9 10 11``
\end{itemize}
\item What will the command ``\%hist 5 10`` display.
\begin{itemize}
\item The recently typed commands from 5 to 10 inclusive of 
      the history command
\item The recently typed commands from 5 to 10 excluding 
      the history command
\end{itemize}
\end{enumerate}
\end{frame}
\begin{frame}
\frametitle{Solutions}
\label{sec-11}


\begin{enumerate}
\item \%hist 5
\item \%save filepath 2-5 7 9-11
\item \%hist 5 10
\end{enumerate}
\end{frame}
\begin{frame}

  \begin{block}{}
  \begin{center}
  \textcolor{blue}{\Large THANK YOU!} 
  \end{center}
  \end{block}
\begin{block}{}
  \begin{center}
    For more Information, visit our website\\
    \url{http://fossee.in/}
  \end{center}  
  \end{block}
\end{frame}

\end{document}