% Created 2010-11-09 Tue 16:07
\documentclass[presentation]{beamer}
\usepackage[latin1]{inputenc}
\usepackage[T1]{fontenc}
\usepackage{fixltx2e}
\usepackage{graphicx}
\usepackage{longtable}
\usepackage{float}
\usepackage{wrapfig}
\usepackage{soul}
\usepackage{textcomp}
\usepackage{marvosym}
\usepackage{wasysym}
\usepackage{latexsym}
\usepackage{amssymb}
\usepackage{hyperref}
\tolerance=1000
\usepackage[english]{babel} \usepackage{ae,aecompl}
\usepackage{mathpazo,courier,euler} \usepackage[scaled=.95]{helvet}
\usepackage{listings}
\lstset{language=Python, basicstyle=\ttfamily\bfseries,
commentstyle=\color{red}\itshape, stringstyle=\color{darkgreen},
showstringspaces=false, keywordstyle=\color{blue}\bfseries}
\providecommand{\alert}[1]{\textbf{#1}}

\title{Manipulating Lists}
\author{FOSSEE}
\date{}

\usetheme{Warsaw}\usecolortheme{default}\useoutertheme{infolines}\setbeamercovered{transparent}
\begin{document}

\maketitle









\begin{frame}
\frametitle{Outline}
\label{sec-1}

  In this session we shall be looking at 
\begin{itemize}
\item Concatenating lists
\item Obtaining parts of lists
\item Sorting lists
\item Reversing lists
\end{itemize}
\end{frame}
\begin{frame}
\frametitle{Question 1}
\label{sec-2}

  Obtain the primes less than 10, from the list \texttt{primes}. 
\end{frame}
\begin{frame}[fragile]
\frametitle{Solution 1}
\label{sec-3}

\lstset{language=Python}
\begin{lstlisting}
primes[0:4]
\end{lstlisting}
\end{frame}
\begin{frame}[fragile]
\frametitle{Slicing}
\label{sec-4}

\lstset{language=Python}
\begin{lstlisting}
p[start:stop]
\end{lstlisting}
\begin{itemize}
\item Returns all elements of \texttt{p} between \texttt{start} and \texttt{stop}
\item The element with index equal to \texttt{stop} is \textbf{not} included.
\end{itemize}
\end{frame}
\begin{frame}
\frametitle{Question 2}
\label{sec-5}

  Obtain all the multiples of three from the list \texttt{num}.
\end{frame}
\begin{frame}[fragile]
\frametitle{Solution 2}
\label{sec-6}

\lstset{language=Python}
\begin{lstlisting}
num[::3]
\end{lstlisting}
\end{frame}
\begin{frame}[fragile]
\frametitle{Question 3}
\label{sec-7}

  Given a list of marks of students in an examination, obtain a list
  with marks in descending order.
\lstset{language=Python}
\begin{lstlisting}
marks = [99, 67, 47, 100, 50, 75, 62]
\end{lstlisting}
\end{frame}
\begin{frame}[fragile]
\frametitle{Solution 3}
\label{sec-8}

\lstset{language=Python}
\begin{lstlisting}
sorted(marks)[::-1]
\end{lstlisting}
OR
\lstset{language=Python}
\begin{lstlisting}
sorted(marks, reverse=True)
\end{lstlisting}
\end{frame}
\begin{frame}
\frametitle{Summary}
\label{sec-9}

  In this tutorial session we learnt
\begin{itemize}
\item Obtaining parts of lists using slicing and striding
\item List concatenation
\item Sorting lists
\item Reversing lists
\end{itemize}
\end{frame}
\begin{frame}
\frametitle{Thank you!}
\label{sec-10}

  \begin{block}{}
  \begin{center}
  This spoken tutorial has been produced by the
  \textcolor{blue}{FOSSEE} team, which is funded by the 
  \end{center}
  \begin{center}
    \textcolor{blue}{National Mission on Education through \\
      Information \& Communication Technology \\ 
      MHRD, Govt. of India}.
  \end{center}  
  \end{block}
\end{frame}

\end{document}
