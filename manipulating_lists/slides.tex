% Created 2011-06-29 Wed 13:50
\documentclass[presentation]{beamer}
\usepackage[latin1]{inputenc}
\usepackage[T1]{fontenc}
\usepackage{fixltx2e}
\usepackage{graphicx}
\usepackage{longtable}
\usepackage{float}
\usepackage{wrapfig}
\usepackage{soul}
\usepackage{textcomp}
\usepackage{marvosym}
\usepackage{wasysym}
\usepackage{latexsym}
\usepackage{amssymb}
\usepackage{hyperref}
\tolerance=1000
\usepackage[english]{babel} \usepackage{ae,aecompl}
\usepackage{mathpazo,courier,euler} \usepackage[scaled=.95]{helvet}
\usepackage{listings}
\lstset{language=Python, basicstyle=\ttfamily\bfseries,
commentstyle=\color{red}\itshape, stringstyle=\color{darkgreen},
showstringspaces=false, keywordstyle=\color{blue}\bfseries}
\providecommand{\alert}[1]{\textbf{#1}}

\title{}
\author{FOSSEE}
\date{}

\usetheme{Warsaw}\usecolortheme{default}\useoutertheme{infolines}\setbeamercovered{transparent}
\begin{document}











\begin{frame}

\begin{center}
\vspace{12pt}
\textcolor{blue}{\huge Manipulating Lists}
\end{center}
\vspace{18pt}
\begin{center}
\vspace{10pt}
\includegraphics[scale=0.95]{../images/fossee-logo.png}\\
\vspace{5pt}
\scriptsize Developed by FOSSEE Team, IIT-Bombay. \\ 
\scriptsize Funded by National Mission on Education through ICT\\
\scriptsize  MHRD,Govt. of India\\
\includegraphics[scale=0.30]{../images/iitb-logo.png}\\
\end{center}
\end{frame}
\begin{frame}
\frametitle{Objectives}
\label{sec-2}

  At the end of this tutorial, you will be able to,


\begin{itemize}
\item Concatenate two lists.
\item Learn the details of slicing and striding of lists.
\item Sort and reverse lists.
\end{itemize}
\end{frame}
\begin{frame}
\frametitle{Pre-requisite}
\label{sec-3}

Spoken tutorial on -

\begin{itemize}
\item Getting started with Lists.
\end{itemize}
\end{frame}
\begin{frame}
\frametitle{Exercise 1}
\label{sec-4}

  Obtain the primes less than 10, from the list  \verb~primes~. 
\end{frame}
\begin{frame}[fragile]
\frametitle{Slicing}
\label{sec-5}

\lstset{language=Python}
\begin{lstlisting}
p[start:stop]
\end{lstlisting}

\begin{itemize}
\item Returns all elements of \verb~p~ between \verb~start~ and \verb~stop~
\item The element with index equal to \verb~stop~ is \textbf{not} included.
\end{itemize}
\end{frame}
\begin{frame}
\frametitle{Exercise 2}
\label{sec-6}

  Obtain all the multiples of three from the list \verb~num~.
\end{frame}
\begin{frame}[fragile]
\frametitle{Solution 2}
\label{sec-7}

\lstset{language=Python}
\begin{lstlisting}
num[::3]
\end{lstlisting}
\end{frame}
\begin{frame}[fragile]
\frametitle{Exercise 3}
\label{sec-8}

  Given a list of marks of students in an examination, obtain a list
  with marks in descending order.
\lstset{language=Python}
\begin{lstlisting}
marks = [99, 67, 47, 100, 50, 75, 62]
\end{lstlisting}
\end{frame}
\begin{frame}
\frametitle{Summary}
\label{sec-9}

  In this tutorial, we have learnt to,


\begin{itemize}
\item Obtain parts of lists using slicing and striding.
\item Concatenate lists using the ``plus'' operator.
\item Sort lists using the ``sort'' method.
\item Use the method ``reverse'' to reverse the lists.
\end{itemize}
\end{frame}
\begin{frame}
\frametitle{Evaluation}
\label{sec-10}


\begin{enumerate}
\item Given the list primes,\\ primes = [2, 3, 5, 7, 11, 13, 17, 19, 23,
   29],\\ How do you obtain the last 4 primes?
\vspace{11pt}   
\item Given a list, p, of unknown length, obtain the first 3 (or all, if
   there are fewer) characters of it.
\vspace{11pt}   
\item ``reversed'' function reverses a list in place. True or False?
\end{enumerate}
\end{frame}
\begin{frame}
\frametitle{Solutions}
\label{sec-11}


\begin{enumerate}
\item primes[-4:]
\vspace{8pt}
\item p[:3]
\vspace{8pt}
\item False
\end{enumerate}
\end{frame}
\begin{frame}

  \begin{block}{}
  \begin{center}
  \textcolor{blue}{\Large THANK YOU!} 
  \end{center}
  \end{block}
\begin{block}{}
  \begin{center}
    For more Information, visit our website\\
    \url{http://fossee.in/}
  \end{center}  
  \end{block}
\end{frame}

\end{document}