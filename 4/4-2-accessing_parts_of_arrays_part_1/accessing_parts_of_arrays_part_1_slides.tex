\documentclass[17pt]{beamer}
\usepackage{amsmath}
\usepackage[english]{babel}
\usepackage{framed}
\definecolor{Blue}{RGB}{0.16,0.32,0.75}
\setbeamercolor{structure}{fg=blue}
\usepackage{beamerthemesplit}
\definecolor{blue}{rgb}{0.16,0.32,0.75}
\setbeamercolor{structure}{fg=blue}
\author[FOSSEE]{}
\institute[IIT Bombay]{}
\date[]{}
% \setbeamercovered{transparent}

% theme split
\usepackage{verbatim}
\newenvironment{colorverbatim}[1][]%
{%
\color{blue}
\verbatim
}%
{%
\endverbatim
}%

\usepackage{mathpazo,courier,euler}
\usepackage{listings}
\lstset{language=sh,
    basicstyle=\ttfamily\bfseries,
  showstringspaces=false,
  keywordstyle=\color{black}\bfseries}

% logo
\logo{\includegraphics[height=1.30 cm]{St-logo.png}}
\logo{\includegraphics[height=1.30 cm]{fossee-logo.png}

\hspace{7.5cm}
\includegraphics[scale=0.3]{fossee-logo.png}\\
\hspace{281pt}
\includegraphics[scale=0.08]{St-logo.png}}


\newcounter{saveenumi}
\newcommand{\seti}{\setcounter{saveenumi}{\value{enumi}}}
\newcommand{\conti}{\setcounter{enumi}{\value{saveenumi}}}

\begin{document}
% sf family, bold font
\sffamily \bfseries
%\LARGE
\title
[Python for Scientific Computing]
%\hspace{0.3cm}
%\insertframenumber/\inserttotalframenumber]
{\large Accessing Parts of Arrays-I}
\author
[FOSSEE, IIT BOMBAY]
{{\small Spoken Tutorial Project \\ http://spoken-tutorial.org \\ National Mission on Education  through ICT  \\ http://sakshat.ac.in } \\[0.1cm]
{\small Script: Aditya Palaparthy }\\
{\small Narration: Kiran K}\\
{\small IIT Bombay} \\ [0.1cm]
{\small  02 November 2015}}
% slide 1
\begin{frame}
   \titlepage
\end{frame}
%%%%%%%%%%%%%%%%%%%%%%%%%%%%%%%%%%%%%%%%%%%%%%%%%%%%%%%%%%%%%%%%%%%%%%%%%%%%%%%%
\begin{frame}
\frametitle{Objectives}
\label{sec-2.1}

  In this tutorial we will learn, \pause
   

\begin{itemize}
\item Access and change individual elements of 
\begin{itemize}
\item single dimensional arrays 
 \item multi-dimensional arrays
\end{itemize}
\item Access and change rows and columns of arrays.
\end{itemize}
\end{frame}
%%%%%%%%%%%%%%%%%%%%%%%%%%%%%%%%%%%%%%%%%%%%%%%%%%%%%%%%%%%%%%%%%%%%%%%%%%%%%%%%
\begin{frame}
\frametitle{Objectives}
\label{sec-2.2} 

\begin{itemize}
\item Access and change other elements of an array, using slicing
    and striding.
\end{itemize}
\end{frame}
%%%%%%%%%%%%%%%%%%%%%%%%%%%%%%%%%%%%%%%%%%%%%%%%%%%%%%%%%%%%%%%%%%%%%%%%%%%%%%%%
\begin{frame}
\frametitle{System Specifications}\pause
\begin{itemize}
\item Ubuntu Linux 14.04\pause
\item \texttt{Python 2.7.6} \pause
\item \texttt{IPython 4.0.0}
\end{itemize}
\end{frame}
%%%%%%%%%%%%%%%%%%%%%%%%%%%%%%%%%%%%%%%%%%%%%%%%%%%%%%%%%%%%%%%%%%%%%%%%%%%%%%%%
\begin{frame}
\frametitle{Pre-requisite}
\label{sec-3}
To practise this tutorial, you should know how to 
\begin{itemize}
\item run basic Python commands on the ipython console.
\item use arrays.
\end{itemize}
If not, see the pre-requisite Python tutorials on {\color{blue}http://spoken-tutorial.org}
\end{frame}

\begin{frame}[fragile]
\frametitle{Sample Arrays}
\label{sec-4}
\lstset{language=Python}
\begin{footnotesize}
\begin{lstlisting}
In []: A = array([1, 2, 3, 4, 5])

In []: C = array([[1, 2, 3, 4, 5],
                  [6, 7, 8, 9, 10],
                  [11, 12, 13, 14, 15],
                  [16, 17, 18, 19, 20],
                  [21, 22, 23, 24, 25]])
\end{lstlisting}
\end{footnotesize}
\end{frame}

\begin{frame}
\frametitle{Exercise 1}
\label{sec-8}

\begin{itemize}
\item First obtain \texttt{[7, 8]} from \texttt{C}.
\item Then obtain \texttt{[1,6,11,16]} from \texttt{C}.
\item Finally obtain \texttt{[6,11,16,0]}.
\end{itemize}
\end{frame}
\begin{frame}
\frametitle{Exercise 2}
\label{sec-9}

  Obtain the elements \texttt{[[8,9],[13,-14]]} from \texttt{C}
\end{frame}

\begin{frame}[fragile]
\frametitle{Exercise 3}
\label{sec-11}

  Obtain the following
\begin{itemize}
\item \texttt{[[2, 5], [17, 20]]}\pause
\item \texttt{[[2, 3, 4], [0, 0, 0]]}
\end{itemize}
\end{frame}

\begin{frame}[fragile]
\frametitle{Solution 3}
\label{sec-12}

\begin{itemize}
\item \texttt{C[::3, 1::3]}\pause
\item \texttt{C[::4, 1:4]}
\end{itemize}

\end{frame}
\begin{frame}
\frametitle{Summary}
\label{sec-13}

  In this tutorial, we have learnt to, \pause

\begin{itemize}
\item Manipulate single \& multi dimensional arrays.\pause
\item Access and change individual elements by using their index numbers.
\end{itemize}
\end{frame}

\begin{frame}
\frametitle{Summary}
\label{sec-13}

\begin{itemize}
\item Access and change rows and columns of arrays by specifying the row 
    and column numbers.\pause
\item Slice and stride on arrays.
\end{itemize}
\end{frame}

\begin{frame}
\frametitle{Evaluation}
\label{sec-14.1}


\begin{enumerate}
\item Given the array,\\ \texttt{A = array([12,15,18,21])},\\ How do we access the element \texttt{18}?
\vspace{2pt}
\seti
\end{enumerate}
\end{frame}
%%%%%%%%%%%%%%%%%%%%%%%%%%%%%%%%%%%%%%%%%%%%%%%%%%%%%%%%%%%%%%%%%%%%%%%%%%%%%%%%
\begin{frame}[fragile]
\frametitle{Evaluation}
\label{sec-14.2}
\begin{enumerate}
\conti
\item Given the array,

\lstset{language=Python}
\begin{lstlisting}
B = array([[10, 11, 12, 13],
           [20, 21, 22, 23],
           [30, 31, 32, 33],
           [40, 41, 42, 43]])
\end{lstlisting}
Obtain the elements, \texttt{[[21, 22], [31, 32]]}
\seti
\end{enumerate}
\end{frame}
%%%%%%%%%%%%%%%%%%%%%%%%%%%%%%%%%%%%%%%%%%%%%%%%%%%%%%%%%%%%%%%%%%%%%%%%%%%%%%%%
\begin{frame}
\frametitle{Solutions}
\label{sec-15}
\begin{enumerate}
\item \texttt{A[ 2 ]}
\vspace{12pt}
\item \texttt{B[1:3, 1:3]}
\end{enumerate}
\end{frame}
%%%%%%%%%%%%%%%%%%%%%%%%%%%%%%%%%%%%%%%%%%%%%%%%%%%%%%%%%%%%%%%%%%%%%%%%%%%%%%%%
\begin{frame}
\frametitle{Forum to answer questions}
\begin{itemize}
\item Do you have questions in THIS Spoken Tutorial?
\item Choose the minute and second where you have the question.
\item Explain your question briefly.
\item Someone from the FOSSEE team will answer them. Please visit 
\end{itemize}
\begin{center}
{\color{blue}{http://forums.spoken-tutorial.org/}}
 \end{center} 
\end{frame}
%%%%%%%%%%%%%%%%%%%%%%%%%%%%%%%%%%%%%%%%%%%%%%%%%%%%%%%%%%%%%%%%%%%%%%%%%%%%%%%%
\begin{frame}
\frametitle{Forum to answer questions}
\begin{itemize}
\item Questions not related to the Spoken Tutorial?
\item Do you have general / technical questions on the Software?
\item Please visit the FOSSEE Forum
\begin{center}
{\color{blue}{http://forums.fossee.in/}}
 \end{center}
\item Choose the Software and post your question.
\end{itemize}
\end{frame}
%%%%%%%%%%%%%%%%%%%%%%%%%%%%%%%%%%%%%%%%%%%%%%%%%%%%%%%%%%%%%%%%%%
\begin{frame}
\frametitle{Textbook Companion Project}
\begin{itemize}
\item The FOSSEE team coordinates coding of solved examples of popular
  books 
\item We give honorarium and certificate to those who do this
\end{itemize}
For more details, please visit this site:
\begin{center}
{\color{blue}{http://tbc-python.fossee.in/}}
\end{center}
\end{frame}
%%%%%%%%%%%%%%%%%%%%%%%%%%%%%%%%%%%%%%%%%%%%%%%%%%%%%%%%%%%%%%%%%%%%%%%%%%%%%%%%
\begin{frame}
\frametitle{Acknowledgements}
\begin{itemize}
\item Spoken Tutorial Project is a part of the Talk to a Teacher  project 
\item It is supported by the National Mission on Education through  ICT, MHRD, Government of India 
\item More information on this Mission is available at: \\{\color{blue}\url{http://spoken-tutorial.org/NMEICT-Intro}}
\end{itemize}
\end{frame}
%%%%%%%%%%%%%%%%%%%%%%%%%%%%%%%%%%%%%%%%%%%%%%%%%%%%%%%%%%%%%%%%%%%%%%%%%%%%%%%%
\begin{frame}

  \begin{block}{}
  \begin{center}
  \textcolor{blue}{\Large THANK YOU!} 
  \end{center}
  \end{block}
\begin{block}{}
  \begin{center}
    For more Information, visit our website\\
    {http://fossee.in/}
  \end{center}  
  \end{block}
\end{frame}

\end{document}
