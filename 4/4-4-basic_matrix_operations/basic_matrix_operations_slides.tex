\documentclass[17pt]{beamer}
\usepackage{amsmath}
\usepackage{framed}
\definecolor{Blue}{RGB}{0.16,0.32,0.75}
\setbeamercolor{structure}{fg=blue}
\usepackage{beamerthemesplit}




\definecolor{blue}{rgb}{0.16,0.32,0.75}
\setbeamercolor{structure}{fg=blue}
\author[FOSSEE]{}
\institute[IIT Bombay]{}
\date[]{}
% \setbeamercovered{transparent}

% theme split
\usepackage{verbatim}
\newenvironment{colorverbatim}[1][]%
{%
\color{blue}
\verbatim
}%
{%
\endverbatim
}%

\usepackage{mathpazo,courier,euler}
\usepackage{listings}
\lstset{language=sh,
    basicstyle=\ttfamily\bfseries,
  showstringspaces=false,
  keywordstyle=\color{black}\bfseries}

% logo
\logo{\includegraphics[height=1.30 cm]{St-logo.png}}
\logo{\includegraphics[height=1.30 cm]{fossee-logo.png}

\hspace{7.5cm}
\includegraphics[scale=0.3]{fossee-logo.png}\\
\hspace{281pt}
\includegraphics[scale=0.08]{St-logo.png}}


\newcounter{saveenumi}
\newcommand{\seti}{\setcounter{saveenumi}{\value{enumi}}}
\newcommand{\conti}{\setcounter{enumi}{\value{saveenumi}}}

\begin{document}
% sf family, bold font
\sffamily \bfseries
%\LARGE
\title
[Python for Scientific Computing]
%\hspace{0.5cm}
%\insertframenumber/\inserttotalframenumber]
{\large Basic Matrix Operations}
\author
[FOSSEE, IIT Bombay]
{{\small Spoken Tutorial Project \\ http://spoken-tutorial.org \\ National Mission on Education  through ICT  \\ http://sakshat.ac.in } \\
{\small Script: Thirumalesh H S}\\
{\small Narrator: Kiran Kishore}\\
{\small IIT Bombay}\\
{\small 19 November 2015}}

% slide 1
\begin{frame}
   \titlepage
\end{frame}

%%%%%%%%%%%%%%%%%%%%%%%%%%%%%%%%%%%%%%%%%%%%%%%%%%%%%%%%%%%%%%%%%%%%%%%%%%%%%%%%
\begin{frame}
\frametitle{Objectives}
\label{sec-2}

  At the end of this tutorial, you will be able to \pause

\begin{itemize}
\item Create matrices using data.\pause
\item Create matrices from lists.\pause
\item Do basic matrix operations like addition, multiplication.
\end{itemize}
\end{frame}
%%%%%%%%%%%%%%%%%%%%%%%%%%%%%%%%%%%%%%%%%%%%%%%%%%%%%%%%%%%%%%%%%%%%%%%%%%%%%%%%
\begin{frame}
\frametitle{Objectives}
\label{sec-2.2}
\begin{itemize}
\item Perform operations to find out the
\begin{itemize}
	\item determinant of a matrix\pause
	\item inverse of a matrix\pause
	\item eigen values and eigen vectors of a matrix
\end{itemize}
\end{itemize}
\end{frame}
%%%%%%%%%%%%%%%%%%%%%%%%%%%%%%%%%%%%%%%%%%%%%%%%%%%%%%%%%%%%%%%%%%%%%%%%%%%%%%%%
\begin{frame}
\frametitle{System Specifications}\pause
\begin{itemize}
\item Ubuntu Linux 14.04\pause
\item \texttt{Python 2.7.6} \pause
\item \texttt{IPython 4.0.0}
\end{itemize}
\end{frame}
%%%%%%%%%%%%%%%%%%%%%%%%%%%%%%%%%%%%%%%%%%%%%%%%%%%%%%%%%%%%%%%%%%%%%%%%%%%%%%%%
\begin{frame}
\frametitle{Pre-requisite}
\label{sec-3}

  To practice this tutorial, you should know how to -\pause

\begin{itemize}
\item use Lists\pause
\item use arrays and access parts of arrays\pause
\end{itemize}
If not, see the pre-requisite Python tutorials on {\color{blue}http://spoken-tutorial.org}
\end{frame}
%%%%%%%%%%%%%%%%%%%%%%%%%%%%%%%%%%%%%%%%%%%%%%%%%%%%%%%%%%%%%%%%%%%%%%%%%%%%%%%%
\begin{frame}
\frametitle{Assignment 1}
\label{sec-4}

\begin{itemize}
\item Create a two dimensional matrix \texttt{m3} of order \texttt{(2, 4)} with
   elements \\ \texttt{5, 6, 7, 8, 9, 10, 11, 12}\pause
\item Use arange() and reshape() methods
\end{itemize}

\end{frame}
%%%%%%%%%%%%%%%%%%%%%%%%%%%%%%%%%%%%%%%%%%%%%%%%%%%%%%%%%%%%%%%%%%%%%%%%%%%%%%%%
\begin{frame}
\frametitle{Determinant of a matrix}
\label{sec-4}

\begin{itemize}
\item The determinant of a square matrix can be obtained by using the function det()
\end{itemize}

\end{frame}
%%%%%%%%%%%%%%%%%%%%%%%%%%%%%%%%%%%%%%%%%%%%%%%%%%%%%%%%%%%%%%%%%%%%%%%%%%%%%%%%
\begin{frame}
\frametitle{Inverse of a matrix}
\label{sec-4}

\begin{itemize}

\item The inverse of a matrix can be obtained using inv()
\end{itemize}

\end{frame}
%%%%%%%%%%%%%%%%%%%%%%%%%%%%%%%%%%%%%%%%%%%%%%%%%%%%%%%%%%%%%%%%%%%%%%%%%%%%%%%%
\begin{frame}
\frametitle{eigen values and eigen vectors}
\label{sec-9}

  Given a square matrix A \pause
\begin{itemize}
\item eig(A) gives its eigen values \pause
\item eigvals(A) gives its eigen vector   
\end{itemize}
  
\end{frame}
%%%%%%%%%%%%%%%%%%%%%%%%%%%%%%%%%%%%%%%%%%%%%%%%%%%%%%%%%%%%%%%%%%%%%%%%%%%%%%%%
\begin{frame}
\frametitle{Summary}
\label{sec-11.1}

  In this tutorial, we have learnt to, \pause

\begin{itemize}
\item Create matrices using arrays.\pause
\item Add, subtract and multiply the elements of a matrix. \pause
\item Use the function det() to find the determinant of a matrix.
\end{itemize}
\end{frame}
%%%%%%%%%%%%%%%%%%%%%%%%%%%%%%%%%%%%%%%%%%%%%%%%%%%%%%%%%%%%%%%%%%%%%%%%%%%%%%%%
\begin{frame}
\frametitle{Summary}
\label{sec-11.2}
\begin{itemize}
\item Find out the inverse of a matrix, using the function \texttt{inv()}. \pause
\item Find out the eigen vectors and eigen values of a matrix, using 
    functions \texttt{eig()} and \texttt{eigvals()}.
\end{itemize}
\end{frame}
%%%%%%%%%%%%%%%%%%%%%%%%%%%%%%%%%%%%%%%%%%%%%%%%%%%%%%%%%%%%%%%%%%%%%%%%%%%%%%%%
\begin{frame}
\frametitle{Self assessment questions}
\label{sec-12.1}

\begin{enumerate}
\item \texttt{A} and \texttt{B} are two array objects. Element wise multiplication in
   matrices is done by,\pause
\begin{itemize}
\item \texttt{A * B}
\item \texttt{multiply(A, B)}
\item \texttt{dot(A, B)}
\item \texttt{element\_multiply(A,B)}\pause
\end{itemize}
\vspace{5pt}
\item \texttt{eig(A)[ 1 ]} and \texttt{eigvals(A)} are the same.\pause
\begin{itemize}
\item True
\item False
\end{itemize}
\seti
\end{enumerate}
\end{frame}
%%%%%%%%%%%%%%%%%%%%%%%%%%%%%%%%%%%%%%%%%%%%%%%%%%%%%%%%%%%%%%%%%%%%%%%%%%%%%%%%
\begin{frame}
\frametitle{Solution of self assessment questions}
\label{sec-13}


\begin{enumerate}
\item \texttt{A * B}\pause
\vspace{12pt}
\item False
\end{enumerate}
\end{frame}
%%%%%%%%%%%%%%%%%%%%%%%%%%%%%%%%%%%%%%%%%%%%%%%%%%%%%%%%%%%%%%%%%%%%%%%%%%%%%%%%
\begin{frame}
\frametitle{Forum to answer questions}
\begin{itemize}
\item Do you have questions in THIS Spoken Tutorial?
\item Choose the minute and second where you have the question.
\item Explain your question briefly.
\item Someone from the FOSSEE team will answer them. Please visit 
\end{itemize}
\begin{center}
{\color{blue}{http://forums.spoken-tutorial.org/}}
 \end{center} 
\end{frame}
%%%%%%%%%%%%%%%%%%%%%%%%%%%%%%%%%%%%%%%%%%%%%%%%%%%%%%%%%%%%%%%%%%%%%%%%%%%%%%%%
\begin{frame}
\frametitle{Forum to answer questions}
\begin{itemize}
\item Questions not related to the Spoken Tutorial?
\item Do you have general / technical questions on the Software?
\item Please visit the FOSSEE Forum
\begin{center}
{\color{blue}{http://forums.fossee.in/}}
 \end{center}
\item Choose the Software and post your question.
\end{itemize}
\end{frame}
%%%%%%%%%%%%%%%%%%%%%%%%%%%%%%%%%%%%%%%%%%%%%%%%%%%%%%%%%%%%%%%%%%%%%%%%%%%%%%%%
\begin{frame}
\frametitle{Textbook Companion Project}
\begin{itemize}
\item The FOSSEE team coordinates coding of solved examples of popular
  books 
\item We give honorarium and certificate to those who do this
\end{itemize}
For more details, please visit this site:
\begin{center}
{\color{blue}{http://tbc-python.fossee.in/}}
\end{center}
\end{frame}
%%%%%%%%%%%%%%%%%%%%%%%%%%%%%%%%%%%%%%%%%%%%%%%%%%%%%%%%%%%%%%%%%%%%%%%%%%%%%%%%
\begin{frame}
\frametitle{Acknowledgements}
\begin{itemize}
\item Spoken Tutorial Project is a part of the Talk to a Teacher  project 
\item It is supported by the National Mission on Education through  ICT, MHRD, Government of India 
\item More information on this Mission is available at: \\{\color{blue}\url{http://spoken-tutorial.org/NMEICT-Intro}}
\end{itemize}
\end{frame}
%%%%%%%%%%%%%%%%%%%%%%%%%%%%%%%%%%%%%%%%%%%%%%%%%%%%%%%%%%%%%%%%%%%%%%%%%%%%%%%%
\begin{frame}

  \begin{block}{}
  \begin{center}
  \textcolor{blue}{\Large THANK YOU!} 
  \end{center}
  \end{block}
\begin{block}{}
  \begin{center}
    For more Information, visit our website\\
    {http://fossee.in/}
  \end{center}  
  \end{block}
\end{frame}

\end{document}
