\documentclass[17pt]{beamer}
\usepackage{amsmath}
\usepackage[english]{babel}
\usepackage{framed}
\definecolor{Blue}{RGB}{0.16,0.32,0.75}
\setbeamercolor{structure}{fg=blue}
\usepackage{beamerthemesplit}
\definecolor{blue}{rgb}{0.16,0.32,0.75}
\setbeamercolor{structure}{fg=blue}
\author[FOSSEE]{}
\institute[IIT Bombay]{}
\date[]{}
% \setbeamercovered{transparent}

% theme split
\usepackage{verbatim}
\newenvironment{colorverbatim}[1][]%
{%
\color{blue}
\verbatim
}%
{%
\endverbatim
}%

\usepackage{mathpazo,courier,euler}
\usepackage{listings}
\lstset{language=sh,
    basicstyle=\ttfamily\bfseries,
  showstringspaces=false,
  keywordstyle=\color{black}\bfseries}

% logo
\logo{\includegraphics[height=1.30 cm]{St-logo.png}}
\logo{\includegraphics[height=1.30 cm]{fossee-logo.png}

\hspace{7.5cm}
\includegraphics[scale=0.3]{fossee-logo.png}\\
\hspace{281pt}
\includegraphics[scale=0.08]{St-logo.png}}


\newcounter{saveenumi}
\newcommand{\seti}{\setcounter{saveenumi}{\value{enumi}}}
\newcommand{\conti}{\setcounter{enumi}{\value{saveenumi}}}

\begin{document}
% sf family, bold font
\sffamily \bfseries
%\LARGE
\title
[Python for Scientific Computing]
%\hspace{0.5cm}
%\insertframenumber/\inserttotalframenumber]
{Least Square Fit}
\author
[FOSSEE, IIT BOMBAY]
{{\small Spoken Tutorial Project \\ http://spoken-tutorial.org \\ National Mission on Education  through ICT  \\ http://sakshat.ac.in } \\[0.1cm]
{\small  Script: Aditya Palaparthy}\\
{\small  Narration: Kiran Kishore}\\
{\small IIT Bombay} \\ [0.1cm]
{\small  16 November 2015}}
% slide 1
\begin{frame}
   \titlepage
\end{frame}
%%%%%%%%%%%%%%%%%%%%%%%%%%%%%%%%%%%%%%%%%%%%%%%%%%%%%%%%%%%%%%%%%%%%%%%%%%%%%%%%
\begin{frame}
\frametitle{Objectives}
\label{sec-2}
At the end of this tutorial, you will be able to,  \pause

\begin{itemize}
\item Generate the least square fit line for a
   given set of points.
\end{itemize}
\end{frame}
%%%%%%%%%%%%%%%%%%%%%%%%%%%%%%%%%%%%%%%%%%%%%%%%%%%%%%%%%%%%%%%%%%%%%%%%%%%%%%%%
\begin{frame}
\frametitle{System Specifications}\pause
\begin{itemize}
\item Ubuntu Linux 14.04 \pause
\item \texttt{Python 2.7.6} \pause
\item \texttt{IPython 4.0.0}
\end{itemize}
\end{frame}
%%%%%%%%%%%%%%%%%%%%%%%%%%%%%%%%%%%%%%%%%%%%%%%%%%%%%%%%%%%%%%%%%%%%%%%%%%%%%%%%
\begin{frame}
\frametitle{Pre-requisite}
\label{sec-3}

  To practise this tutorial, you should know how to  -\pause

\begin{itemize}
\item use plot interactively.\pause
\item load data from files.\pause
\item use arrays and matrices.
\end{itemize}
If not, see the pre-requisite Python tutorials on {\color{blue}http://spoken-tutorial.org}.
\end{frame}
\begin{frame}
\frametitle{Exercise 1}
\label{sec-4}


\begin{itemize}
\item Generate a least square fit line for l v/s t$^2$ using the data in the file
    `pendulum.txt'.
\end{itemize}
\end{frame}

\begin{frame}[fragile]
  \frametitle{Matrix Formulation}
  \begin{itemize}
  \item We need to fit a line through points for the equation $T^2 = m \cdot L+c$
  \item In matrix form, the equation can be represented as $T_{sq} = A \cdot p$, where $T_{sq}$ is
    $\begin{bmatrix}
    T^2_1 \\
    T^2_2 \\
    \vdots\\
    T^2_N \\
  \end{bmatrix}$
    , A is   
    $\begin{bmatrix}
    L_1 & 1 \\
    L_2 & 1 \\
    \vdots & \vdots\\
    L_N & 1 \\
  \end{bmatrix}$
\end{itemize}
\end{frame}

\begin{frame}[fragile]
  \frametitle{Matrix Formulation}
  
    and p is 
    $\begin{bmatrix}
      m\\
      c\\
    \end{bmatrix}$
   \\  We need to find $p$ to plot the line
\end{frame}

\begin{frame}
\frametitle{Summary}
\label{sec-5}
In this tutorial, we have learnt to,
\begin{itemize}
\item Generate a least square fit using matrices. \pause
\item Use the function \texttt{lstsq()} to generate a least square fit line.
\end{itemize}
\end{frame}

%%%%%%%%%%%%%%%%%%%%%%%%%%%%%%%%%%%%%%%%%%%%%%%%%%

\begin{frame}
\frametitle{Evaluation}
\label{sec-6}


\begin{enumerate}
\item What does \texttt{ones\_{like}([1, 2, 3])} produce \pause
\begin{itemize}
\item \texttt{array([1, 1, 1])}
\item \texttt{[1, 1, 1]}
\item \texttt{[1.0, 1.0, 1.0]}
\item Error\pause
\end{itemize}
\vspace{10pt}
\item The plot of ``u'' vs ``v'' is a bunch of scattered points that show a
     linear trend. How do you find the least square fit line of ``u'' v/s ``v''.
\end{enumerate}
\end{frame}
\begin{frame}
\frametitle{Solutions}
\label{sec-7}


\begin{enumerate}
\item \texttt{array([1, 1, 1])}\pause
\vspace{15pt}
\item \texttt{A = array(u, ones\_like(u)).T}\\
     \texttt{result = lstsq(A, v)}\\
     \texttt{m, c = result[ 0 ]}\\
     \texttt{lst\_line = m * u + c}
\end{enumerate}
\end{frame}

%%%%%%%%%%%%%%%%%%%%%%%%%%%%%%%%%%%%%%%%%%%%%%%%%%%%%%%%%%%%%%%%%%%%%%%%%%%%%%%%
\begin{frame}
\frametitle{Forum to answer questions}
\begin{itemize}
\item Do you have questions in THIS Spoken Tutorial?
\item Choose the minute and second where you have the question.
\item Explain your question briefly.
\item Someone from the FOSSEE team will answer them. Please visit 
\end{itemize}
\begin{center}
{\color{blue}{http://forums.spoken-tutorial.org/}}
 \end{center} 
\end{frame}
%%%%%%%%%%%%%%%%%%%%%%%%%%%%%%%%%%%%%%%%%%%%%%%%%%%%%%%%%%%%%%%%%%%%%%%%%%%%%%%%
\begin{frame}
\frametitle{Forum to answer questions}
\begin{itemize}
\item Questions not related to the Spoken Tutorial?
\item Do you have general / technical questions on the Software?
\item Please visit the FOSSEE Forum
\begin{center}
{\color{blue}{http://forums.fossee.in/}}
 \end{center}
\item Choose the Software and post your question.
\end{itemize}
\end{frame}
%%%%%%%%%%%%%%%%%%%%%%%%%%%%%%%%%%%%%%%%%%%%%%%%%%%%%%%%%%%%%%%%%%%%%%%%%%%%%%%%
\begin{frame}
\frametitle{Textbook Companion Project}
\begin{itemize}
\item The FOSSEE team coordinates coding of solved examples of popular
  books 
\item We give honorarium and certificate to those who do this
\end{itemize}
For more details, please visit this site:
\begin{center}
{\color{blue}{http://tbc-python.fossee.in/}}
\end{center}
\end{frame}
%%%%%%%%%%%%%%%%%%%%%%%%%%%%%%%%%%%%%%%%%%%%%%%%%%%%%%%%%%%%%%%%%%%%%%%%%%%%%%%%
\begin{frame}
\frametitle{Acknowledgements}
\begin{itemize}
\item Spoken Tutorial Project is a part of the Talk to a Teacher  project 
\item It is supported by the National Mission on Education through  ICT, MHRD, Government of India 
\item More information on this Mission is available at: \\{\color{blue}\url{http://spoken-tutorial.org/NMEICT-Intro}}
\end{itemize}
\end{frame}
%%%%%%%%%%%%%%%%%%%%%%%%%%%%%%%%%%%%%%%%%%%%%%%%%%%%%%%%%%%%%%%%%%%%%%%%%%%%%%%%
\begin{frame}

  \begin{block}{}
  \begin{center}
  \textcolor{blue}{\Large THANK YOU!} 
  \end{center}
  \end{block}
\begin{block}{}
  \begin{center}
    For more Information, visit our website\\
    {http://fossee.in/}
  \end{center}  
  \end{block}
\end{frame}

\end{document}
