% Created 2010-10-12 Tue 12:55
\documentclass[presentation]{beamer}
\usepackage[latin1]{inputenc}
\usepackage[T1]{fontenc}
\usepackage{fixltx2e}
\usepackage{graphicx}
\usepackage{longtable}
\usepackage{float}
\usepackage{wrapfig}
\usepackage{soul}
\usepackage{t1enc}
\usepackage{textcomp}
\usepackage{marvosym}
\usepackage{wasysym}
\usepackage{latexsym}
\usepackage{amssymb}
\usepackage{hyperref}
\tolerance=1000
\usepackage[english]{babel} \usepackage{ae,aecompl}
\usepackage{mathpazo,courier,euler} \usepackage[scaled=.95]{helvet}
\usepackage{listings}
\lstset{language=Python, basicstyle=\ttfamily\bfseries,
commentstyle=\color{red}\itshape, stringstyle=\color{darkgreen},
showstringspaces=false, keywordstyle=\color{blue}\bfseries}
\providecommand{\alert}[1]{\textbf{#1}}

\title{Getting started with for}
\author{FOSSEE}
\date{}

\usetheme{Warsaw}\usecolortheme{default}\useoutertheme{infolines}\setbeamercovered{transparent}
\begin{document}

\maketitle









\begin{frame}
\frametitle{Outline}
\label{sec-1}

\begin{itemize}
\item \texttt{for} loop in Python.
\item Blocks of code in Python.

\begin{itemize}
\item Indentation
\end{itemize}

\end{itemize}
\end{frame}
\begin{frame}[fragile]
\frametitle{Whitespace in Python}
\label{sec-2}

\begin{itemize}
\item Whitespace is significant

\begin{itemize}
\item blocks are visually separated
\end{itemize}

\item Blocks are indented using 4 spaces
\begin{verbatim}
     Block A
     Block A
         Block B
         Block B
     Block A
\end{verbatim}

    \texttt{Block B} is an inner block and is indented using 4 spaces
\end{itemize}
\end{frame}
\begin{frame}[fragile]
\frametitle{Exercise 1}
\label{sec-3}

  Write a \texttt{for} loop which iterates through a list of numbers and find
  the square root of each number.
\begin{verbatim}
   
\end{verbatim}

  The numbers are,
\begin{verbatim}
   1369, 7225, 3364, 7056, 5625, 729, 7056, 
   576, 2916
\end{verbatim}
\end{frame}
\begin{frame}[fragile]
\frametitle{Solution 1}
\label{sec-4}

\begin{itemize}
\item Open text editor and type the following code
\end{itemize}

\begin{verbatim}
numbers = [1369, 7225, 3364, 7056, 5625, 729, 7056, 
           576, 2916]

for each in numbers:
    print "Square root of", each, "is", sqrt(each)

print "This is not in for loop!"
\end{verbatim}
\end{frame}
\begin{frame}[fragile]
\frametitle{Save \& run script}
\label{sec-5}

\begin{itemize}
\item Save the script as \texttt{list\_roots.py}
\item Run in \texttt{ipython} interpreter as,
\begin{verbatim}
     In []: %run -i list_roots.py
\end{verbatim}

\end{itemize}
\end{frame}
\begin{frame}[fragile]
\frametitle{Exercise 2}
\label{sec-6}

  From the given numbers make a list of perfect squares and a list of those which are not.
\begin{verbatim}
   
\end{verbatim}

  The numbers are,
\begin{verbatim}
   7225, 3268, 3364, 2966, 7056, 5625, 729, 5547, 
   7056, 576, 2916
\end{verbatim}
\end{frame}
\begin{frame}[fragile]
\frametitle{Exercise 3 (indentation in \texttt{ipython})}
\label{sec-7}

  Print the square root of numbers in the list.
\begin{verbatim}
   
\end{verbatim}

  Numbers are,
\begin{verbatim}
   7225, 3268, 3364, 2966, 7056, 5625, 729, 5547, 
   7056, 576, 2916
\end{verbatim}
\end{frame}
\begin{frame}[fragile]
\frametitle{Indentation in \texttt{ipython}}
\label{sec-8}

\begin{verbatim}
   In []: numbers = [1369, 7225, 3364, 7056, 5625, 
     ...:  729, 7056, 576, 2916]
\end{verbatim}


\begin{verbatim}
   In []: for each in numbers:
     ...:     
\end{verbatim}

  Note the four spaces here
\begin{verbatim}
   
   
   
   
   
   
\end{verbatim}
\end{frame}
\begin{frame}[fragile]
\frametitle{Indentation in \texttt{ipython} (cont'd)}
\label{sec-9}

\begin{verbatim}
   In []: numbers = [1369, 7225, 3364, 7056, 5625, 
     ...:  729, 7056, 576, 2916]
   In []: for each in numbers:
     ...:     
\end{verbatim}

  Note the four spaces here
\begin{verbatim}
   
\end{verbatim}

  Now type the rest of the code
\begin{verbatim}
     ...:     print "Square root of", each, 
     ...:     print "is", sqrt(each)
     ...:     
     ...:     
\end{verbatim}
\end{frame}
\begin{frame}[fragile]
\frametitle{Indentation in \texttt{python} interpreter}
\label{sec-10}

  Find out the cube of all the numbers from 1 to 10.
\begin{verbatim}
   
\end{verbatim}

  \emph{do it in the python interpreter}
\end{frame}
\begin{frame}[fragile]
\frametitle{Indentation in \texttt{python} interpreter (cont'd)}
\label{sec-11}

\begin{verbatim}
>>> for i in range(1, 11):
...     print i, "cube is", i**3
...
\end{verbatim}
\end{frame}
\begin{frame}
\frametitle{\texttt{range()} function}
\label{sec-12}

\begin{itemize}
\item in built function in Python
\item generates a list of integers

\begin{itemize}
\item \emph{syntax:} range([start,] stop[, step])
\item \emph{example:}

\begin{itemize}
\item range(1, 20) - \emph{generates integers from 1 to 20}
\item range(20) - \emph{generates integers from 0 to 20}
\end{itemize}

\end{itemize}

\end{itemize}
\end{frame}
\begin{frame}
\frametitle{Exercise 4}
\label{sec-13}

  Print all the odd numbers from 1 to 50.
\end{frame}
\begin{frame}
\frametitle{Summary}
\label{sec-14}

\begin{itemize}
\item blocks in \texttt{python}
\item indentation
\item blocks in \texttt{ipython} interpreter
\item \texttt{for} loop
\item iterating over list using \texttt{for} loop
\item \texttt{range()} function
\end{itemize}
\end{frame}
\begin{frame}
\frametitle{Thank you!}
\label{sec-15}

  \begin{block}{}
  \begin{center}
  This spoken tutorial has been produced by the
  \textcolor{blue}{FOSSEE} team, which is funded by the 
  \end{center}
  \begin{center}
    \textcolor{blue}{National Mission on Education through \\
      Information \& Communication Technology \\ 
      MHRD, Govt. of India}.
  \end{center}  
  \end{block}
\end{frame}

\end{document}
