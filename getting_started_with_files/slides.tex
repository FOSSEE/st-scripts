% Created 2011-05-24 Tue 11:43
\documentclass[presentation]{beamer}
\usepackage[utf8]{inputenc}
\usepackage[T1]{fontenc}
\usepackage{fixltx2e}
\usepackage{graphicx}
\usepackage{longtable}
\usepackage{float}
\usepackage{wrapfig}
\usepackage{soul}
\usepackage{textcomp}
\usepackage{marvosym}
\usepackage{wasysym}
\usepackage{latexsym}
\usepackage{amssymb}
\usepackage{hyperref}
\tolerance=1000
\usepackage[english]{babel} \usepackage{ae,aecompl}
\usepackage{mathpazo,courier,euler} \usepackage[scaled=.95]{helvet}
\usepackage{listings}
\lstset{language=Python, basicstyle=\ttfamily\bfseries,
commentstyle=\color{red}\itshape, stringstyle=\color{darkgreen},
showstringspaces=false, keywordstyle=\color{blue}\bfseries}
\providecommand{\alert}[1]{\textbf{#1}}

\title{}
\author{FOSSEE}
\date{}

\usetheme{Warsaw}\usecolortheme{default}\useoutertheme{infolines}\setbeamercovered{transparent}
\begin{document}











\begin{frame}

\begin{center}
\vspace{12pt}
\textcolor{blue}{\huge Getting started with Files}
\end{center}
\vspace{18pt}
\begin{center}
\vspace{10pt}
\includegraphics[scale=0.95]{../images/fossee-logo.png}\\
\vspace{5pt}
\scriptsize Developed by FOSSEE Team, IIT-Bombay. \\ 
\scriptsize Funded by National Mission on Education through ICT\\
\scriptsize  MHRD,Govt. of India\\
\includegraphics[scale=0.30]{../images/iitb-logo.png}\\
\end{center}
\end{frame}
\begin{frame}
\frametitle{Objectives}
\label{sec-2}

  At the end of this tutorial, you will be able to, 

\begin{itemize}
\item Open a file.
\item Read the contents of the file line by line.
\item Read the entire content of file at once.
\item Append the lines of a file to a list.
\item Close the file.
\end{itemize}
\end{frame}
\begin{frame}
\frametitle{Pre-requisite}
\label{sec-3}

  Spoken tutorial on -

\begin{itemize}
\item Getting started with Lists.
\item Getting started with For.
\end{itemize}
\end{frame}
\begin{frame}
\frametitle{Exercise 1}
\label{sec-4}

  Split the variable into a list, \texttt{pend\_list}, of the lines in the
  file.
\end{frame}
\begin{frame}
\frametitle{Exercise 2}
\label{sec-5}

  Re-open the file \texttt{pendulum.txt} with \texttt{f} as the file object.
\end{frame}
\begin{frame}
\frametitle{Summary}
\label{sec-6}

  In this tutorial, we have learnt to –

\begin{itemize}
\item Open and close files using the ``open`` and ``close`` functions respectively.
\item Read the data in the files as a whole,by using the ``read`` function.
\item Read the data in the files line by line by iterating over the file object
    using the ``for`` loop.
\item Append the lines of a file to a list using the ``append`` function within
    the  ``for`` loop.
\end{itemize}
\end{frame}
\begin{frame}
\frametitle{Evaluation}
\label{sec-7}


\begin{enumerate}
\item The ``open`` function returns a
\begin{itemize}
\item string
\item list
\item file object
\item function
\end{itemize}
\vspace{8pt}
\item What does the function ``splitlines()`` do.
\begin{itemize}
\item Displays the data as strings,all in a line
\item Displays the data line by line as strings
\item Displays the data line by line but not as strings
\end{itemize}
\end{enumerate}
\end{frame}
\begin{frame}
\frametitle{Solutions}
\label{sec-8}


\begin{enumerate}
\item file object
\vspace{10pt}
\item Displays the data line by line as strings
\end{enumerate}
\end{frame}
\begin{frame}

  \begin{block}{}
  \begin{center}
  \textcolor{blue}{\Large THANK YOU!} 
  \end{center}
  \end{block}
\begin{block}{}
  \begin{center}
    For more Information, visit our website\\
    \url{http://fossee.in/}
  \end{center}  
  \end{block}
\end{frame}

\end{document}