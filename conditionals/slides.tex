% Created 2011-06-24 Fri 10:08
\documentclass[presentation]{beamer}
\usepackage[latin1]{inputenc}
\usepackage[T1]{fontenc}
\usepackage{fixltx2e}
\usepackage{graphicx}
\usepackage{longtable}
\usepackage{float}
\usepackage{wrapfig}
\usepackage{soul}
\usepackage{textcomp}
\usepackage{marvosym}
\usepackage{wasysym}
\usepackage{latexsym}
\usepackage{amssymb}
\usepackage{hyperref}
\tolerance=1000
\usepackage[english]{babel} \usepackage{ae,aecompl}
\usepackage{mathpazo,courier,euler} \usepackage[scaled=.95]{helvet}
\usepackage{listings}
\lstset{language=Python, basicstyle=\ttfamily\bfseries,
commentstyle=\color{red}\itshape, stringstyle=\color{darkgreen},
showstringspaces=false, keywordstyle=\color{blue}\bfseries}
\providecommand{\alert}[1]{\textbf{#1}}

\title{}
\author{FOSSEE}
\date{}

\usetheme{Warsaw}\usecolortheme{default}\useoutertheme{infolines}\setbeamercovered{transparent}
\begin{document}











\begin{frame}

\begin{center}
\vspace{12pt}
\textcolor{blue}{\huge Conditionals}
\end{center}
\vspace{18pt}
\begin{center}
\vspace{10pt}
\includegraphics[scale=0.95]{../images/fossee-logo.png}\\
\vspace{5pt}
\scriptsize Developed by FOSSEE Team, IIT-Bombay. \\ 
\scriptsize Funded by National Mission on Education through ICT\\
\scriptsize  MHRD,Govt. of India\\
\includegraphics[scale=0.30]{../images/iitb-logo.png}\\
\end{center}
\end{frame}
\begin{frame}
\frametitle{Objectives}
\label{sec-2}

At the end of this tutorial, you will be able to, 


\begin{itemize}
\item Use if/else blocks.
\item Use if/elif/else blocks.
\item Use the Ternary conditional statement - C if X else Y.
\end{itemize}
  
\end{frame}
\begin{frame}
\frametitle{Pre-requisite}
\label{sec-3}

Spoken tutorial on -

\begin{itemize}
\item Basic Datatypes and Operators
\end{itemize}
\end{frame}
\begin{frame}[fragile]
\frametitle{\verb~if/elif~ ladder}
\label{sec-4}

\lstset{language=Python}
\begin{lstlisting}
if user == 'admin':
    # Do admin operations
elif user == 'moderator':
    # Do moderator operations
elif user == 'client':
    # Do customer operations
\end{lstlisting}
\end{frame}
\begin{frame}
\frametitle{Exercise 1}
\label{sec-5}

  Given a number, num. Write an if else block to print num, as is, if
  it is divisible by 10, else print 10 * num.
\end{frame}
\begin{frame}[fragile]
\frametitle{Solution 1}
\label{sec-6}

\lstset{language=Python}
\begin{lstlisting}
if num%10 == 0: 
    print num   
else:           
    print 10*num
\end{lstlisting}
\end{frame}
\begin{frame}
\frametitle{Exercise 2}
\label{sec-7}

  Given a number, num. Write a ternary operator to print num, as is,
  if it is divisible by 10, else print 10 * num.
\end{frame}
\begin{frame}[fragile]
\frametitle{Solution 2}
\label{sec-8}

\lstset{language=Python}
\begin{lstlisting}
print num if num%10 == 0 else 10*num
\end{lstlisting}
\end{frame}
\begin{frame}[fragile]
\frametitle{Pass statement}
\label{sec-9}

a = raw\_input("Enter `c' to calculate and exit, `d' to display the existing results exit and `x' to exit and any other key to continue: ")
\lstset{language=Python}
\begin{lstlisting}
if a == 'c':
   # Calculate the marks and exit
elif a == 'd':
   # Display the results and exit
elif a == 'x':
   # Exit the program
else:
   pass
\end{lstlisting}
\end{frame}
\begin{frame}
\frametitle{Summary}
\label{sec-10}

 In this tutorial, we have learnt to,


\begin{itemize}
\item Understand the conditional statements in Python.
\item Use if/else statement.
\item Use if/elif/else statement.
\item Apply the ternary conditional statement - C if X else Y.
\item Use ``pass'' statement.
\end{itemize}
\end{frame}
\begin{frame}[fragile]
\frametitle{Evaluation}
\label{sec-11}


\begin{enumerate}
\item Use conditional statements for the following.
   Given a variable ``time'', print ``Good Morning'' if it is less
   than 12, otherwise print ``Hello''.
\vspace{10pt}   
\item Convert the if else ladder below into a ternary conditional
   statement.
\lstset{language=Python}
\begin{lstlisting}
x = 20

if x > 10:
    print x * 100
else:
    print x
\end{lstlisting}
\end{enumerate}
\end{frame}
\begin{frame}[fragile]
\frametitle{Solutions}
\label{sec-12}


\begin{enumerate}
\item \lstset{language=Python}
\begin{lstlisting}
if time < 12:
    print "Good Morning"
else:
    print "Hello"
\end{lstlisting}

\vspace{9pt}
\item \lstset{language=Python}
\begin{lstlisting}
print x * 100 if x > 10 else x
\end{lstlisting}
\end{enumerate}
\end{frame}
\begin{frame}

  \begin{block}{}
  \begin{center}
  \textcolor{blue}{\Large THANK YOU!} 
  \end{center}
  \end{block}
\begin{block}{}
  \begin{center}
    For more Information, visit our website\\
    \url{http://fossee.in/}
  \end{center}  
  \end{block}
\end{frame}
\end{frame}

\end{document}