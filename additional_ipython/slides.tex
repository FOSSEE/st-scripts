% Created 2010-10-10 Sun 17:30
\documentclass[presentation]{beamer}
\usepackage[latin1]{inputenc}
\usepackage[T1]{fontenc}
\usepackage{fixltx2e}
\usepackage{graphicx}
\usepackage{longtable}
\usepackage{float}
\usepackage{wrapfig}
\usepackage{soul}
\usepackage{textcomp}
\usepackage{marvosym}
\usepackage{wasysym}
\usepackage{latexsym}
\usepackage{amssymb}
\usepackage{hyperref}
\tolerance=1000
\usepackage[english]{babel} \usepackage{ae,aecompl}
\usepackage{mathpazo,courier,euler} \usepackage[scaled=.95]{helvet}
\usepackage{listings}
\lstset{language=Python, basicstyle=\ttfamily\bfseries,
commentstyle=\color{red}\itshape, stringstyle=\color{darkgreen},
showstringspaces=false, keywordstyle=\color{blue}\bfseries}
\providecommand{\alert}[1]{\textbf{#1}}

\title{Additional Features of \texttt{ipython}}
\author{FOSSEE}
\date{}

\usetheme{Warsaw}\usecolortheme{default}\useoutertheme{infolines}\setbeamercovered{transparent}
\begin{document}

\maketitle










\begin{frame}
\frametitle{Outline}
\label{sec-1}

\begin{itemize}
\item Retrieving ipython history
\item Viewing a part of the history
\item Saving (relevant) parts of the history to a file
\item Running a script from within ipython
\end{itemize}
\end{frame}
\begin{frame}
\frametitle{Question 1}
\label{sec-2}

  Read through the documentation of \texttt{\%hist} and find out how to list
  all the commands between 5 and 10
\end{frame}
\begin{frame}[fragile]
\frametitle{Solution 1}
\label{sec-3}

\lstset{language=Python}
\begin{lstlisting}
In []: %hist 5 10
\end{lstlisting}
\end{frame}
\begin{frame}
\frametitle{Question 2}
\label{sec-4}

  Change the label on y-axis to ``y'' and save the lines of code
  accordingly
\end{frame}
\begin{frame}[fragile]
\frametitle{Solution 2}
\label{sec-5}

\lstset{language=Python}
\begin{lstlisting}
In []: ylabel("y")
In []: %save /home/fossee/example_plot.py 1 3-6 10
\end{lstlisting}
\end{frame}
\begin{frame}
\frametitle{Question 3}
\label{sec-6}

  Use \texttt{\%hist} and \texttt{\%save} and create a script that has show in it and
  run it to produce and show the plot.
\end{frame}
\begin{frame}[fragile]
\frametitle{Solution 3}
\label{sec-7}

\lstset{language=Python}
\begin{lstlisting}
In []: %hist 20

In []: %save /home/fossee/show_included.py 1 3-6 8 10 13
In []: %run -i /home/fossee/show_included.py
\end{lstlisting}
\end{frame}
\begin{frame}
\frametitle{Question 4}
\label{sec-8}

  Run the script without using the -i option. Do you find any
  difference?
\end{frame}
\begin{frame}
\frametitle{Solution 4}
\label{sec-9}

  We see that it raises \texttt{NameError} saying the name \texttt{linspace} is not
  found.
\end{frame}
\begin{frame}
\frametitle{Summary}
\label{sec-10}

\begin{itemize}
\item Retreiving history using \texttt{\%hist} command
\item Vieweing only a part of history by passing an argument to \%hist
\item Saving the required lines of code in required order using \%save
\item Using \%run -i command to run the saved script
\end{itemize}
\end{frame}
\begin{frame}
\frametitle{Thank you!}
\label{sec-11}

  \begin{block}{}
  \begin{center}
  This spoken tutorial has been produced by the
  \textcolor{blue}{FOSSEE} team, which is funded by the 
  \end{center}
  \begin{center}
    \textcolor{blue}{National Mission on Education through \\
      Information \& Communication Technology \\ 
      MHRD, Govt. of India}.
  \end{center}  
  \end{block}
\end{frame}

\end{document}
