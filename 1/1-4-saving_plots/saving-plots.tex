\documentclass[17pt]{beamer}
\usepackage{amsmath}
\usepackage{framed}
\definecolor{Blue}{RGB}{0.16,0.32,0.75}
\setbeamercolor{structure}{fg=blue}
\usepackage{beamerthemesplit}




\definecolor{blue}{rgb}{0.16,0.32,0.75}
\setbeamercolor{structure}{fg=blue}
\author[FOSSEE]{}
\institute[IIT Bombay]{}
\date[]{}
% \setbeamercovered{transparent}

% theme split
\usepackage{verbatim}
\newenvironment{colorverbatim}[1][]%
{%
\color{blue}
\verbatim
}%
{%
\endverbatim
}%

\usepackage{mathpazo,courier,euler}
\usepackage{listings}
\lstset{language=sh,
    basicstyle=\ttfamily\bfseries,
  showstringspaces=false,
  keywordstyle=\color{black}\bfseries}

% logo
\logo{\includegraphics[height=1.30 cm]{St-logo.png}}
\logo{\includegraphics[height=1.30 cm]{fossee-logo.png}

\hspace{7.5cm}
\includegraphics[scale=0.3]{fossee-logo.png}\\
\hspace{281pt}
\includegraphics[scale=0.08]{St-logo.png}}


\newcounter{saveenumi}
\newcommand{\seti}{\setcounter{saveenumi}{\value{enumi}}}
\newcommand{\conti}{\setcounter{enumi}{\value{saveenumi}}}

\begin{document}
% sf family, bold font
\sffamily \bfseries
%\LARGE
\title
[Python for Scientific Computing]
%\hspace{0.5cm}
%\insertframenumber/\inserttotalframenumber]
{\large Saving Plots}
\author
[FOSSEE, IIT Bombay]
{{\small Spoken Tutorial Project \\ http://spoken-tutorial.org \\ National Mission on Education  through ICT  \\ http://sakshat.ac.in } \\
{\small Script: Thirumalesh H S}\\
{\small Narrator: Kiran Kishore}\\
{\small IIT Bombay}\\
{\small  16 October 2015}}

% slide 1
\begin{frame}
   \titlepage
\end{frame}
%%%%%%%%%%%%%%%%%%%%%%%%%%%%%%%%%%%%%%%%%%%%%%%%%%%%%%%%%%%%%%%%%%%%%%%%%%%%%%%%
\begin{frame}
\frametitle{Objectives} 
\label{sec-2}
At the end of this tutorial, you will be able to -\pause
\begin{itemize}
\item Save plots using \texttt{savefig()}function. \pause
\item Save plots in different formats.
\end{itemize}
\end{frame}
%%%%%%%%%%%%%%%%%%%%%%%%%%%%%%%%%%%%%%%%%%%%%%%%%%%%%%%%%%%%%%%%%%%%%%%%%%%%%%%%
\begin{frame}
\frametitle{System Specifications}\pause
\begin{itemize}
\item Ubuntu Linux 14.04\pause
\item \texttt{Python 2.7.6} \pause
\item \texttt{IPython 4.0.0}
\end{itemize}
\end{frame}
%%%%%%%%%%%%%%%%%%%%%%%%%%%%%%%%%%%%%%%%%%%%%%%%%%%%%%%%%%%%%%%%%%%%%%%%%%%%%%%%
\begin{frame}
\frametitle{Pre-requisites}
\label{sec-3}
To practise this tutorial, you should know how to -
\begin{itemize}
\item Use Plot Command Interactively
\end{itemize}
If not, see the pre-requisite Python tutorials on
{\color{blue}http://spoken-tutorial.org}
\end{frame}
%%%%%%%%%%%%%%%%%%%%%%%%%%%%%%%%%%%%%%%%%%%%%%%%%%%%%%%%%%%%%%%%%%%%%%%%%%%%%%%%
\begin{frame}[fragile]
\frametitle{Creating a basic plot}
\label{sec-4}
Let us plot a sine wave from -3*pi to 3*pi.

\end{frame}
%%%%%%%%%%%%%%%%%%%%%%%%%%%%%%%%%%%%%%%%%%%%%%%%%%%%%%%%%%%%%%%%%%%%%%%%%%%%%%%%
\begin{frame}[fragile]
\frametitle{savefig()}
\label{sec-5}
\begin{itemize}
\item \texttt{savefig()} - to save plots\\\pause
syntax: \texttt{savefig(fname)}\pause
\item savefig function takes one argument which is the filename.
\end{itemize} % ends low level
\end{frame}
%%%%%%%%%%%%%%%%%%%%%%%%%%%%%%%%%%%%%%%%%%%%%%%%%%%%%%%%%%%%%%%%%%%%%%%%%%%%%%%%
\begin{frame}
\frametitle{More on savefig()}
\label{sec-6}
savefig can save the plot in many formats, such as \pause
\begin{itemize}

\label{sec-6_2}%\begin{itemize}
\begin{footnotesize}
\item .pdf - PDF(Portable Document Format)\pause\\
\label{sec-6_2_1}%
\item .ps - PS(Post Script)\pause\\
\label{sec-6_2_2}%
\item .eps - Encapsulated Post Script\\
\label{sec-6_2_3}%
\verb~to be used with~ \LaTeX{} \verb~documents~\pause
\item .svg - Scalable Vector Graphics\\
\label{sec-6_2_4}%
\item .png - Portable Network Graphics\\
\end{footnotesize}
\end{itemize} % ends low level
\end{frame}
%%%%%%%%%%%%%%%%%%%%%%%%%%%%%%%%%%%%%%%%%%%%%%%%%%%%%%%%%%%%%%%%%%%%%%%%%%%%%%%%
\begin{frame}
\frametitle{Exercise 1}
\label{sec-7}
Save the sine plot in the eps format
\end{frame}
%%%%%%%%%%%%%%%%%%%%%%%%%%%%%%%%%%%%%%%%%%%%%%%%%%%%%%%%%%%%%%%%%%%%%%%%%%%%%%%%
\begin{frame}
\frametitle{Exercise 2}
\label{sec-8}
Save the sine plot in PDF, PS and SVG formats.
\end{frame}
%%%%%%%%%%%%%%%%%%%%%%%%%%%%%%%%%%%%%%%%%%%%%%%%%%%%%%%%%%%%%%%%%%%%%%%%%%%%%%%%
\begin{frame}
\frametitle{Summary}
\label{sec-9}
In this tutorial, we have learnt to -\pause
\begin{itemize}
\item Save plots using the \texttt{savefig()} function.\pause
\item Save the plots in different formats.\pause
\begin{itemize}
\item pdf
\item ps
\item png
\item svg
\item eps
\end{itemize}
\end{itemize}
\end{frame}
%%%%%%%%%%%%%%%%%%%%%%%%%%%%%%%%%%%%%%%%%%%%%%%%%%%%%%%%%%%%%%%%%%%%%%%%%%%%%%%%
\begin{frame}
\frametitle{Assignment}
\label{sec-10}
\begin{enumerate}
\item Which command is used to save a plot.\pause
	\begin{itemize}
	\item \texttt{saveplot()}\pause
	\item \texttt{savefig()}\pause
	\item \texttt{savefigure()}\pause
	\item \texttt{saveplt()}
	\end{itemize}
	\seti
\end{enumerate}
\end{frame}
%%%%%%%%%%%%%%%%%%%%%%%%%%%%%%%%%%%%%%%%%%%%%%%%%%%%%%%%%%%%%%%%%%%%%%%%%%%%%%%%
\begin{frame}
\frametitle{Assignment}
\begin{enumerate}
\conti
\item \texttt{savefig('sine.png')} saves the plot in,
\begin{itemize}
\item The root directory ``/'' (on GNU/Linux, Unix based systems),
``C:'' (on windows).\pause
\item Will result in an error as full path is not supplied.\pause
\item The current working directory.\pause
\item Predefined directory like ``/documents''.
\end{itemize}
\end{enumerate}
\end{frame}
%%%%%%%%%%%%%%%%%%%%%%%%%%%%%%%%%%%%%%%%%%%%%%%%%%%%%%%%%%%%%%%%%%%%%%%%%%%%%%%%
\begin{frame}
\frametitle{Solutions}
\label{sec-11}
\begin{enumerate}
\item \texttt{savefig()}
\item The current working directory
\end{enumerate}
\end{frame}
%%%%%%%%%%%%%%%%%%%%%%%%%%%%%%%%%%%%%%%%%%%%%%%%%%%%%%%%%%%%%%%%%%%%%%%%%%%%%%%%
\begin{frame}
\frametitle{Forum to answer questions}
\begin{itemize}
\item Do you have questions in THIS Spoken Tutorial?
\item Choose the minute and second where you have the question.
\item Explain your question briefly.
\item Someone from the FOSSEE team will answer them. Please visit 
\end{itemize}
\begin{center}
{\color{blue}{http://forums.spoken-tutorial.org/}}
 \end{center} 
\end{frame}
%%%%%%%%%%%%%%%%%%%%%%%%%%%%%%%%%%%%%%%%%%%%%%%%%%%%%%%%%%%%%%%%%%%%%%%%%%%%%%%%
\begin{frame}
\frametitle{Forum to answer questions}
\begin{itemize}
\item Questions not related to the Spoken Tutorial?
\item Do you have general / technical questions on the Software?
\item Please visit the FOSSEE Forum
\begin{center}
{\color{blue}{http://forums.fossee.in/}}
 \end{center}
\item Choose the Software and post your question.
\end{itemize}
\end{frame}
%%%%%%%%%%%%%%%%%%%%%%%%%%%%%%%%%%%%%%%%%%%%%%%%%%%%%%%%%%%%%%%%%%%%%%%%%%%%%%%%
\begin{frame}
\frametitle{Textbook Companion Project}
\begin{itemize}
\item The FOSSEE team coordinates coding of solved examples of popular
  books 
\item We give honorarium and certificate to those who do this
\end{itemize}
For more details, please visit this site:
\begin{center}
{\color{blue}{http://tbc-python.fossee.in/}}
\end{center}
\end{frame}
%%%%%%%%%%%%%%%%%%%%%%%%%%%%%%%%%%%%%%%%%%%%%%%%%%%%%%%%%%%%%%%%%%%%%%%%%%%%%%%%
\begin{frame}
\frametitle{Acknowledgements}
\begin{itemize}
\item Spoken Tutorial Project is a part of the Talk to a Teacher  project 
\item It is supported by the National Mission on Education through  ICT, MHRD, Government of India 
\item More information on this Mission is available at: \\{\color{blue}\url{http://spoken-tutorial.org/NMEICT-Intro}}
\end{itemize}
\end{frame}
%%%%%%%%%%%%%%%%%%%%%%%%%%%%%%%%%%%%%%%%%%%%%%%%%%%%%%%%%%%%%%%%%%%%%%%%%%%%%%%%
\begin{frame}

  \begin{block}{}
  \begin{center}
  \textcolor{blue}{\Large THANK YOU!} 
  \end{center}
  \end{block}
\begin{block}{}
  \begin{center}
    For more Information, visit our website\\
    {http://fossee.in/}
  \end{center}  
  \end{block}
\end{frame}

\end{document}
