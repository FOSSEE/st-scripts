% Created 2010-10-11 Mon 17:08
\documentclass[presentation]{beamer}
\usepackage[latin1]{inputenc}
\usepackage[T1]{fontenc}
\usepackage{fixltx2e}
\usepackage{graphicx}
\usepackage{longtable}
\usepackage{float}
\usepackage{wrapfig}
\usepackage{soul}
\usepackage{t1enc}
\usepackage{textcomp}
\usepackage{marvosym}
\usepackage{wasysym}
\usepackage{latexsym}
\usepackage{amssymb}
\usepackage{hyperref}
\tolerance=1000
\usepackage[english]{babel} \usepackage{ae,aecompl}
\usepackage{mathpazo,courier,euler} \usepackage[scaled=.95]{helvet}
\usepackage{listings}
\lstset{language=Python, basicstyle=\ttfamily\bfseries,
commentstyle=\color{red}\itshape, stringstyle=\color{darkgreen},
showstringspaces=false, keywordstyle=\color{blue}\bfseries}
\providecommand{\alert}[1]{\textbf{#1}}

\title{Savefig}
\author{FOSSEE}
\date{2010-10-11 Mon}

\usetheme{Warsaw}\usecolortheme{default}\useoutertheme{infolines}\setbeamercovered{transparent}
\begin{document}

\maketitle









\begin{frame}
\frametitle{Outline}
\label{sec-1}

\begin{itemize}
\item Saving plots.
\item Plotting in different formats.
\item Locating the file in the file system.
\end{itemize}
\end{frame}
\begin{frame}[fragile]
\frametitle{Creating a basic plot}
\label{sec-2}

  Plot a sine wave from -3pi to 3pi.
\begin{verbatim}
In []: x = linspace(-3*pi,3*pi,100)

In []: plot(x, sin(x))
\end{verbatim}
\end{frame}
\begin{frame}[fragile]
\frametitle{savefig()}
\label{sec-3}
\begin{itemize}

\item savefig() - to save plots
\label{sec-3_1}%
\begin{verbatim}
    syntax: savefig(fname)
\end{verbatim}


\item example
\label{sec-3_2}%
\begin{itemize}

\item savefig('/home/fossee/sine.png')
\label{sec-3_2_1}%
\begin{itemize}
\item file sine.png saved to the folder /home/fossee
\item .png - file type
\end{itemize}


\end{itemize} % ends low level
\end{itemize} % ends low level
\end{frame}
\begin{frame}
\frametitle{More on savefig()}
\label{sec-4}
\begin{itemize}

\item Recall
\label{sec-4_1}%
\begin{itemize}
\item .png - file type
\end{itemize}


\item File types supported
\label{sec-4_2}%
\begin{itemize}

\item .pdf - PDF(Portable Document Format)\\
\label{sec-4_2_1}%
\item .ps - PS(Post Script)\\
\label{sec-4_2_2}%
\item .eps - Encapsulated Post Script\\
\label{sec-4_2_3}%
\texttt{to be used with} \LaTeX{} \texttt{documents}

\item .svg - Scalable Vector Graphics\\
\label{sec-4_2_4}%
\texttt{vector graphics}

\item .png - Portable Network Graphics\\
\label{sec-4_2_5}%
\texttt{supports transparency}
\end{itemize} % ends low level
\end{itemize} % ends low level
\end{frame}
\begin{frame}
\frametitle{Exercise 1}
\label{sec-5}

  Save the sine plot in the format EPS which can be embedded in \LaTeX{} documents.
\end{frame}
\begin{frame}[fragile]
\frametitle{Solution 1}
\label{sec-6}

\begin{verbatim}
savefig('/home/fossee/sine.eps')
\end{verbatim}
\end{frame}
\begin{frame}
\frametitle{Exercise 2}
\label{sec-7}

  Save the sine plot in PDF, PS and SVG formats.
\end{frame}
\begin{frame}
\frametitle{Summary}
\label{sec-8}

  You should now be able to
\begin{itemize}
\item Use \texttt{savefig()} function
\item Save plots in different formats

\begin{itemize}
\item PDF
\item PS
\item PNG
\item SVG
\item EPS
\end{itemize}

\item Locating the files in file system.
\end{itemize}

    
\end{frame}
\begin{frame}
\frametitle{Thank you!}
\label{sec-9}

  \begin{block}{}
  \begin{center}
  This spoken tutorial has been produced by the
  \textcolor{blue}{FOSSEE} team, which is funded by the 
  \end{center}
  \begin{center}
    \textcolor{blue}{National Mission on Education through \\
      Information \& Communication Technology \\ 
      MHRD, Govt. of India}.
  \end{center}  
  \end{block}
\end{frame}

\end{document}
