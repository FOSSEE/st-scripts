% Created 2011-06-15 Wed 11:05
\documentclass[presentation]{beamer}
\usepackage[latin1]{inputenc}
\usepackage[T1]{fontenc}
\usepackage{fixltx2e}
\usepackage{graphicx}
\usepackage{longtable}
\usepackage{float}
\usepackage{wrapfig}
\usepackage{soul}
\usepackage{textcomp}
\usepackage{marvosym}
\usepackage{wasysym}
\usepackage{latexsym}
\usepackage{amssymb}
\usepackage{hyperref}
\tolerance=1000
\usepackage[english]{babel} \usepackage{ae,aecompl}
\usepackage{mathpazo,courier,euler} \usepackage[scaled=.95]{helvet}
\usepackage{listings}
\lstset{language=Python, basicstyle=\ttfamily\bfseries,
commentstyle=\color{red}\itshape, stringstyle=\color{darkgreen},
showstringspaces=false, keywordstyle=\color{blue}\bfseries}
\providecommand{\alert}[1]{\textbf{#1}}

\title{}
\author{FOSSEE}
\date{}

\usetheme{Warsaw}\usecolortheme{default}\useoutertheme{infolines}\setbeamercovered{transparent}
\begin{document}











\begin{frame}

\begin{center}
\vspace{12pt}
\textcolor{blue}{\huge Using Sage}
\end{center}
\vspace{18pt}
\begin{center}
\vspace{10pt}
\includegraphics[scale=0.95]{../images/fossee-logo.png}\\
\vspace{5pt}
\scriptsize Developed by FOSSEE Team, IIT-Bombay. \\ 
\scriptsize Funded by National Mission on Education through ICT\\
\scriptsize  MHRD,Govt. of India\\
\includegraphics[scale=0.30]{../images/iitb-logo.png}\\
\end{center}
\end{frame}
\begin{frame}
\frametitle{Objectives}
\label{sec-2}

 At the end of this tutorial, you will be able to,


\begin{itemize}
\item Learn the range of things for which Sage can be used.
\item Know the functions used for calculus in Sage.
\item Learn about graph theory and number theory using Sage.
\end{itemize}
\end{frame}
\begin{frame}
\frametitle{Pre-requisite}
\label{sec-3}

  Spoken tuorial on -

\begin{itemize}
\item Getting started with Sage
\end{itemize}
\end{frame}
\begin{frame}
\frametitle{Equation}
\label{sec-4}

  Ax = v,\\
  where A is the matrix, ``matrix([[1,2],[3,4]])''\\
  v is the vector, ``vector([1,2])''. 
\end{frame}
\begin{frame}
\frametitle{Summary}
\label{sec-5}

In this tutorial, we have learnt to,  

\begin{itemize}
\item Use functions for calculus like --
\begin{itemize}
\item lim()-- to find out the limit of a function
\item integrate()-- to integrate over an expression
\item integral()-- to find out the definite integral of an 
      expression by specifying the limits
\item solve()-- to solve a function, relative to it's postion.
\end{itemize}
\item Create both a simple graph and a directed graph, using the 
    functions ``graph()'' and ``digraph()'' respectively.
\item Use functions for Number theory.For eg:
\begin{itemize}
\item primes$\_{\mathrm{range}}$()-- to find out the prime numbers within the 
      specified range
\item factor()-- to find out the factorized form of the number specified
\item Permutations(), Combinations()-- to obtain the required permutation 
      and combinations for the given set of values.
\end{itemize}
\end{itemize}
\end{frame}
\begin{frame}
\frametitle{Evaluation}
\label{sec-6}


\begin{enumerate}
\item How do you find the limit of the function ``x/sin(x)'' as ``x'' tends 
    to ``0'' from the negative side.
\vspace{3pt}
\item List all the primes between 2009 and 2900.
\vspace{3pt}
\item Solve the system of linear equations
     
    x-2y+3z = 7\\
    2x+3y-z = 5\\
    x+2y+4z = 9
\end{enumerate}
\end{frame}
\begin{frame}
\frametitle{Solutions}
\label{sec-7}


\begin{enumerate}
\item lim(x/sin(x), x=0, dir=``left'')
\vspace{4pt}
\item prime$\_{\mathrm{range}}$(2009, 2901)
\vspace{4pt}
\item A = Matrix([[1, -2, 3], \\
\hspace{1.78cm}
                  [2, 3, -1], \\
\hspace{1.78cm}             
                  [1, 2, 4]]) \\
\vspace{2pt}
   b = vector([7, 5, 9])\\
\vspace{2pt}
   x = A.solve$\_{\mathrm{right}}$(b)\\
\vspace{2pt}
   x   
\end{enumerate}
\end{frame}
\begin{frame}

  \begin{block}{}
  \begin{center}
  \textcolor{blue}{\Large THANK YOU!} 
  \end{center}
  \end{block}
\begin{block}{}
  \begin{center}
    For more Information, visit our website\\
    \url{http://fossee.in/}
  \end{center}  
  \end{block}
\end{frame}

\end{document}