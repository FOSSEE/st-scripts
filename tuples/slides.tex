% Created 2011-07-05 Tue 10:27
\documentclass[presentation]{beamer}
\usepackage[latin1]{inputenc}
\usepackage[T1]{fontenc}
\usepackage{fixltx2e}
\usepackage{graphicx}
\usepackage{longtable}
\usepackage{float}
\usepackage{wrapfig}
\usepackage{soul}
\usepackage{textcomp}
\usepackage{marvosym}
\usepackage{wasysym}
\usepackage{latexsym}
\usepackage{amssymb}
\usepackage{hyperref}
\tolerance=1000
\usepackage[english]{babel} \usepackage{ae,aecompl}
\usepackage{mathpazo,courier,euler} \usepackage[scaled=.95]{helvet}
\usepackage{listings}
\lstset{language=Python, basicstyle=\ttfamily\bfseries,
commentstyle=\color{red}\itshape, stringstyle=\color{darkgreen},
showstringspaces=false, keywordstyle=\color{blue}\bfseries}
\providecommand{\alert}[1]{\textbf{#1}}

\title{}
\author{FOSSEE}
\date{}

\usetheme{Warsaw}\usecolortheme{default}\useoutertheme{infolines}\setbeamercovered{transparent}
\begin{document}











\begin{frame}

\begin{center}
\vspace{12pt}
\textcolor{blue}{\huge Getting started with \texttt{tuples}}
\end{center}
\vspace{18pt}
\begin{center}
\vspace{10pt}
\includegraphics[scale=0.95]{../images/fossee-logo.png}\\
\vspace{5pt}
\scriptsize Developed by FOSSEE Team, IIT-Bombay. \\ 
\scriptsize Funded by National Mission on Education through ICT\\
\scriptsize  MHRD,Govt. of India\\
\includegraphics[scale=0.30]{../images/iitb-logo.png}\\
\end{center}
\end{frame}
\begin{frame}
\frametitle{Objectives}
\label{sec-2}

  At the end of the tutorial, you will be able to,


\begin{itemize}
\item Understand of what tuples are.
\item Compare them with lists.
\item Know why they are needed and where to use them.
\end{itemize}
\end{frame}
\begin{frame}
\frametitle{Pre-requisite}
\label{sec-3}

Spoken tutorial on -

\begin{itemize}
\item Getting started with Lists
\end{itemize}
\end{frame}
\begin{frame}
\frametitle{Exericse 1}
\label{sec-4}

  Given, \verb~a = 5~ and \verb~b = 7~. Swap the values of \verb~a~ and \verb~b~.
\end{frame}
\begin{frame}
\frametitle{Summary}
\label{sec-5}

  In this tutorial, we have learnt to,


\begin{itemize}
\item Define tuples.
\item Understand the similarities of tuples with lists, like indexing and 
    iterability.
\item Know about the immutability of tuples.
\item Swap values, the python way.
\item Understand the concept of packing and unpacking of tuples.
\end{itemize}
\end{frame}
\begin{frame}
\frametitle{Evaluation}
\label{sec-6}


\begin{enumerate}
\item Define a tuple containing two values. The first being integer 4 and 
   second is a float 2.5
\vspace{8pt}
\item If ``a = 5,'' then what is the type of a ?
\begin{itemize}
\item int
\item float
\item tuple
\item string
\end{itemize}
\vspace{8pt}
\item If ``a = (2, 3)'',\\ What does ``a[ 0 ], a[ 1 ] = (3, 4)'' produce ?
\end{enumerate}
\end{frame}
\begin{frame}
\frametitle{Solutions}
\label{sec-7}


\begin{enumerate}
\item (4, 2.5)
\vspace{12pt}
\item tuple
\vspace{12pt}
\item Error
\end{enumerate}
\end{frame}
\begin{frame}

  \begin{block}{}
  \begin{center}
  \textcolor{blue}{\Large THANK YOU!} 
  \end{center}
  \end{block}
\begin{block}{}
  \begin{center}
    For more Information, visit our website\\
    \url{http://fossee.in/}
  \end{center}  
  \end{block}
\end{frame}

\end{document}