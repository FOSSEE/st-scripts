% Created 2011-05-06 Fri 12:18
\documentclass[presentation]{beamer}
\usepackage[latin1]{inputenc}
\usepackage[T1]{fontenc}
\usepackage{fixltx2e}
\usepackage{graphicx}
\usepackage{longtable}
\usepackage{float}
\usepackage{wrapfig}
\usepackage{soul}
\usepackage{textcomp}
\usepackage{marvosym}
\usepackage{wasysym}
\usepackage{latexsym}
\usepackage{amssymb}
\usepackage{hyperref}
\tolerance=1000
\usepackage[english]{babel} \usepackage{ae,aecompl}
\usepackage{mathpazo,courier,euler} \usepackage[scaled=.95]{helvet}
\usepackage{listings}
\lstset{language=Python, basicstyle=\ttfamily\bfseries,
commentstyle=\color{red}\itshape, stringstyle=\color{darkgreen},
showstringspaces=false, keywordstyle=\color{blue}\bfseries}
\providecommand{\alert}[1]{\textbf{#1}}

\title{}
\author{FOSSEE}
\date{}

\usetheme{Warsaw}\usecolortheme{default}\useoutertheme{infolines}\setbeamercovered{transparent}
\begin{document}











\begin{frame}

\begin{center}
\textcolor{blue}{Loading Data from Files}
\end{center}
\begin{center}
\includegraphics[scale=0.25]{../images/iitb-logo.png}\\
Developed by FOSSEE Team, IIT-Bombay. \\ 
Funded by National Mission on Education through ICT

MHRD, Govt. of India
\end{center}
\end{frame}
\begin{frame}
\frametitle{Objectives}
\label{sec-2}

  At the end of this tutorial, you will be able to,

\begin{itemize}
\item Read data from files with a single column of data.
\item Read data from files with multiple columns seperated by
    spaces and other delimiters.
\end{itemize}
\end{frame}
\begin{frame}
\frametitle{Question 1}
\label{sec-3}

  Read the file  \verb~pendulum\_semicolon.txt~  which contains the same data
  as  \verb~pendulum.txt~ , but the columns are seperated by semi-colons instead
  of spaces.Use the IPython help to see how to do this.
\end{frame}
\begin{frame}
\frametitle{Summary}
\label{sec-4}

  In this tutorial,we have learnt to-

\begin{itemize}
\item Read data from files, containing a single column of data using the
    \verb~loadtxt~ command.
\item Read multiple columns of data, separated by spaces or other
    delimiters by adding additional arguments to the \verb~loadtxt~ command.
\end{itemize}
\end{frame}
\begin{frame}
\frametitle{Evaluation}
\label{sec-5}


\begin{enumerate}
\item ``loadtxt`` can read data only from a file with one column only.
     True or False?
\item Given a file ``data.txt`` with three columns of data separated by
     spaces, read it into 3 separate simple sequences.
\item Given a file ``data.txt`` with three columns of data separated by
     ``:'', read it into 3 separate simple sequences.
\end{enumerate}
  
\end{frame}
\begin{frame}
\frametitle{Solutions}
\label{sec-6}


\begin{enumerate}
\item False
\item x = loadtxt(``data.txt'', unpack=True)
\item x = loadtxt(``data.txt'', unpack=True, delimiter='':'')
\end{enumerate}
\end{frame}
\begin{frame}
\frametitle{Acknowledgement}
\label{sec-7}

   \begin{block}{}
  \begin{center}
  \textcolor{blue}{\Large THANK YOU!} 
  \end{center}
  \end{block}
\begin{block}{}
  \begin{center}
    For more Information, visit our website\\
    \url{http://fossee.in/}
  \end{center}  
  \end{block}
\end{frame}

\end{document}